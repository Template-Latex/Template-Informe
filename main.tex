% Template:     Informe/Reporte LaTeX
% Documento:    Archivo principal
% Versión:      5.3.6 (21/05/2018)
% Codificación: UTF-8
%
% Autor: Pablo Pizarro R. @ppizarror
%        Facultad de Ciencias Físicas y Matemáticas
%        Universidad de Chile
%        pablo.pizarro@ing.uchile.cl, ppizarror.com
%
% Manual template: [http://latex.ppizarror.com/Template-Informe/]
% Licencia MIT:    [https://opensource.org/licenses/MIT/]

% CREACIÓN DEL DOCUMENTO
\documentclass[letterpaper,11pt]{article} % Articulo tamaño carta, 11pt
\usepackage[utf8]{inputenc} % Codificación UTF-8

% INFORMACIÓN DEL DOCUMENTO
\def\titulodelinforme {Título del informe}
\def\temaatratar {Tema a tratar}

\def\autordeldocumento {Nombre del autor}
\def\nombredelcurso {Curso}
\def\codigodelcurso {CO-1234}

\def\nombreuniversidad {Universidad de Chile}
\def\nombrefacultad {Facultad de Ciencias Físicas y Matemáticas}
\def\departamentouniversidad {Departamento de la Universidad}
\def\imagendepartamento {departamentos/fcfm}
\def\imagendepartamentoescala {0.2}
\def\localizacionuniversidad {Santiago, Chile}

% INTEGRANTES, PROFESORES Y FECHAS
\def\tablaintegrantes {
\begin{tabular}{ll}
	Integrantes:
	& \begin{tabular}[t]{@{}l@{}}
		Integrante 1 \\
		Integrante 2
	\end{tabular} \\
	Profesor:
	& \begin{tabular}[t]{@{}l@{}}
		Profesor 1
	\end{tabular} \\
	Auxiliares:
	& \begin{tabular}[t]{@{}l@{}}
		Auxiliar 1 \\
		Auxiliar 2
	\end{tabular} \\
	Ayudantes:
	& \begin{tabular}[t]{@{}l@{}}
		Ayudante 1 \\
		Ayudante 2
	\end{tabular} \\
	\multicolumn{2}{l}{Ayudante del laboratorio: Ayudante 1} \\
	& \\
	\multicolumn{2}{l}{Fecha de realización: \today} \\
	\multicolumn{2}{l}{Fecha de entrega: \today} \\
	\multicolumn{2}{l}{\localizacionuniversidad}
\end{tabular}}{
}

% CONFIGURACIONES
% Template:     Informe/Reporte LaTeX
% Documento:    Configuraciones del template
% Versión:      4.1.1 (04/07/2017)
% Codificación: UTF-8
%
% Autor: Pablo Pizarro R.
%        Facultad de Ciencias Físicas y Matemáticas
%        Universidad de Chile
%        pablo.pizarro@ing.uchile.cl, ppizarror.com
%
% Manual template: [http://ppizarror.com/Template-Informe/]
% Licencia MIT:    [https://opensource.org/licenses/MIT/]

% CONFIGURACIONES GENERALES
\def\addemptypagetwosides {false} % Añade pags. en blanco al imprimir por ambas caras
\def\defaultimagefolder {images/} % Carpeta raíz de las imágenes
\def\defaultinterline {1.0}       % Interlineado por defecto [pt]
\def\defaultnewlinesize {11.0}    % Tamaño del salto de línea [pt]
\def\fontdocument {lmodern}       % Tipografía del documento (lmodern,arial,helvet)
\def\numberedequation {true}      % Ecuaciones con \insert... numeradas
\def\pointdecimal {false}         % Decimales con punto en vez de coma
\def\romanpageuppercase {true}    % Páginas en número romano en mayúsculas
\def\showdotontitles {true}       % Punto al final de cada número de título/subtítulo
\def\tablepadding {1.0}           % Ancho de celda de las tablas


% CONFIGURACIÓN DE LAS LEYENDAS - CAPTION
\def\captionlessmarginimage {0.1} % Margen sup/inf de fig. si no hay leyenda [cm]
\def\captionlrmargin {2.0}        % Márgenes izq/der de la leyenda [cm]
\def\captionalignment {justified} % Alineación leyenda: justified,centered,left,right
\def\captiontbmarginfigure {9.35} % Margen sup/inf de la leyenda en figuras [pt]
\def\captiontbmargintable {7.0}   % Margen sup/inf de la leyenda en tablas [pt]
\def\captiontextbold {false}      % Etiquetas (Figura,Tabla,Código) en negrita
\def\codecaptiontop {true}        % Leyenda arriba del código fuente
\def\figurecaptiontop {false}     % Leyenda arriba de las imágenes
\def\showsectiononcaption {false} % Muestra el número de sección en las leyendas
\def\tablecaptiontop {true}       % Leyenda arriba de las tablas


% CONFIGURACIÓN DEL ÍNDICE
\def\indexdepth {3}               % Profundidad máxima del índice
\def\indextitlemargin {7.0}       % Margen título en índice \insertindextitle [pt]
\def\showindex {true}             % Muestra el índice
\def\showindexofcode {true}       % Muestra la lista de códigos fuente
\def\showindexofcontents {true}   % Muestra la lista de contenidos
\def\showindexoffigures {true}    % Muestra la lista de figuras
\def\showindexoftables {true}     % Muestra la lista de tablas


% CONFIGURACIÓN DE LOS COLORES DEL DOCUMENTO
\def\captioncolor {black}         % Color de la etiqueta (Figura, Tabla, Código)
\def\captiontextcolor {black}     % Color de la leyenda
\def\citecolor {black}            % Color del número de las referencias o citas
\def\highlightcolor {yellow}      % Color del subrayado con \hl
\def\indextitlecolor {black}      % Color de los títulos del índice
\def\linkcolor {black}            % Color de los links del doc.
\def\maintextcolor {black}        % Color principal del texto
\def\portraittitlecolor {black}   % Color de los títulos de la portada
\def\showborderonlinks {false}    % Color de un links por un recuadro de color
\def\subsubtitlecolor {black}     % Color de los sub-subtítulos
\def\subtitlecolor {black}        % Color de los subtítulos
\def\tablelinecolor {black}       % Color de las líneas de las tablas
\def\titlecolor {black}           % Color de los títulos
\def\urlcolor {magenta}           % Color de los enlaces web (\url,\href)


% MÁRGENES DE FIGURAS
\def\marginbottomimages {-0.2}    % Margen inferior figura [cm]
\def\marginfloatimages {-13.0}    % Margen sup. fig flot. \insertimageleft/right [pt]
\def\margintopimages {0.0}        % Margen superior figura [cm]


% CITAS Y REFERENCIAS
\def\apaciterefsep {9}            % Separación entre referencias [pt] {apacite}
\def\bibtexrefsep {9}             % Separación entre referencias [pt] {bibtex}
\def\natbibrefsep {5}             % Separación entre referencias [pt] {natbib}
\def\referencenumsection {true}   % Sección de referencias numerada
\def\stylecitereferences {bibtex} % Estilo citas y referencias (bibtex,apacite,natbib)
\def\twocolumnreferences {false}  % Referencias en dos columnas


% CONFIGURACIÓN PORTADA Y HEADERS
\def\gradecodeonportrait {false}  % Muestra el código del curso en la portada
\def\showfooter {true}            % Muestra el footer
\def\showheadertitle {true}       % Muestra título de la sección en el header


% MÁRGENES DE PÁGINA
\def\firstpagemargintop {3.8}     % Margen superior página portada [cm]
\def\pagemarginbottom {2.7}       % Margen inferior página [cm]
\def\pagemarginleft {2.54}        % Margen izquierdo página [cm]
\def\pagemarginright {2.54}       % Margen derecho página [cm]
\def\pagemargintop {3.0}          % Margen superior página [cm]


% ESTILO Y TAMAÑO DE TÍTULOS
\def\fontsizesubsubtitle {\large} % Tamaño sub-subtítulos
\def\fontsizesubtitle {\Large}    % Tamaño subtítulos
\def\fontsizetitle {\huge}        % Tamaño títulos
\def\fontsizetitlei {\huge}       % Tamaño títulos en el índice
\def\stylesubsubtitle {\bfseries} % Estilo sub-subtítulos
\def\stylesubtitle {\bfseries}    % Estilo subtítulos
\def\styletitle {\bfseries}       % Estilo títulos
\def\styletitlei {\bfseries}      % Estilo títulos en el índice


% OPCIONES DEL PDF COMPILADO
\def\cfgbookmarksopenlevel {1}    % Nivel de los marcadores a mostrar (1: secciones)
\def\cfgpdfbookmarkopen {true}    % Abre el panel de marcadores al abrir el pdf
\def\cfgpdfcenterwindow {true}    % Centra la ventana del lector al abrir el pdf
\def\cfgpdfdisplaydoctitle {true} % Muestra el título del informe como título del pdf
\def\cfgpdffitwindow {false}      % Ajusta la ventana del lector al tamaño del pdf
\def\cfgpdftoolbar {true}         % Muestra la barra de herramientas del lector pdf


% NOMBRE DE OBJETOS
\def\nameabstract {Resumen}           % Nombre del resumen-abstract
\def\nameportraitpage {Portada}       % Etiqueta página de la portada
\def\namereferences {Referencias}     % Nombre de la sección de referencias
\def\nomltcont {Índice de Contenidos} % Nombre del índice de contenidos
\def\nomltfigure {Lista de Figuras}   % Nombre del índice de la lista de figuras
\def\nomltsrc {Lista de Códigos}      % Nombre del índice de la lista de código
\def\nomlttable {Lista de Tablas}     % Nombre del índice de la lista de tablas
\def\nomltwsrc {Código}               % Etiqueta leyenda del código fuente
\def\nomltwfigure {Figura}            % Etiqueta leyenda de las figuras
\def\nomltwtable {Tabla}              % Etiqueta leyenda de las tablas


% IMPORTACIÓN DE LIBRERÍAS
% Template:     Informe/Reporte LaTeX
% Documento:    Importación de librerías
% Versión:      6.0.4 (29/10/2018)
% Codificación: UTF-8
%
% Autor: Pablo Pizarro R. @ppizarror
%        Facultad de Ciencias Físicas y Matemáticas
%        Universidad de Chile
%        pablo.pizarro@ing.uchile.cl, ppizarror.com
%
% Manual template: [https://latex.ppizarror.com/Template-Informe/]
% Licencia MIT:    [https://opensource.org/licenses/MIT/]

% Lanza un mensaje de error indicando mala configuración
%	#1	Mensaje de error
% 	#2	Configuración usada
%	#3	Valores esperados
\newcommand{\throwbadconfig}[3]{
	\errmessage{LaTeX Warning: #1 \noexpand #2=#2. Valores esperados: #3}
	\stop
}

% -----------------------------------------------------------------------------
% LIBRERÍAS IMPORTANTES
% -----------------------------------------------------------------------------
\usepackage[spanish,es-nosectiondot,es-lcroman,es-noquoting]{babel} % Define el idioma
\usepackage{ifthen} % Manejo de condicionales

% -----------------------------------------------------------------------------
% PARCHES DE LIBRERÍAS
% -----------------------------------------------------------------------------
\let\counterwithout\relax
\let\counterwithin\relax

% -----------------------------------------------------------------------------
% TIPOGRAFÍA DEL DOCUMENTO
% -----------------------------------------------------------------------------
\ifthenelse{\equal{\fontdocument}{lmodern}}{ % Default
	\usepackage{lmodern}
}{
\ifthenelse{\equal{\fontdocument}{arial}}{
	\usepackage{helvet}
	\renewcommand{\familydefault}{\sfdefault}
}{
\ifthenelse{\equal{\fontdocument}{accantis}}{
	\usepackage{accanthis}
}{
\ifthenelse{\equal{\fontdocument}{alegreya}}{
	\usepackage{Alegreya}
	\renewcommand*\oldstylenums[1]{{\AlegreyaOsF #1}}
}{
\ifthenelse{\equal{\fontdocument}{alegreyasans}}{
	\usepackage[sfdefault]{AlegreyaSans}
	\renewcommand*\oldstylenums[1]{{\AlegreyaSansOsF #1}}
}{
\ifthenelse{\equal{\fontdocument}{algolrevived}}{
	\usepackage{algolrevived}
}{
\ifthenelse{\equal{\fontdocument}{antiqua}}{
	\usepackage{antiqua}
}{
\ifthenelse{\equal{\fontdocument}{antpolt}}{
	\usepackage{antpolt}
}{
\ifthenelse{\equal{\fontdocument}{antpoltlight}}{
	\usepackage[light]{antpolt}
}{
\ifthenelse{\equal{\fontdocument}{anttor}}{
	\usepackage[math]{anttor}
}{
\ifthenelse{\equal{\fontdocument}{anttorcondensed}}{
	\usepackage[condensed,math]{anttor}
}{
\ifthenelse{\equal{\fontdocument}{anttorlight}}{
	\usepackage[light,math]{anttor}
}{
\ifthenelse{\equal{\fontdocument}{anttorlightcondensed}}{
	\usepackage[light,condensed,math]{anttor}
}{
\ifthenelse{\equal{\fontdocument}{arev}}{
	\usepackage{arev}
}{
\ifthenelse{\equal{\fontdocument}{arimo}}{
	\usepackage[sfdefault]{arimo}
	\renewcommand*\familydefault{\sfdefault}
}{
\ifthenelse{\equal{\fontdocument}{aurical}}{
	\usepackage{aurical}
}{
\ifthenelse{\equal{\fontdocument}{avant}}{
	\usepackage{avant}
}{
\ifthenelse{\equal{\fontdocument}{baskervald}}{
	\usepackage{baskervald}
}{
\ifthenelse{\equal{\fontdocument}{berasans}}{
	\usepackage[scaled]{berasans}
	\renewcommand*\familydefault{\sfdefault}
}{
\ifthenelse{\equal{\fontdocument}{beraserif}}{
	\usepackage{bera}
}{
\ifthenelse{\equal{\fontdocument}{biolinum}}{
	\usepackage{libertine}
	\renewcommand*\familydefault{\sfdefault}
}{
\ifthenelse{\equal{\fontdocument}{cabin}}{
	\usepackage[sfdefault]{cabin}
	\renewcommand*\familydefault{\sfdefault}
}{
\ifthenelse{\equal{\fontdocument}{cabincondensed}}{
	\usepackage[sfdefault,condensed]{cabin}
	\renewcommand*\familydefault{\sfdefault}
}{
\ifthenelse{\equal{\fontdocument}{cantarell}}{
	\usepackage[default]{cantarell}
}{
\ifthenelse{\equal{\fontdocument}{caladea}}{
	\usepackage{caladea}
}{
\ifthenelse{\equal{\fontdocument}{carlito}}{
	\usepackage[sfdefault]{carlito}
	\renewcommand*\familydefault{\sfdefault}
}{
\ifthenelse{\equal{\fontdocument}{chivolight}}{
	\usepackage[familydefault,light]{Chivo}
}{
\ifthenelse{\equal{\fontdocument}{chivoregular}}{
	\usepackage[familydefault,regular]{Chivo}
}{
\ifthenelse{\equal{\fontdocument}{clearsans}}{
	\usepackage[sfdefault]{ClearSans}
	\renewcommand*\familydefault{\sfdefault}
}{
\ifthenelse{\equal{\fontdocument}{comfortaa}}{
	\usepackage[default]{comfortaa}
}{
\ifthenelse{\equal{\fontdocument}{comicneue}}{
	\usepackage[default]{comicneue}
}{
\ifthenelse{\equal{\fontdocument}{comicneueangular}}{
	\usepackage[default,angular]{comicneue}
}{
\ifthenelse{\equal{\fontdocument}{crimson}}{
	\usepackage{crimson}
}{
\ifthenelse{\equal{\fontdocument}{cyklop}}{
	\usepackage{cyklop}
}{
\ifthenelse{\equal{\fontdocument}{dejavusans}}{
	\usepackage{DejaVuSans}
	\renewcommand*\familydefault{\sfdefault}
}{
\ifthenelse{\equal{\fontdocument}{dejavusanscondensed}}{
	\usepackage{DejaVuSansCondensed}
	\renewcommand*\familydefault{\sfdefault}
}{
\ifthenelse{\equal{\fontdocument}{droidsans}}{
	\usepackage[defaultsans]{droidsans}
	\renewcommand*\familydefault{\sfdefault}
}{
\ifthenelse{\equal{\fontdocument}{fetamont}}{ % Falta implementar aún
	\usepackage{fetamont}
	\renewcommand*\familydefault{\sfdefault}
}{
\ifthenelse{\equal{\fontdocument}{firasans}}{
	\usepackage[sfdefault]{FiraSans}
	\renewcommand*\familydefault{\sfdefault}
}{
\ifthenelse{\equal{\fontdocument}{iwona}}{
	\usepackage[math]{iwona}
}{
\ifthenelse{\equal{\fontdocument}{iwonacondensed}}{
	\usepackage[math]{iwona}
}{
\ifthenelse{\equal{\fontdocument}{iwonalight}}{
	\usepackage[light,math]{iwona}
}{
\ifthenelse{\equal{\fontdocument}{iwonalightcondensed}}{
	\usepackage[light,condensed,math]{iwona}
}{
\ifthenelse{\equal{\fontdocument}{kurier}}{
	\usepackage[math]{kurier}
}{
\ifthenelse{\equal{\fontdocument}{kuriercondensed}}{
	\usepackage[condensed,math]{kurier}
}{
\ifthenelse{\equal{\fontdocument}{kurierlight}}{
	\usepackage[light,math]{kurier}
}{
\ifthenelse{\equal{\fontdocument}{kurierlightcondensed}}{
	\usepackage[light,condensed,math]{kurier}
}{
\ifthenelse{\equal{\fontdocument}{lato}}{
	\usepackage[default]{lato}
}{
\ifthenelse{\equal{\fontdocument}{libris}}{
	\usepackage{libris}
	\renewcommand*\familydefault{\sfdefault}
}{
\ifthenelse{\equal{\fontdocument}{lxfonts}}{
	\usepackage{lxfonts}
}{
\ifthenelse{\equal{\fontdocument}{merriweather}}{
	\usepackage[sfdefault]{merriweather}
}{
\ifthenelse{\equal{\fontdocument}{merriweatherlight}}{
	\usepackage[sfdefault,light]{merriweather}
}{
\ifthenelse{\equal{\fontdocument}{mintspirit}}{
	\usepackage[default]{mintspirit}
}{
\ifthenelse{\equal{\fontdocument}{montserratalternatesextralight}}{
	\usepackage[defaultfam,extralight,tabular,lining,alternates]{montserrat}
	\renewcommand*\oldstylenums[1]{{\fontfamily{Montserrat-TOsF}\selectfont #1}}
}{
\ifthenelse{\equal{\fontdocument}{montserratalternatesregular}}{
	\usepackage[defaultfam,tabular,lining,alternates]{montserrat}
	\renewcommand*\oldstylenums[1]{{\fontfamily{Montserrat-TOsF}\selectfont #1}}
}{
\ifthenelse{\equal{\fontdocument}{montserratalternatesthin}}{
	\usepackage[defaultfam,thin,tabular,lining,alternates]{montserrat}
	\renewcommand*\oldstylenums[1]{{\fontfamily{Montserrat-TOsF}\selectfont #1}}
}{
\ifthenelse{\equal{\fontdocument}{montserratextralight}}{
	\usepackage[defaultfam,extralight,tabular,lining]{montserrat}
	\renewcommand*\oldstylenums[1]{{\fontfamily{Montserrat-TOsF}\selectfont #1}}
}{
\ifthenelse{\equal{\fontdocument}{montserratlight}}{
	\usepackage[defaultfam,light,tabular,lining]{montserrat}
	\renewcommand*\oldstylenums[1]{{\fontfamily{Montserrat-TOsF}\selectfont #1}}
}{
\ifthenelse{\equal{\fontdocument}{montserratregular}}{
	\usepackage[defaultfam,tabular,lining]{montserrat}
	\renewcommand*\oldstylenums[1]{{\fontfamily{Montserrat-TOsF}\selectfont #1}}
}{
\ifthenelse{\equal{\fontdocument}{montserratthin}}{
	\usepackage[defaultfam,thin,tabular,lining]{montserrat}
	\renewcommand*\oldstylenums[1]{{\fontfamily{Montserrat-TOsF}\selectfont #1}}
}{
\ifthenelse{\equal{\fontdocument}{nimbussans}}{
	\usepackage{nimbussans}
	\renewcommand*\familydefault{\sfdefault}
}{
\ifthenelse{\equal{\fontdocument}{noto}}{
	\usepackage[sfdefault]{noto}
	\renewcommand*\familydefault{\sfdefault}
}{
\ifthenelse{\equal{\fontdocument}{opensans}}{
	\usepackage[default,osfigures,scale=0.95]{opensans}
}{
\ifthenelse{\equal{\fontdocument}{overlock}}{
	\usepackage[sfdefault]{overlock}
	\renewcommand*\familydefault{\sfdefault}
}{
\ifthenelse{\equal{\fontdocument}{paratype}}{
	\usepackage{paratype}
	\renewcommand*\familydefault{\sfdefault}
}{
\ifthenelse{\equal{\fontdocument}{paratypesanscaption}}{
	\usepackage{PTSansCaption}
	\renewcommand*\familydefault{\sfdefault}
}{
\ifthenelse{\equal{\fontdocument}{paratypesansnarrow}}{
	\usepackage{PTSansNarrow}
	\renewcommand*\familydefault{\sfdefault}
}{
\ifthenelse{\equal{\fontdocument}{quattrocento}}{
	\usepackage[sfdefault]{quattrocento}
}{
\ifthenelse{\equal{\fontdocument}{raleway}}{
	\usepackage[default]{raleway}
}{
\ifthenelse{\equal{\fontdocument}{roboto}}{
	\usepackage[sfdefault]{roboto}
}{
\ifthenelse{\equal{\fontdocument}{robotocondensed}}{
	\usepackage[sfdefault,condensed]{roboto}
}{
\ifthenelse{\equal{\fontdocument}{robotolight}}{
	\usepackage[sfdefault,light]{roboto}
}{
\ifthenelse{\equal{\fontdocument}{robotolightcondensed}}{
	\usepackage[sfdefault,light,condensed]{roboto}
}{
\ifthenelse{\equal{\fontdocument}{robotothin}}{
	\usepackage[sfdefault,thin]{roboto}
}{
\ifthenelse{\equal{\fontdocument}{rosario}}{
	\usepackage[familydefault]{Rosario}
}{
\ifthenelse{\equal{\fontdocument}{sourcesanspro}}{
	\usepackage[default]{sourcesanspro}
}{
\ifthenelse{\equal{\fontdocument}{uarial}}{
	\usepackage{uarial}
	\renewcommand*\familydefault{\sfdefault}
}{
\ifthenelse{\equal{\fontdocument}{ugq}}{
	\renewcommand*\sfdefault{ugq}
	\renewcommand*\familydefault{\sfdefault}
}{
\ifthenelse{\equal{\fontdocument}{universalis}}{
	\usepackage[sfdefault]{universalis}
}{
\ifthenelse{\equal{\fontdocument}{universaliscondensed}}{
	\usepackage[condensed,sfdefault]{universalis}
}{
\ifthenelse{\equal{\fontdocument}{venturis}}{
	\usepackage[lf]{venturis}
	\renewcommand*\familydefault{\sfdefault}
}{
	\throwbadconfig{Fuente desconocida}{\fontdocument}{lmodern,arial,helvet,
	accantis,alegreya,alegreyasans,algolrevived,antiqua,antpolt,antpoltlight,
	anttor,anttorcondensed,anttorlight,anttorlightcondensed,arev,arimo,aurical,
	avant,baskervald,berasans,beraserif,biolinum,cabin,cabincondensed,cantarell,
	caladea,carlito,chivolight,chivoregular,clearsans,comfortaa,comicneue,
	comicneueangular,crimson,cyklop,dejavusans,dejavusanscondensed,droidsans,
	firasans,iwona,iwonacondensed,iwonalight,iwonalightcondensed,kurier,
	kuriercondensed,kurierlight,kurierlightcondensed,lato,libris,lxfonts,
	merriweather,merriweatherlight,mintspirit,montserratalternatesextralight,
	montserratalternatesregular,montserratalternatesthin,montserratextralight,
	montserratlight,montserratregular,montserratthin,nimbussans,noto,opensans,
	overlock,paratype,paratypesanscaption,paratypesansnarrow,quattrocento,
	raleway,roboto,robotolight,robotolightcondensed,robotothin,rosario,
	sourcesanspro,uarial,ugq,universalis,universaliscondensed,venturis}
	}}}}}}}}}}}}}}}}}}}}}}}}}}}}}}}}}}}}}}}}}}}}}}}}}}}}}}}}}}}}}}}}}}}}}}}}}}}}}}}}
}

% -----------------------------------------------------------------------------
% TIPOGRAFÍA TYPEWRITER
% -----------------------------------------------------------------------------
\ifthenelse{\equal{\fonttypewriter}{tmodern}}{ % Default
	\renewcommand*\ttdefault{lmvtt}
}{
\ifthenelse{\equal{\fonttypewriter}{anonymouspro}}{
	\usepackage[ttdefault=true]{AnonymousPro}
}{
\ifthenelse{\equal{\fonttypewriter}{ascii}}{
	\usepackage{ascii}
	\let\SI\relax
}{
\ifthenelse{\equal{\fonttypewriter}{beramono}}{
	\usepackage[scaled]{beramono}
}{
\ifthenelse{\equal{\fonttypewriter}{cmpica}}{
	\usepackage{addfont}
	\addfont{OT1}{cmpica}{\pica}
	\addfont{OT1}{cmpicab}{\picab}
	\addfont{OT1}{cmpicati}{\picati}
	\renewcommand*\ttdefault{pica}
}{
\ifthenelse{\equal{\fonttypewriter}{courier}}{
	\usepackage{courier}
}{
\ifthenelse{\equal{\fonttypewriter}{dejavusansmono}}{
	\usepackage[scaled]{DejaVuSansMono}
}{
\ifthenelse{\equal{\fonttypewriter}{firamono}}{
	\usepackage[scale=0.85]{FiraMono}
}{
\ifthenelse{\equal{\fonttypewriter}{gomono}}{
	\usepackage[scale=0.85]{GoMono}
}{
\ifthenelse{\equal{\fonttypewriter}{inconsolata}}{
	\usepackage{inconsolata}
}{
\ifthenelse{\equal{\fonttypewriter}{nimbusmono}}{
	\usepackage{nimbusmono}
}{
\ifthenelse{\equal{\fonttypewriter}{newtxtt}}{
	\usepackage[zerostyle=d]{newtxtt}
}{
\ifthenelse{\equal{\fonttypewriter}{nimbusmono}}{
	\usepackage{nimbusmono}
}{
\ifthenelse{\equal{\fonttypewriter}{nimbusmononarrow}}{
	\usepackage{nimbusmononarrow}
}{
\ifthenelse{\equal{\fonttypewriter}{lcmtt}}{
	\renewcommand*\ttdefault{lcmtt}
}{
\ifthenelse{\equal{\fonttypewriter}{sourcecodepro}}{
	\usepackage[ttdefault=true,scale=0.85]{sourcecodepro}
}{
\ifthenelse{\equal{\fonttypewriter}{texgyrecursor}}{
	\usepackage{tgcursor}
}{
	\throwbadconfig{Fuente desconocida}{\fonttypewriter}{anonymouspro,ascii,beramono,cmpica,courier,dejavusansmono,firamono,gomono,inconsolata,kpmonospaced,lcmtt,newtxtt,nimbusmono,nimbusmononarrow,texgyrecursor,tmodern}
	}}}}}}}}}}}}}}}}
}
\usepackage[T1]{fontenc} % Caracteres acentuados
\ifthenelse{\equal{\showlinenumbers}{true}}{ % Muestra los números de línea
	\usepackage[switch,columnwise,running]{lineno}}{
}

% -----------------------------------------------------------------------------
% LIBRERÍAS INDEPENDIENTES
% -----------------------------------------------------------------------------
\usepackage{amsmath}       % Librerías matemáticas
\usepackage{amssymb}       % Librerías matemáticas
\usepackage{array}         % Nuevas características a las tablas
\usepackage{bigstrut}      % Líneas horizontales en tablas
\usepackage{bm}            % Caracteres en negrita en ecuaciones
\usepackage{booktabs}      % Permite manejar elementos visuales en tablas
\usepackage{caption}       % Leyendas
\usepackage{changepage}    % Condicionales para administrar páginas
\usepackage{chngcntr}      % Añade números a las leyendas
\usepackage{color}         % Colores
\usepackage{colortbl}      % Administración de color en las tablas
\usepackage{csquotes}      % Citas y comillas
\usepackage{datetime}      % Fechas
\usepackage{floatpag}      % Maneja números de páginas
\usepackage{floatrow}      % Permite adminisrar posiciones en los caption
\usepackage{framed}        % Permite creación de recuadros
\usepackage{gensymb}       % Simbología común
\usepackage{geometry}      % Dimensiones y geometría del documento
\usepackage{graphicx}      % Propiedades extra para los gráficos
\usepackage{lipsum}        % Permite crear párrafos de prueba
\usepackage{listings}      % Permite añadir código fuente
\usepackage{listingsutf8}  % Acepta codificación UTF-8 en código fuente
\usepackage{longtable}     % Permite utilizar tablas en varias hojas
\usepackage{mathtools}     % Permite utilizar notaciones matemáticas
\usepackage{multicol}      % Múltiples columnas
\usepackage{needspace}     % Maneja los espacios en página
\usepackage{pdflscape}     % Modo página horizontal de página
\usepackage{pdfpages}      % Permite administrar páginas en pdf
\usepackage{physics}       % Paquete de matemáticas
\usepackage{rotating}      % Permite rotación de objetos
\usepackage{sectsty}       % Cambia el estilo de los títulos
\usepackage{selinput}      % Compatibilidad con acentos
\usepackage{setspace}      % Cambia el espacio entre líneas
\usepackage{siunitx}       % Unidades del sistema internacional
\usepackage{soul}          % Permite subrayar texto
\usepackage{subfig}        % Permite agrupar imágenes
\usepackage{textcomp}      % Simbología común
\usepackage{url}           % Permite añadir enlaces
\usepackage{wasysym}       % Contiene caracteres misceláneos
\usepackage{wrapfig}       % Permite comprimir imágenes
\usepackage{xspace}        % Adminsitra espacios en párrafos y líneas

% -----------------------------------------------------------------------------
% LIBRERÍAS CON PARÁMETROS
% -----------------------------------------------------------------------------
\usepackage[makeroom]{cancel} % Cancelar términos en fórmulas
\usepackage[inline]{enumitem} % Permite enumerar ítems
\usepackage[bottom,norule,hang]{footmisc} % Estilo pie de página
\usepackage[subfigure,titles]{tocloft} % Maneja entradas en el índice
\usepackage[pdfencoding=auto,psdextra]{hyperref} % Enlaces, referencias
\usepackage[figure,table,lstlisting]{totalcount} % Contador de objetos
\usepackage[normalem]{ulem} % Permite tachar y subrayar
\usepackage[usenames,dvipsnames]{xcolor} % Paquete de colores avanzado

% -----------------------------------------------------------------------------
% LIBRERÍAS CONDICIONALES
% -----------------------------------------------------------------------------
\ifthenelse{\equal{\showdotontitles}{true}}{ % Agrega punto a títulos/subtítulos
	\usepackage{secdot}
	\sectiondot{subsection}
	\sectiondot{subsubsection}}{
}
\ifthenelse{\equal{\stylecitereferences}{natbib}}{ % Formato citas natbib
	\ifthenelse{\equal{\natbibrefstyle}{apa}}{
		\usepackage{natbib}
	}{
	\ifthenelse{\equal{\natbibrefstyle}{ieeetr}}{
		\usepackage[numbers]{natbib}
	}{
	\ifthenelse{\equal{\natbibrefstyle}{unsrt}}{
		\usepackage[numbers]{natbib}
	}{
	\ifthenelse{\equal{\natbibrefstyle}{abbrvnat}}{
		\usepackage[numbers]{natbib}
	}{
		\usepackage{natbib}
	}}}}
}{
	\ifthenelse{\equal{\stylecitereferences}{apacite}}{ % Formato citas apacite
		\usepackage{apacite}
	}{
		\ifthenelse{\equal{\stylecitereferences}{bibtex}}{ % Formato citas bibtex
		}{}
	}
}
\ifthenelse{\equal{\showappendixsecindex}{true}}{ % Anexos/Apéndices
	\usepackage[toc]{appendix}
}{
	\usepackage{appendix}
}
\ifthenelse{\equal{\importtikz}{true}}{ % Importa la librería tikz
	\usepackage{tikz}}{
}

% -----------------------------------------------------------------------------
% ESTILO PORTADA
% -----------------------------------------------------------------------------
\ifthenelse{\equal{\hfstyle}{style11}}{
	\usepackage{lastpage}}{}
\ifthenelse{\equal{\hfstyle}{style12}}{
	\usepackage{lastpage}}{}
\ifthenelse{\equal{\hfstyle}{style13}}{
	\usepackage{lastpage}}{}
\ifthenelse{\equal{\hfstyle}{style14}}{
	\usepackage{lastpage}}{
}

% -----------------------------------------------------------------------------
% LIBRERÍAS DEPENDIENTES
% -----------------------------------------------------------------------------
\usepackage{bookmark}      % Administración de marcadores en pdf
\usepackage{fancyhdr}      % Encabezados y pie de páginas
\usepackage{float}         % Administrador de posiciones de objetos
\usepackage{hyperxmp}      % Etiquetas opcionales para el pdf compilado
\usepackage{multirow}      % Agrega nuevas opciones a las tablas
\usepackage{notoccite}     % Desactiva las citas en el índice
\usepackage{titlesec}      % Administración de títulos

% END

% IMPORTACIÓN DE FUNCIONES Y ENTORNOS
% Template:     Informe/Reporte LaTeX
% Documento:    Estilos del template
% Versión:      6.0.7 (02/11/2018)
% Codificación: UTF-8
%
% Autor: Pablo Pizarro R. @ppizarror
%        Facultad de Ciencias Físicas y Matemáticas
%        Universidad de Chile
%        pablo.pizarro@ing.uchile.cl, ppizarror.com
%
% Manual template: [https://latex.ppizarror.com/Template-Informe/]
% Licencia MIT:    [https://opensource.org/licenses/MIT/]

% Template:     Informe/Reporte LaTeX
% Documento:    Definición de colores
% Versión:      6.1.3 (07/12/2018)
% Codificación: UTF-8
%
% Autor: Pablo Pizarro R. @ppizarror
%        Facultad de Ciencias Físicas y Matemáticas
%        Universidad de Chile
%        pablo.pizarro@ing.uchile.cl, ppizarror.com
%
% Manual template: [https://latex.ppizarror.com/Template-Informe/]
% Licencia MIT:    [https://opensource.org/licenses/MIT/]

\colorlet{numb}{magenta!60!black}
\colorlet{punct}{red!60!black}
\definecolor{delim}{RGB}{20,105,176}
\definecolor{dkcyan}{RGB}{0,123,167}
\definecolor{dkgray}{rgb}{0.35,0.35,0.35}
\definecolor{dkgreen}{rgb}{0,0.6,0}
\definecolor{dkyellow}{cmyk}{0,0,0.8,0.3}
\definecolor{gray}{rgb}{0.5,0.5,0.5}
\definecolor{lbrown}{RGB}{255,252,249}
\definecolor{lgray}{RGB}{240,240,240}
\definecolor{lyellow}{rgb}{1.0,1.0,0.88}
\definecolor{mauve}{rgb}{0.58,0,0.82}
\definecolor{mygray}{rgb}{0.5,0.5,0.5}
\definecolor{ocher}{rgb}{1,0.5,0}
\definecolor{ocre}{RGB}{243,102,25}

% END
% Template:     Informe/Reporte LaTeX
% Documento:    Estilos de código fuente
% Versión:      5.0.4 (25/04/2018)
% Codificación: UTF-8
%
% Autor: Pablo Pizarro R. @ppizarror
%        Facultad de Ciencias Físicas y Matemáticas
%        Universidad de Chile
%        pablo.pizarro@ing.uchile.cl, ppizarror.com
%
% Manual template: [http://latex.ppizarror.com/Template-Informe/]
% Licencia MIT:    [https://opensource.org/licenses/MIT/]

% Lenguaje C
\lstdefinestyle{c}{
	language=C,
	breakatwhitespace=false,
	breaklines=true,
	columns=flexible,
	commentstyle=\color{mygreen},
	keepspaces=true,
	keywordstyle=\color{magenta},
	showspaces=false,
	showstringspaces=false,
	showtabs=false,
	stepnumber=1,
	stringstyle=\color{mauve},
	tabsize=3
}

% Lenguaje C++
\lstdefinestyle{cpp}{
	language=C++,
	breakatwhitespace=false,
	breaklines=true,
	columns=flexible,
	commentstyle=\color{mygreen},
	keywordstyle=\color{blue}\ttfamily,
	morecomment=[l][\color{magenta}]{\#},
	showspaces=false,
	showstringspaces=false,
	showtabs=false,
	stepnumber=1,
	stringstyle=\color{red}\ttfamily,
	tabsize=3
}

% Lenguaje C#
\lstdefinestyle{csharp}{
	language=csh,
	breaklines=true,
	commentstyle=\color{mygreen},
	keywordstyle=\color{cyan},
	morecomment=[l]{//},
	morecomment=[s]{/*}{*/},
	morekeywords={abstract,event,new,struct,as,explicit,null,switch,base,extern,object,this,bool,false,operator,throw,break,finally,out,true,byte,fixed,override,try,case,float,params,typeof,catch,for,private,uint,char,foreach,protected,ulong,checked,goto,public,unchecked,class,if,readonly,unsafe,const,implicit,ref,ushort,continue,in,return,using,decimal,int,sbyte,virtual,default,interface,sealed,volatile,delegate,internal,short,void,do,is,sizeof,while,double,lock,stackalloc,else,long,static,enum,namespace,string},
	showspaces=false,
	showtabs=false,
	showstringspaces=false,
	stringstyle=\color{blue}\ttfamily,
	tabsize=3
}

% CSS, añade más etiquetas
\lstdefinelanguage{CSS}{
	morecomment=[s]{/*}{*/},
	morekeywords={accelerator,azimuth,background,background-attachment,background-color,background-image,background-position,background-position-x,background-position-y,background-repeat,behavior,border,border-bottom,border-bottom-color,border-bottom-style,border-bottom-width,border-collapse,border-color,border-left,border-left-color,border-left-style,border-left-width,border-right,border-right-color,border-right-style,border-right-width,border-spacing,border-style,border-top,border-top-color,border-top-style,border-top-width,border-width,bottom,caption-side,clear,clip,color,content,counter-increment,counter-reset,cue,cue-after,cue-before,cursor,direction,display,elevation,empty-cells,filter,float,font,font-family,font-size,font-size-adjust,font-stretch,font-style,font-variant,font-weight,height,ime-mode,include-source,layer-background-color,layer-background-image,layout-flow,layout-grid,layout-grid-char,layout-grid-char-spacing,layout-grid-line,layout-grid-mode,layout-grid-type,left,letter-spacing,line-break,line-height,list-style,list-style-image,list-style-position,list-style-type,margin,margin-bottom,margin-left,margin-right,margin-top,marker-offset,marks,max-height,max-width,min-height,min-width,-moz-binding,-moz-border-radius,-moz-border-radius-topleft,-moz-border-radius-topright,-moz-border-radius-bottomright,-moz-border-radius-bottomleft,-moz-border-top-colors,-moz-border-right-colors,-moz-border-bottom-colors,-moz-border-left-colors,-moz-opacity,-moz-outline,-moz-outline-color,-moz-outline-style,-moz-outline-width,-moz-user-focus,-moz-user-input,-moz-user-modify,-moz-user-select,orphans,outline,outline-color,outline-style,outline-width,overflow,overflow-X,overflow-Y,padding,padding-bottom,padding-left,padding-right,padding-top,page,page-break-after,page-break-before,page-break-inside,pause,pause-after,pause-before,pitch,pitch-range,play-during,position,quotes,-replace,richness,right,ruby-align,ruby-overhang,ruby-position,-set-link-source,size,speak,speak-header,speak-numeral,speak-punctuation,speech-rate,stress,scrollbar-arrow-color,scrollbar-base-color,scrollbar-dark-shadow-color,scrollbar-face-color,scrollbar-highlight-color,scrollbar-shadow-color,scrollbar-3d-light-color,scrollbar-track-color,table-layout,text-align,text-align-last,text-decoration,text-indent,text-justify,text-overflow,text-shadow,text-transform,text-autospace,text-kashida-space,text-underline-position,top,unicode-bidi,-use-link-source,vertical-align,visibility,voice-family,volume,white-space,widows,width,word-break,word-spacing,word-wrap,writing-mode,z-index,zoom},
	morestring=[s]{:}{;},
	sensitive=true
}

% Docker
\lstdefinelanguage{docker}{
	comment=[l]{\#},
	keywords={FROM,RUN,COPY,ADD,ENTRYPOINT,CMD,ENV,WORKDIR,EXPOSE,LABEL,USER,VOLUME,STOPSIGNAL,ONBUILD,MAINTAINER},
	morestring=[b]',
	morestring=[b]"
}
\lstdefinestyle{docker}{
	language=docker,
	breakatwhitespace=true,
	breaklines=true,
	columns=flexible,
	commentstyle=\color{dkgreen}\ttfamily,
	identifierstyle=\color{black},
	keywordstyle=\color{blue}\bfseries,
	stepnumber=1,
	stringstyle=\color{red}\ttfamily,
	tabsize=3
}

% HTML5
\lstdefinelanguage{HTML5}{
	language=html,
	alsoletter={<>=-},
	morecomment=[s]{<!--}{-->},
	ndkeywords={=,charset=,id=,width=,height=,border:,transform:,-moz-transform:,transition-duration:,transition-property:,transition-timing-function:},
	otherkeywords={<html>,<head>,<title>,</title>,<meta,/>,</head>,<body>,<canvas,</canvas>,<script>,</script>,</body>,</html>,<!,html>,<style>,</style>,><},
	sensitive=true,
	tag=[s]
}
\lstdefinestyle{html5}{
	language=HTML5,
	alsodigit={.:;},
	alsolanguage=JavaScript,
	breaklines=true,
	commentstyle=\color{darkgray}\ttfamily,
	firstnumber=1,
	keywordstyle=\color{blue}\bfseries,
	ndkeywordstyle=\color{editorGreen}\bfseries,
	numberfirstline=true,
	showspaces=false,
	showstringspaces=false,
	showtabs=false,
	stepnumber=1,
	stringstyle=\color{editorOcher},
	tabsize=3
}

% Lenguaje Java
\lstdefinestyle{java}{
	language=Java,
	breakatwhitespace=true,
	breaklines=true,
	columns=flexible,
	commentstyle=\color{dkgreen},
	keepspaces=true,
	keywordstyle=\color{blue},
	showstringspaces=false,
	stepnumber=1,
	stringstyle=\color{mauve},
	tabsize=3
}

% Lenguaje Javascript
\lstdefinelanguage{JavaScript}{
	breaklines=true,
	columns=flexible,
	comment=[l]{//},
	commentstyle=\color{mygreen},
	identifierstyle=\color{black},
	keepspaces=true,
	keywords={typeof,new,true,false,catch,function,return,null,catch,switch,var,if,in,while,do,else,case,break},
	keywordstyle=\color{blue},
	morecomment=[s]{/*}{*/},
	morestring=[b]',
	morestring=[b]",
	ndkeywords={class,export,boolean,throw,implements,import,this},
	ndkeywordstyle=\color{darkgray}\bfseries,
	sensitive=false,
	showstringspaces=false,
	stepnumber=1,
	stringstyle=\color{red}\ttfamily,
	tabsize=3
}
\lstdefinestyle{js}{
	language=JavaScript
}

% Estilo JSON
\lstdefinestyle{json}{
	literate=*{0}{{{\color{numb}0}}}{1}{1}{{{\color{numb}1}}}{1}{2}{{{\color{numb}2}}}{1}{3}{{{\color{numb}3}}}{1}{4}{{{\color{numb}4}}}{1}{5}{{{\color{numb}5}}}{1}{6}{{{\color{numb}6}}}{1}{7}{{{\color{numb}7}}}{1}{8}{{{\color{numb}8}}}{1}{9}{{{\color{numb}9}}}{1}{:}{{{\color{punct}{:}}}}{1}{,}{{{\color{punct}{,}}}}{1}{\{}{{{\color{delim}{\{}}}}{1}{\}}{{{\color{delim}{\}}}}}{1}{[}{{{\color{delim}{[}}}}{1}{]}{{{\color{delim}{]}}}}{1},
	showstringspaces=false,
	stepnumber=1,
	tabsize=2
}

% Lenguaje Matlab
\lstdefinestyle{matlab}{
	language=Matlab,
	backgroundcolor=\color{backcolour},
	breaklines=true,
	columns=flexible,
	commentstyle=\color{mygreen},
	emph=[1]{for,end,break},emphstyle=[1]\color{red},
	identifierstyle=\color{black},
	keepspaces=true,
	keywordstyle=\color{blue},
	morekeywords={matlab2tikz},
	showstringspaces=false,
	stepnumber=1,
	stringstyle=\color{mauve},
	tabsize=3
}

% Estilo LaTeX
\lstdefinestyle{latex}{
	language=TeX,
	commentstyle=\color{gray},
	keywordstyle=\color{blue}\bfseries,
	morekeywords={align,begin,label,section,subsection,lipsum,insertimage,insertequation,addimage,addimageboxed,newpage,newp,inserteqimage,insertdoubleqimage,insertequationcaptioned,insertimageleft,insertimageright,insertgathered,insertgather,insertalign,insertaligned},
	showstringspaces=false
}

% Lenguaje Perl
\lstdefinestyle{perl}{
	language=Perl,
	alsoletter={\%},
	breakatwhitespace=false,
	breaklines=true,
	columns=flexible,
	commentstyle=\color{purple!40!black},
	identifierstyle=\color{blue},
	keepspaces=true,
	keywordstyle=\bfseries\color{green!40!black},
	showspaces=false,
	showstringspaces=false,
	showtabs=false,
	stepnumber=1,
	stringstyle=\color{codepurple},
	tabsize=3
}

% Lenguaje PHP
\lstdefinestyle{php}{
	language=php,
	commentstyle=\color{gray},
	emph=[1]{php},
	emph=[2]{if,and,or,else},
	emph=[3]{var,const,abstract,protected,private,public,static,final,extends,implements,global,if,else,foreach,for,endforeach,endif,endfor,elseif,as},
	emphstyle=[1]\color{black},
	emphstyle=[2]\color{dkyellow},
	identifierstyle=\color{dkgreen},
	keywords={__halt_compiler,abstract,and,array,as,break,callable,case,catch,class,clone,const,continue,declare,default,die,do,echo,else,elseif,empty,enddeclare,endfor,endforeach,endif,endswitch,endwhile,eval,exit,extends,final,finally,for,foreach,function,global,goto,if,implements,include,include_once,instanceof,insteadof,interface,isset,list,namespace,new,or,print,private,protected,public,require,require_once,return,static,switch,throw,trait,try,unset,use,var,while,xor,yield},
	keywordstyle=\color{dkblue},
	showlines=true,
	showspaces=false,
	showstringspaces=false,
	showtabs=false,
	stepnumber=1,
	stringstyle=\color{red},
	tabsize=3,
	upquote=true
}

% Lenguaje Python
\lstdefinestyle{python}{
	language=Python,
	breakatwhitespace=false,
	breaklines=true,
	columns=flexible,
	commentstyle=\color{codegreen},
	keepspaces=true,
	keywordstyle=\color{magenta},
	showspaces=false,
	showstringspaces=false,
	showtabs=false,
	stepnumber=1,
	stringstyle=\color{codepurple},
	tabsize=3
}

% Lenguaje ruby
\lstdefinestyle{ruby} {
	language=Ruby,
	breakatwhitespace=true,
	breaklines=true,
	columns=flexible,
	commentstyle=\color{dkgreen},
	keywordstyle=\color{blue},
	morestring=[s][]{\#\{}{\}},
	morestring=*[d]{"},
	sensitive=true,
	showstringspaces=false,
	stringstyle=\color{mauve},
	tabsize=3
}

% Instrucciones SQL
\lstdefinestyle{sql}{
	language=SQL,
	breakatwhitespace=true,
	breaklines=true,
	columns=flexible,
	commentstyle=\color{gray},
	keywordstyle=\color{blue},
	showspaces=false,
	tabsize=3
}

% Syntax XML
\lstdefinelanguage{XML}{
	identifierstyle=\color{dkblue},
	keywordstyle=\color{cyan},
	morecomment=[s]{<?}{?>},
	morekeywords={xmlns,version,type,encoding},
	morestring=[b]",
	morestring=[s]{>}{<},
	stringstyle=\color{black}
}
\lstdefinestyle{xml}{
	language=XML,
	commentstyle=\color{gray}\upshape,
	columns=fullflexible,
	showstringspaces=false,
	tabsize=2
}

% Configuración de códigos fuente
\lstset{
	aboveskip=3mm,
	backgroundcolor=\color{backcolour},
	basicstyle={\small\ttfamily},
	belowskip=3mm,
	extendedchars=true,
	keepspaces=true,
	literate={á}{{\'a}}1 {é}{{\'e}}1 {í}{{\'i}}1 {ó}{{\'o}}1 {ú}{{\'u}}1 {Á}{{\'A}}1 {É}{{\'E}}1 {Í}{{\'I}}1 {Ó}{{\'O}}1 {Ú}{{\'U}}1 {à}{{\`a}}1 {è}{{\`e}}1 {ì}{{\`i}}1 {ò}{{\`o}}1 {ù}{{\`u}}1 {À}{{\`A}}1 {È}{{\'E}}1 {Ì}{{\`I}}1 {Ò}{{\`O}}1 {Ù}{{\`U}}1 {ä}{{\"a}}1 {ë}{{\"e}}1 {ï}{{\"i}}1 {ö}{{\"o}}1 {ü}{{\"u}}1 {Ä}{{\"A}}1 {Ë}{{\"E}}1 {Ï}{{\"I}}1 {Ö}{{\"O}}1 {Ü}{{\"U}}1 {â}{{\^a}}1 {ê}{{\^e}}1 {î}{{\^i}}1 {ô}{{\^o}}1 {û}{{\^u}}1 {Â}{{\^A}}1 {Ê}{{\^E}}1 {Î}{{\^I}}1 {Ô}{{\^O}}1 {Û}{{\^U}}1 {œ}{{\oe}}1 {Œ}{{\OE}}1 {æ}{{\ae}}1 {Æ}{{\AE}}1 {ß}{{\ss}}1 {ű}{{\H{u}}}1 {Ű}{{\H{U}}}1 {ő}{{\H{o}}}1 {Ő}{{\H{O}}}1 {ç}{{\c c}}1 {Ç}{{\c C}}1 {ø}{{\o}}1 {å}{{\r a}}1 {Å}{{\r A}}1 {€}{{\EUR}}1 {£}{{\pounds}}1 {ñ}{{\~n}}1 {Ñ}{{\~N}}1 {¿}{{?``}}1 {¡}{{!``}}1,
	numbers=left,
	numbersep=5pt,
	numberstyle=\tiny\color{codegray}
}

% Template:     Informe/Reporte LaTeX
% Documento:    Funciones para insertar elementos
% Versión:      4.7.6 (06/04/2018)
% Codificación: UTF-8
%
% Autor: Pablo Pizarro R.
%        Facultad de Ciencias Físicas y Matemáticas
%        Universidad de Chile
%        pablo.pizarro@ing.uchile.cl, ppizarror.com
%
% Manual template: [http://latex.ppizarror.com/Template-Informe/]
% Licencia MIT:    [https://opensource.org/licenses/MIT/]

\newcommand{\newp}{
	% Insertar párrafo
	\hbadness=10000 \vspace{\defaultnewlinesize pt} \par
}

\newcommand{\newpar}[1]{
	% Insertar párrafo
	% 	#1	Párrafo
	\hbadness=10000 #1 \newp
}

\newcommand{\newparnl}[1]{
	% Insertar párrafo sin nueva línea al final
	% 	#1	Párrafo
	#1 \par
}

\newcommand{\itemresize}[2]{
	% Redimensiona un ítem en textwidth
	% 	#1	Tamaño del nuevo objeto (En textwidth)
	%	#2	Objeto a redimensionar
	\emptyvarerr{\itemresize}{#1}{Tamano del nuevo objeto no definido}
	\emptyvarerr{\itemresize}{#2}{Objeto a redimensionar no definido}
	\resizebox{#1\textwidth}{!}{#2}
}

\newcommand{\insertemptypage}{
	% Crea una página vacía
	\newpage
	\setcounter{templatepagecounter}{\thepage}
	\pagenumbering{gobble}
	\null
	\thispagestyle{empty}
	\newpage
	\pagenumbering{arabic}
	\setcounter{page}{\thetemplatepagecounter}
}

% Inserta un texto entre comillas
\newcommand{\quotes}[1]{\enquote*{#1}}

% Inserta un email con un link cliqueable
\newcommand{\insertemail}[1]{\href{mailto:#1}{\texttt{#1}}}


% END

% IMPORTACIÓN DE ESTILOS
% Template:     Informe/Reporte LaTeX
% Documento:    Estilos del template
% Versión:      6.0.7 (02/11/2018)
% Codificación: UTF-8
%
% Autor: Pablo Pizarro R. @ppizarror
%        Facultad de Ciencias Físicas y Matemáticas
%        Universidad de Chile
%        pablo.pizarro@ing.uchile.cl, ppizarror.com
%
% Manual template: [https://latex.ppizarror.com/Template-Informe/]
% Licencia MIT:    [https://opensource.org/licenses/MIT/]

% Template:     Informe/Reporte LaTeX
% Documento:    Definición de colores
% Versión:      6.1.3 (07/12/2018)
% Codificación: UTF-8
%
% Autor: Pablo Pizarro R. @ppizarror
%        Facultad de Ciencias Físicas y Matemáticas
%        Universidad de Chile
%        pablo.pizarro@ing.uchile.cl, ppizarror.com
%
% Manual template: [https://latex.ppizarror.com/Template-Informe/]
% Licencia MIT:    [https://opensource.org/licenses/MIT/]

\colorlet{numb}{magenta!60!black}
\colorlet{punct}{red!60!black}
\definecolor{delim}{RGB}{20,105,176}
\definecolor{dkcyan}{RGB}{0,123,167}
\definecolor{dkgray}{rgb}{0.35,0.35,0.35}
\definecolor{dkgreen}{rgb}{0,0.6,0}
\definecolor{dkyellow}{cmyk}{0,0,0.8,0.3}
\definecolor{gray}{rgb}{0.5,0.5,0.5}
\definecolor{lbrown}{RGB}{255,252,249}
\definecolor{lgray}{RGB}{240,240,240}
\definecolor{lyellow}{rgb}{1.0,1.0,0.88}
\definecolor{mauve}{rgb}{0.58,0,0.82}
\definecolor{mygray}{rgb}{0.5,0.5,0.5}
\definecolor{ocher}{rgb}{1,0.5,0}
\definecolor{ocre}{RGB}{243,102,25}

% END
% Template:     Informe/Reporte LaTeX
% Documento:    Estilos de código fuente
% Versión:      5.0.4 (25/04/2018)
% Codificación: UTF-8
%
% Autor: Pablo Pizarro R. @ppizarror
%        Facultad de Ciencias Físicas y Matemáticas
%        Universidad de Chile
%        pablo.pizarro@ing.uchile.cl, ppizarror.com
%
% Manual template: [http://latex.ppizarror.com/Template-Informe/]
% Licencia MIT:    [https://opensource.org/licenses/MIT/]

% Lenguaje C
\lstdefinestyle{c}{
	language=C,
	breakatwhitespace=false,
	breaklines=true,
	columns=flexible,
	commentstyle=\color{mygreen},
	keepspaces=true,
	keywordstyle=\color{magenta},
	showspaces=false,
	showstringspaces=false,
	showtabs=false,
	stepnumber=1,
	stringstyle=\color{mauve},
	tabsize=3
}

% Lenguaje C++
\lstdefinestyle{cpp}{
	language=C++,
	breakatwhitespace=false,
	breaklines=true,
	columns=flexible,
	commentstyle=\color{mygreen},
	keywordstyle=\color{blue}\ttfamily,
	morecomment=[l][\color{magenta}]{\#},
	showspaces=false,
	showstringspaces=false,
	showtabs=false,
	stepnumber=1,
	stringstyle=\color{red}\ttfamily,
	tabsize=3
}

% Lenguaje C#
\lstdefinestyle{csharp}{
	language=csh,
	breaklines=true,
	commentstyle=\color{mygreen},
	keywordstyle=\color{cyan},
	morecomment=[l]{//},
	morecomment=[s]{/*}{*/},
	morekeywords={abstract,event,new,struct,as,explicit,null,switch,base,extern,object,this,bool,false,operator,throw,break,finally,out,true,byte,fixed,override,try,case,float,params,typeof,catch,for,private,uint,char,foreach,protected,ulong,checked,goto,public,unchecked,class,if,readonly,unsafe,const,implicit,ref,ushort,continue,in,return,using,decimal,int,sbyte,virtual,default,interface,sealed,volatile,delegate,internal,short,void,do,is,sizeof,while,double,lock,stackalloc,else,long,static,enum,namespace,string},
	showspaces=false,
	showtabs=false,
	showstringspaces=false,
	stringstyle=\color{blue}\ttfamily,
	tabsize=3
}

% CSS, añade más etiquetas
\lstdefinelanguage{CSS}{
	morecomment=[s]{/*}{*/},
	morekeywords={accelerator,azimuth,background,background-attachment,background-color,background-image,background-position,background-position-x,background-position-y,background-repeat,behavior,border,border-bottom,border-bottom-color,border-bottom-style,border-bottom-width,border-collapse,border-color,border-left,border-left-color,border-left-style,border-left-width,border-right,border-right-color,border-right-style,border-right-width,border-spacing,border-style,border-top,border-top-color,border-top-style,border-top-width,border-width,bottom,caption-side,clear,clip,color,content,counter-increment,counter-reset,cue,cue-after,cue-before,cursor,direction,display,elevation,empty-cells,filter,float,font,font-family,font-size,font-size-adjust,font-stretch,font-style,font-variant,font-weight,height,ime-mode,include-source,layer-background-color,layer-background-image,layout-flow,layout-grid,layout-grid-char,layout-grid-char-spacing,layout-grid-line,layout-grid-mode,layout-grid-type,left,letter-spacing,line-break,line-height,list-style,list-style-image,list-style-position,list-style-type,margin,margin-bottom,margin-left,margin-right,margin-top,marker-offset,marks,max-height,max-width,min-height,min-width,-moz-binding,-moz-border-radius,-moz-border-radius-topleft,-moz-border-radius-topright,-moz-border-radius-bottomright,-moz-border-radius-bottomleft,-moz-border-top-colors,-moz-border-right-colors,-moz-border-bottom-colors,-moz-border-left-colors,-moz-opacity,-moz-outline,-moz-outline-color,-moz-outline-style,-moz-outline-width,-moz-user-focus,-moz-user-input,-moz-user-modify,-moz-user-select,orphans,outline,outline-color,outline-style,outline-width,overflow,overflow-X,overflow-Y,padding,padding-bottom,padding-left,padding-right,padding-top,page,page-break-after,page-break-before,page-break-inside,pause,pause-after,pause-before,pitch,pitch-range,play-during,position,quotes,-replace,richness,right,ruby-align,ruby-overhang,ruby-position,-set-link-source,size,speak,speak-header,speak-numeral,speak-punctuation,speech-rate,stress,scrollbar-arrow-color,scrollbar-base-color,scrollbar-dark-shadow-color,scrollbar-face-color,scrollbar-highlight-color,scrollbar-shadow-color,scrollbar-3d-light-color,scrollbar-track-color,table-layout,text-align,text-align-last,text-decoration,text-indent,text-justify,text-overflow,text-shadow,text-transform,text-autospace,text-kashida-space,text-underline-position,top,unicode-bidi,-use-link-source,vertical-align,visibility,voice-family,volume,white-space,widows,width,word-break,word-spacing,word-wrap,writing-mode,z-index,zoom},
	morestring=[s]{:}{;},
	sensitive=true
}

% Docker
\lstdefinelanguage{docker}{
	comment=[l]{\#},
	keywords={FROM,RUN,COPY,ADD,ENTRYPOINT,CMD,ENV,WORKDIR,EXPOSE,LABEL,USER,VOLUME,STOPSIGNAL,ONBUILD,MAINTAINER},
	morestring=[b]',
	morestring=[b]"
}
\lstdefinestyle{docker}{
	language=docker,
	breakatwhitespace=true,
	breaklines=true,
	columns=flexible,
	commentstyle=\color{dkgreen}\ttfamily,
	identifierstyle=\color{black},
	keywordstyle=\color{blue}\bfseries,
	stepnumber=1,
	stringstyle=\color{red}\ttfamily,
	tabsize=3
}

% HTML5
\lstdefinelanguage{HTML5}{
	language=html,
	alsoletter={<>=-},
	morecomment=[s]{<!--}{-->},
	ndkeywords={=,charset=,id=,width=,height=,border:,transform:,-moz-transform:,transition-duration:,transition-property:,transition-timing-function:},
	otherkeywords={<html>,<head>,<title>,</title>,<meta,/>,</head>,<body>,<canvas,</canvas>,<script>,</script>,</body>,</html>,<!,html>,<style>,</style>,><},
	sensitive=true,
	tag=[s]
}
\lstdefinestyle{html5}{
	language=HTML5,
	alsodigit={.:;},
	alsolanguage=JavaScript,
	breaklines=true,
	commentstyle=\color{darkgray}\ttfamily,
	firstnumber=1,
	keywordstyle=\color{blue}\bfseries,
	ndkeywordstyle=\color{editorGreen}\bfseries,
	numberfirstline=true,
	showspaces=false,
	showstringspaces=false,
	showtabs=false,
	stepnumber=1,
	stringstyle=\color{editorOcher},
	tabsize=3
}

% Lenguaje Java
\lstdefinestyle{java}{
	language=Java,
	breakatwhitespace=true,
	breaklines=true,
	columns=flexible,
	commentstyle=\color{dkgreen},
	keepspaces=true,
	keywordstyle=\color{blue},
	showstringspaces=false,
	stepnumber=1,
	stringstyle=\color{mauve},
	tabsize=3
}

% Lenguaje Javascript
\lstdefinelanguage{JavaScript}{
	breaklines=true,
	columns=flexible,
	comment=[l]{//},
	commentstyle=\color{mygreen},
	identifierstyle=\color{black},
	keepspaces=true,
	keywords={typeof,new,true,false,catch,function,return,null,catch,switch,var,if,in,while,do,else,case,break},
	keywordstyle=\color{blue},
	morecomment=[s]{/*}{*/},
	morestring=[b]',
	morestring=[b]",
	ndkeywords={class,export,boolean,throw,implements,import,this},
	ndkeywordstyle=\color{darkgray}\bfseries,
	sensitive=false,
	showstringspaces=false,
	stepnumber=1,
	stringstyle=\color{red}\ttfamily,
	tabsize=3
}
\lstdefinestyle{js}{
	language=JavaScript
}

% Estilo JSON
\lstdefinestyle{json}{
	literate=*{0}{{{\color{numb}0}}}{1}{1}{{{\color{numb}1}}}{1}{2}{{{\color{numb}2}}}{1}{3}{{{\color{numb}3}}}{1}{4}{{{\color{numb}4}}}{1}{5}{{{\color{numb}5}}}{1}{6}{{{\color{numb}6}}}{1}{7}{{{\color{numb}7}}}{1}{8}{{{\color{numb}8}}}{1}{9}{{{\color{numb}9}}}{1}{:}{{{\color{punct}{:}}}}{1}{,}{{{\color{punct}{,}}}}{1}{\{}{{{\color{delim}{\{}}}}{1}{\}}{{{\color{delim}{\}}}}}{1}{[}{{{\color{delim}{[}}}}{1}{]}{{{\color{delim}{]}}}}{1},
	showstringspaces=false,
	stepnumber=1,
	tabsize=2
}

% Lenguaje Matlab
\lstdefinestyle{matlab}{
	language=Matlab,
	backgroundcolor=\color{backcolour},
	breaklines=true,
	columns=flexible,
	commentstyle=\color{mygreen},
	emph=[1]{for,end,break},emphstyle=[1]\color{red},
	identifierstyle=\color{black},
	keepspaces=true,
	keywordstyle=\color{blue},
	morekeywords={matlab2tikz},
	showstringspaces=false,
	stepnumber=1,
	stringstyle=\color{mauve},
	tabsize=3
}

% Estilo LaTeX
\lstdefinestyle{latex}{
	language=TeX,
	commentstyle=\color{gray},
	keywordstyle=\color{blue}\bfseries,
	morekeywords={align,begin,label,section,subsection,lipsum,insertimage,insertequation,addimage,addimageboxed,newpage,newp,inserteqimage,insertdoubleqimage,insertequationcaptioned,insertimageleft,insertimageright,insertgathered,insertgather,insertalign,insertaligned},
	showstringspaces=false
}

% Lenguaje Perl
\lstdefinestyle{perl}{
	language=Perl,
	alsoletter={\%},
	breakatwhitespace=false,
	breaklines=true,
	columns=flexible,
	commentstyle=\color{purple!40!black},
	identifierstyle=\color{blue},
	keepspaces=true,
	keywordstyle=\bfseries\color{green!40!black},
	showspaces=false,
	showstringspaces=false,
	showtabs=false,
	stepnumber=1,
	stringstyle=\color{codepurple},
	tabsize=3
}

% Lenguaje PHP
\lstdefinestyle{php}{
	language=php,
	commentstyle=\color{gray},
	emph=[1]{php},
	emph=[2]{if,and,or,else},
	emph=[3]{var,const,abstract,protected,private,public,static,final,extends,implements,global,if,else,foreach,for,endforeach,endif,endfor,elseif,as},
	emphstyle=[1]\color{black},
	emphstyle=[2]\color{dkyellow},
	identifierstyle=\color{dkgreen},
	keywords={__halt_compiler,abstract,and,array,as,break,callable,case,catch,class,clone,const,continue,declare,default,die,do,echo,else,elseif,empty,enddeclare,endfor,endforeach,endif,endswitch,endwhile,eval,exit,extends,final,finally,for,foreach,function,global,goto,if,implements,include,include_once,instanceof,insteadof,interface,isset,list,namespace,new,or,print,private,protected,public,require,require_once,return,static,switch,throw,trait,try,unset,use,var,while,xor,yield},
	keywordstyle=\color{dkblue},
	showlines=true,
	showspaces=false,
	showstringspaces=false,
	showtabs=false,
	stepnumber=1,
	stringstyle=\color{red},
	tabsize=3,
	upquote=true
}

% Lenguaje Python
\lstdefinestyle{python}{
	language=Python,
	breakatwhitespace=false,
	breaklines=true,
	columns=flexible,
	commentstyle=\color{codegreen},
	keepspaces=true,
	keywordstyle=\color{magenta},
	showspaces=false,
	showstringspaces=false,
	showtabs=false,
	stepnumber=1,
	stringstyle=\color{codepurple},
	tabsize=3
}

% Lenguaje ruby
\lstdefinestyle{ruby} {
	language=Ruby,
	breakatwhitespace=true,
	breaklines=true,
	columns=flexible,
	commentstyle=\color{dkgreen},
	keywordstyle=\color{blue},
	morestring=[s][]{\#\{}{\}},
	morestring=*[d]{"},
	sensitive=true,
	showstringspaces=false,
	stringstyle=\color{mauve},
	tabsize=3
}

% Instrucciones SQL
\lstdefinestyle{sql}{
	language=SQL,
	breakatwhitespace=true,
	breaklines=true,
	columns=flexible,
	commentstyle=\color{gray},
	keywordstyle=\color{blue},
	showspaces=false,
	tabsize=3
}

% Syntax XML
\lstdefinelanguage{XML}{
	identifierstyle=\color{dkblue},
	keywordstyle=\color{cyan},
	morecomment=[s]{<?}{?>},
	morekeywords={xmlns,version,type,encoding},
	morestring=[b]",
	morestring=[s]{>}{<},
	stringstyle=\color{black}
}
\lstdefinestyle{xml}{
	language=XML,
	commentstyle=\color{gray}\upshape,
	columns=fullflexible,
	showstringspaces=false,
	tabsize=2
}

% Configuración de códigos fuente
\lstset{
	aboveskip=3mm,
	backgroundcolor=\color{backcolour},
	basicstyle={\small\ttfamily},
	belowskip=3mm,
	extendedchars=true,
	keepspaces=true,
	literate={á}{{\'a}}1 {é}{{\'e}}1 {í}{{\'i}}1 {ó}{{\'o}}1 {ú}{{\'u}}1 {Á}{{\'A}}1 {É}{{\'E}}1 {Í}{{\'I}}1 {Ó}{{\'O}}1 {Ú}{{\'U}}1 {à}{{\`a}}1 {è}{{\`e}}1 {ì}{{\`i}}1 {ò}{{\`o}}1 {ù}{{\`u}}1 {À}{{\`A}}1 {È}{{\'E}}1 {Ì}{{\`I}}1 {Ò}{{\`O}}1 {Ù}{{\`U}}1 {ä}{{\"a}}1 {ë}{{\"e}}1 {ï}{{\"i}}1 {ö}{{\"o}}1 {ü}{{\"u}}1 {Ä}{{\"A}}1 {Ë}{{\"E}}1 {Ï}{{\"I}}1 {Ö}{{\"O}}1 {Ü}{{\"U}}1 {â}{{\^a}}1 {ê}{{\^e}}1 {î}{{\^i}}1 {ô}{{\^o}}1 {û}{{\^u}}1 {Â}{{\^A}}1 {Ê}{{\^E}}1 {Î}{{\^I}}1 {Ô}{{\^O}}1 {Û}{{\^U}}1 {œ}{{\oe}}1 {Œ}{{\OE}}1 {æ}{{\ae}}1 {Æ}{{\AE}}1 {ß}{{\ss}}1 {ű}{{\H{u}}}1 {Ű}{{\H{U}}}1 {ő}{{\H{o}}}1 {Ő}{{\H{O}}}1 {ç}{{\c c}}1 {Ç}{{\c C}}1 {ø}{{\o}}1 {å}{{\r a}}1 {Å}{{\r A}}1 {€}{{\EUR}}1 {£}{{\pounds}}1 {ñ}{{\~n}}1 {Ñ}{{\~N}}1 {¿}{{?``}}1 {¡}{{!``}}1,
	numbers=left,
	numbersep=5pt,
	numberstyle=\tiny\color{codegray}
}

% Template:     Informe/Reporte LaTeX
% Documento:    Funciones para insertar elementos
% Versión:      4.7.6 (06/04/2018)
% Codificación: UTF-8
%
% Autor: Pablo Pizarro R.
%        Facultad de Ciencias Físicas y Matemáticas
%        Universidad de Chile
%        pablo.pizarro@ing.uchile.cl, ppizarror.com
%
% Manual template: [http://latex.ppizarror.com/Template-Informe/]
% Licencia MIT:    [https://opensource.org/licenses/MIT/]

\newcommand{\newp}{
	% Insertar párrafo
	\hbadness=10000 \vspace{\defaultnewlinesize pt} \par
}

\newcommand{\newpar}[1]{
	% Insertar párrafo
	% 	#1	Párrafo
	\hbadness=10000 #1 \newp
}

\newcommand{\newparnl}[1]{
	% Insertar párrafo sin nueva línea al final
	% 	#1	Párrafo
	#1 \par
}

\newcommand{\itemresize}[2]{
	% Redimensiona un ítem en textwidth
	% 	#1	Tamaño del nuevo objeto (En textwidth)
	%	#2	Objeto a redimensionar
	\emptyvarerr{\itemresize}{#1}{Tamano del nuevo objeto no definido}
	\emptyvarerr{\itemresize}{#2}{Objeto a redimensionar no definido}
	\resizebox{#1\textwidth}{!}{#2}
}

\newcommand{\insertemptypage}{
	% Crea una página vacía
	\newpage
	\setcounter{templatepagecounter}{\thepage}
	\pagenumbering{gobble}
	\null
	\thispagestyle{empty}
	\newpage
	\pagenumbering{arabic}
	\setcounter{page}{\thetemplatepagecounter}
}

% Inserta un texto entre comillas
\newcommand{\quotes}[1]{\enquote*{#1}}

% Inserta un email con un link cliqueable
\newcommand{\insertemail}[1]{\href{mailto:#1}{\texttt{#1}}}


% END

% CONFIGURACIÓN INICIAL DEL DOCUMENTO
% Template:     Informe/Reporte LaTeX
% Documento:    Configuración inicial del template
% Versión:      DEV
% Codificación: UTF-8
%
% Autor: Pablo Pizarro R. @ppizarror
%        Facultad de Ciencias Físicas y Matemáticas
%        Universidad de Chile
%        pablo.pizarro@ing.uchile.cl, ppizarror.com
%
% Manual template: [https://latex.ppizarror.com/Template-Informe/]
% Licencia MIT:    [https://opensource.org/licenses/MIT/]

% -----------------------------------------------------------------------------
% Se revisa si las variables no han sido borradas
% -----------------------------------------------------------------------------
\checkvardefined{\autordeldocumento}
\checkvardefined{\codigodelcurso}
\checkvardefined{\departamentouniversidad}
\checkvardefined{\localizacionuniversidad}
\checkvardefined{\nombredelcurso}
\checkvardefined{\nombrefacultad}
\checkvardefined{\nombreuniversidad}
\checkvardefined{\temaatratar}
\checkvardefined{\titulodelinforme}

% -----------------------------------------------------------------------------
% Se añade \xspace a las variables
% -----------------------------------------------------------------------------
\makeatletter
	\g@addto@macro\autordeldocumento\xspace
	\g@addto@macro\codigodelcurso\xspace
	\g@addto@macro\departamentouniversidad\xspace
	\g@addto@macro\localizacionuniversidad\xspace
	\g@addto@macro\nombredelcurso\xspace
	\g@addto@macro\nombrefacultad\xspace
	\g@addto@macro\nombreuniversidad\xspace
	\g@addto@macro\temaatratar\xspace
	\g@addto@macro\titulodelinforme\xspace
\makeatother

% -----------------------------------------------------------------------------
% Se crean variables si se borraron
% -----------------------------------------------------------------------------
\ifthenelse{\isundefined{\tablaintegrantes}}{
	\errmessage{LaTeX Warning: Se borro la variable \noexpand\tablaintegrantes, creando una vacia}
	\def\tablaintegrantes {}}{
}

% -----------------------------------------------------------------------------
% Se activan números en menú marcadores del pdf
% -----------------------------------------------------------------------------
\ifthenelse{\equal{\cfgpdfsecnumbookmarks}{true}}{
	\bookmarksetup{numbered}}{
}

% -----------------------------------------------------------------------------
% Se define metadata del pdf
% -----------------------------------------------------------------------------
\ifthenelse{\equal{\cfgshowbookmarkmenu}{true}}{
	\def\cdfpagemodepdf {UseOutlines}
	}{
	\def\cdfpagemodepdf {UseNone}
}
\hypersetup{
	bookmarksopen={\cfgpdfbookmarkopen},
	bookmarksopenlevel={\cfgbookmarksopenlevel},
	bookmarkstype={toc},
	pdfauthor={\autordeldocumento},
	pdfcenterwindow={\cfgpdfcenterwindow},
	pdfcopyright={\cfgpdfcopyright},
	pdfcreator={LaTeX},
	pdfdisplaydoctitle={\cfgpdfdisplaydoctitle},
	pdfencoding=unicode,
	pdffitwindow={\cfgpdffitwindow},
	pdfinfo={
		Curso.Codigo={\codigodelcurso},
		Curso.Nombre={\nombredelcurso},
		Documento.Autor={\autordeldocumento},
		Documento.Tema={\temaatratar},
		Documento.Titulo={\titulodelinforme},
		Template.Autor.Alias={ppizarror},
		Template.Autor.Email={pablo.pizarro@ing.uchile.cl},
		Template.Autor.Nombre={Pablo Pizarro R.},
		Template.Autor.Web={https://ppizarror.com/},
		Template.Codificacion={UTF-8},
		Template.Fecha={DEV},
		Template.Latex.Compilador={pdflatex},
		Template.Licencia.Tipo={MIT},
		Template.Licencia.Web={https://opensource.org/licenses/MIT/},
		Template.Nombre={Template-Informe},
		Template.Tipo={Normal},
		Template.Version.Dev={DEV},
		Template.Version.Hash={DEV},
		Template.Version.Release={DEV},
		Template.Web.Dev={https://github.com/Template-Latex/Template-Informe/},
		Template.Web.Manual={https://latex.ppizarror.com/Template-Informe/},
		Universidad.Departamento={\departamentouniversidad},
		Universidad.Nombre={\nombreuniversidad},
		Universidad.Ubicacion={\localizacionuniversidad}
	},
	pdfkeywords={\codigodelcurso, \nombredelcurso, \nombreuniversidad, \localizacionuniversidad},
	pdflang={\documentlang},
	pdfmenubar={\cfgpdfmenubar},
	pdfpagelayout={\cfgpdfpagemode},
	pdfpagemode={\cdfpagemodepdf},
	pdfproducer={Template-Informe DEV | (Pablo Pizarro R.) ppizarror.com},
	pdfremotestartview={Fit},
	pdfstartpage={1},
	pdfstartview={\cfgpdfpageview},
	pdfsubject={\temaatratar},
	pdftitle={\titulodelinforme},
	pdftoolbar={\cfgpdftoolbar}
}

% -----------------------------------------------------------------------------
% Establece la carpeta de imágenes por defecto
% -----------------------------------------------------------------------------
\graphicspath{{./\defaultimagefolder}{./\defaultimagefolder/departamentos/}}

% -----------------------------------------------------------------------------
% Definición de valores e dimensiones
% -----------------------------------------------------------------------------
\renewcommand{\baselinestretch}{\defaultinterline} % Ajuste del entrelineado
\setlength{\headheight}{64 pt} % Tamaño de la cabecera sin fancyhdr
% \setcounter{MaxMatrixCols}{20} % Número máximo de columnas en matrices
\setlength{\footnotemargin}{\marginfootnote pt} % Margen del footnote
\setlength{\columnsep}{\columnsepwidth em} % Separación entre columnas
\ifthenelse{\equal{\showlinenumbers}{true}}{
	\setlength{\linenumbersep}{0.50cm}
	\renewcommand\linenumberfont{\normalfont\tiny\color{\linenumbercolor}}
	}{
}

% -----------------------------------------------------------------------------
% Posición inicial de los objetos
% -----------------------------------------------------------------------------
\floatplacement{figure}{\imagedefaultplacement}
\floatplacement{table}{\tabledefaultplacement}
\floatplacement{tikz}{\tikzdefaultplacement}

% -----------------------------------------------------------------------------
% Configuración de los colores
% -----------------------------------------------------------------------------
\color{\maintextcolor} % Color principal
\arrayrulecolor{\tablelinecolor} % Color de las líneas de las tablas
\sethlcolor{\highlightcolor} % Color del subrayado por defecto
\ifthenelse{\equal{\showborderonlinks}{true}}{
	% Color de links con borde
	\hypersetup{
		citebordercolor=\numcitecolor,
		linkbordercolor=\linkcolor,
		urlbordercolor=\urlcolor
	}
}{
	% Color de links sin borde
	\hypersetup{ % No reorganizar
		hidelinks,
		colorlinks=true,
		citecolor=\numcitecolor,
		linkcolor=\linkcolor,
		urlcolor=\urlcolor
	}
}
\ifthenelse{\equal{\colorpage}{white}}{}{
	\pagecolor{\colorpage}
}

% -----------------------------------------------------------------------------
% Configuración de las leyendas
% -----------------------------------------------------------------------------

% Márgenes de las leyendas por defecto
\setcaptionmargincm{\captionlrmargin}
\ifthenelse{\equal{\captiontextbold}{true}}{ % Texto en negrita en etiquetas
	\renewcommand{\captiontextbold}{bf}}{
	\renewcommand{\captiontextbold}{}
}
\ifthenelse{\equal{\captiontextsubnumbold}{true}}{ % Número en negritas
	\renewcommand{\captiontextsubnumbold}{bf}}{
	\renewcommand{\captiontextsubnumbold}{}
}

% Se configura el texto de los caption
\captionsetup{
	labelfont={color=\captioncolor, \captiontextbold},
	labelformat={\captionlabelformat},
	labelsep={\captionlabelsep},
	textfont={color=\captiontextcolor},
	singlelinecheck=on
}

% Configura texto de los subcaption
\captionsetup*[subfigure]{
	labelfont={color=\captioncolor, \captiontextsubnumbold},
	labelformat={\subcaptionlabelformat},
	labelsep={\subcaptionlabelsep},
	textfont={color=\captiontextcolor},
	singlelinecheck=on
}
\captionsetup*[subtable]{
	labelfont={color=\captioncolor, \captiontextsubnumbold},
	labelformat={\subcaptionlabelformat},
	labelsep={\subcaptionlabelsep},
	textfont={color=\captiontextcolor},
	singlelinecheck=on
}

% Configuración de márgenes en las figuras
\floatsetup[figure]{
	captionskip=\captiontbmarginfigure pt
}

% Configuración de márgenes en las tablas
\floatsetup[table]{
	captionskip=\captiontbmargintable pt
}

% Caption superior en figuras
\ifthenelse{\equal{\figurecaptiontop}{true}}{
	\floatsetup[figure]{position=above}}{
}

% Caption superior en tablas
\ifthenelse{\equal{\tablecaptiontop}{true}}{
	\floatsetup[table]{position=top}
	}{
	\floatsetup[table]{position=bottom}
}

% Alineado de leyendas
\ifthenelse{\equal{\captionalignment}{justified}}{ % Leyenda justificada
	\captionsetup{
		format=plain,
		justification=justified
	}
}{
\ifthenelse{\equal{\captionalignment}{centered}}{ % Leyenda centrada
	\captionsetup{
		justification=centering
	}
}{
\ifthenelse{\equal{\captionalignment}{left}}{ % Leyenda alineada a la izquierda
	\captionsetup{
		justification=raggedright,
		singlelinecheck=false
	}
}{
\ifthenelse{\equal{\captionalignment}{right}}{ % Leyenda alineada a la derecha
	\captionsetup{
		justification=raggedleft,
		singlelinecheck=false
	}
}{
	\throwbadconfig{Posicion de leyendas desconocida}{\captionalignment}{justified,centered,left,right}}}}
}

% -----------------------------------------------------------------------------
% Configuración de referencias y citas
% -----------------------------------------------------------------------------
\ifthenelse{\equal{\stylecitereferences}{natbib}}{
	\bibliographystyle{\natbibrefstyle}
	\setlength{\bibsep}{\natbibrefsep pt}
}{
\ifthenelse{\equal{\stylecitereferences}{apacite}}{
	\bibliographystyle{apacite}
	\setlength{\bibitemsep}{\apaciterefsep pt}
}{
\ifthenelse{\equal{\stylecitereferences}{bibtex}}{
	\bibliographystyle{apa}
	\newlength{\bibitemsep}
	\setlength{\bibitemsep}{.2\baselineskip plus .05\baselineskip minus .05\baselineskip}
	\newlength{\bibparskip}\setlength{\bibparskip}{0pt}
	\let\oldthebibliography\thebibliography
	\renewcommand\thebibliography[1]{
		\oldthebibliography{#1}
		\setlength{\parskip}{\bibitemsep}
		\setlength{\itemsep}{\bibparskip}
	}
	\setlength{\bibitemsep}{\bibtexrefsep pt}
}{
	\throwbadconfig{Estilo citas desconocido}{\stylecitereferences}{bibtex,apacite,natbib}}}
}
% Referencias en 2 columnas
\makeatletter
\ifthenelse{\equal{\twocolumnreferences}{true}}{
	\renewenvironment{thebibliography}[1]
	{\begin{multicols}{2}[\section*{\refname}]
		\@mkboth{\MakeUppercase\refname}{\MakeUppercase\refname}
		\list{\@biblabel{\@arabic\c@enumiv}}
		{\settowidth\labelwidth{\@biblabel{#1}}
			\leftmargin\labelwidth
			\advance\leftmargin\labelsep
			\@openbib@code
			\usecounter{enumiv}
			\let\p@enumiv\@empty
			\renewcommand\theenumiv{\@arabic\c@enumiv}}
		\sloppy
		\clubpenalty 4000
		\@clubpenalty \clubpenalty
		\widowpenalty 4000
		\sfcode`\.\@m}
		{\def\@noitemerr
		{\@latex@warning{Ambiente `thebibliography' no definido}}
		\endlist\end{multicols}}}{}
\makeatother

% -----------------------------------------------------------------------------
% Configuración anexo
% -----------------------------------------------------------------------------
\patchcmd{\appendices}{\quad}{\sectionappendixlastchar\quad}{}{}

% -----------------------------------------------------------------------------
% Se añade listings (código fuente) a tocloft
% -----------------------------------------------------------------------------
\begingroup
	\makeatletter
	\let\newcounter\@gobble\let\setcounter\@gobbletwo
	\globaldefs\@ne\let\c@loldepth\@ne
	\newlistof{listings}{lol}{\lstlistlistingname}
	\newlistentry{lstlisting}{lol}{0}
	\makeatother
\endgroup

% -----------------------------------------------------------------------------
% Reconfiguración de tamaño de páginas
% -----------------------------------------------------------------------------
\makeatletter
	\def\ifGm@preamble#1{\@firstofone}
	\appto\restoregeometry{
		\pdfpagewidth=\paperwidth
		\pdfpageheight=\paperheight}
	\apptocmd\newgeometry{
		\pdfpagewidth=\paperwidth
		\pdfpageheight=\paperheight}{}{}
\makeatother

% -----------------------------------------------------------------------------
% Configuración de hbox y vbox
% -----------------------------------------------------------------------------
\hfuzz=200pt
\vfuzz=200pt
\hbadness=\maxdimen
\vbadness=\maxdimen
% \sloppy Sloppy arruina portadas al exigir "justify", desde 6.4.2 se desactiva

% -----------------------------------------------------------------------------
% Configuraciones de las tablas
% -----------------------------------------------------------------------------

% Reinicia el número de cada fila en todas las tablas
\makeatletter
\preto\tabular{\global\rownum=\z@}
\preto\tabularx{\global\rownum=\z@}
\makeatother

% -----------------------------------------------------------------------------
% Se activa el modo estricto de revisión de números de página
% -----------------------------------------------------------------------------
\strictpagecheck

% -----------------------------------------------------------------------------
% Actualización márgen títulos
% -----------------------------------------------------------------------------
\titlespacing{\section}{0pt}{20pt}{10pt}
\titlespacing{\subsection}{0pt}{15pt}{10pt}

% -----------------------------------------------------------------------------
% Se activa el word-wrap para textos con \texttt{}
% -----------------------------------------------------------------------------
\ttfamily \hyphenchar\the\font=`\-

% -----------------------------------------------------------------------------
% Se define el tipo de texto de los url
% -----------------------------------------------------------------------------
\urlstyle{\fonturl}

% -----------------------------------------------------------------------------
% Se revisa si se importa tikz
% -----------------------------------------------------------------------------
\ifthenelse{\equal{\portraitstyle}{style16}}{\coreimporttikz}{}
\ifthenelse{\equal{\portraitstyle}{\bgtemplatetestcode}}{\coreimporttikz}{}

% -----------------------------------------------------------------------------
% Configuraciones del motor de compilación
% -----------------------------------------------------------------------------

% Nivel de compresión
\pdfcompresslevel=\pdfcompilecompression

% El óptimo es 2, según
% http://texdoc.net/texmf-dist/doc/pdftex/manual/pdftex-a.pdf p.20
\pdfdecimaldigits=2

% Inclusión de PDF
\pdfinclusionerrorlevel=0

% Versión
\pdfminorversion=\pdfcompileversion

% Compresión de objetos
\pdfobjcompresslevel=\pdfcompileobjcompression

% -----------------------------------------------------------------------------
% Profundidad de las secciones
% -----------------------------------------------------------------------------
\setcounter{secnumdepth}{4} % Límite máximo subsubsubsección

% Crea las subsubsubsecciones
\newcounter{subsubsubsection}[subsubsection]

% Establece el número de las subsubsubsecciones
\ifthenelse{\equal{\showdotaftersnum}{true}}{ % Agrega punto tras el número
	\renewcommand{\thesubsubsubsection}{\thesubsubsection.\arabic{subsubsubsection}.}
	\renewcommand{\theparagraph}{\thesubsubsubsection.\arabic{paragraph}.}
}{
	\renewcommand{\thesubsubsubsection}{\thesubsubsection.\arabic{subsubsubsection}}
	\renewcommand{\theparagraph}{\thesubsubsubsection.\arabic{paragraph}}
}

% -----------------------------------------------------------------------------
% Agrega compatibilidad de subsubsubsecciones al TOC
% -----------------------------------------------------------------------------
\makeatletter
	\def\toclevel@subsubsubsection{4}
	\def\toclevel@paragraph{5}
	\def\toclevel@subparagraph{6}
	\renewcommand\paragraph{\@startsection{paragraph}{5}{\z@}
		{3.25ex \@plus 1ex \@minus .2ex}
		{-1em}
		{\normalfont\normalsize\bfseries}}
	\renewcommand\subparagraph{\@startsection{subparagraph}{6}{\parindent}
		{3.25ex \@plus 1ex \@minus .2ex}
		{-1em}
		{\normalfont\normalsize\bfseries}}
	\ifthenelse{\equal{\showdotaftersnum}{true}}{
		\def\l@subsubsubsection{\@dottedtocline{4}{7.83em}{4.15em}} % Incremento 0.77+3.35 a 3.35
		\def\l@paragraph{\@dottedtocline{5}{11.98em}{4.92em}}
		\def\l@subparagraph{\@dottedtocline{6}{14.65em}{5.69em}}
	}{
		\def\l@subsubsubsection{\@dottedtocline{4}{6.97em}{4em}}
		\def\l@paragraph{\@dottedtocline{5}{10.97em}{5em}}
		\def\l@subparagraph{\@dottedtocline{6}{14em}{6em}}
	}
\makeatother

% -----------------------------------------------------------------------------
% Profundidad del índice y bookmarks pdf
% -----------------------------------------------------------------------------
\setcounter{tocdepth}{\indexdepth}

% -----------------------------------------------------------------------------
% Restauración número footnote
% -----------------------------------------------------------------------------
\ifthenelse{\equal{\footnoterestart}{none}}{
}{
\ifthenelse{\equal{\footnoterestart}{sec}}{
	\counterwithin*{footnote}{section}
}{
\ifthenelse{\equal{\footnoterestart}{ssec}}{
	\counterwithin*{footnote}{subsection}
}{
\ifthenelse{\equal{\footnoterestart}{sssec}}{
	\counterwithin*{footnote}{subsubsection}
}{
\ifthenelse{\equal{\footnoterestart}{ssssec}}{
	\counterwithin*{footnote}{subsubsubsection}
}{
\ifthenelse{\equal{\footnoterestart}{page}}{
	\counterwithin*{footnote}{page}
}{
	\throwbadconfig{Formato reinicio numero footnote desconocido}{\footnoterestart}{none,page,sec,ssec,sssec,ssssec}}}}}}
}

% -----------------------------------------------------------------------------
% Restauración número ecuación, NOTA: NO hace nada, sólo se modifica en title.tex
% -----------------------------------------------------------------------------
\ifthenelse{\equal{\equationrestart}{none}}{
}{
\ifthenelse{\equal{\equationrestart}{sec}}{
}{
\ifthenelse{\equal{\equationrestart}{ssec}}{
}{
\ifthenelse{\equal{\equationrestart}{sssec}}{
}{
\ifthenelse{\equal{\equationrestart}{ssssec}}{
}{
	\throwbadconfig{Formato reinicio numero ecuacion desconocido}{\equationrestart}{none,sec,ssec,sssec,ssssec}}}}}
}

% -----------------------------------------------------------------------------
% Configuraciones del idioma
% -----------------------------------------------------------------------------

% Desactiva carácteres acentuados en operaciones matemáticas
\unaccentedoperators

% -----------------------------------------------------------------------------
% Configura número de objetos en el final del documento
% -----------------------------------------------------------------------------
\AtEndDocument{
	\addtocounter{equation}{\value{templateEquations}}
	\addtocounter{figure}{\value{templateFigures}}
	\addtocounter{lstlisting}{\value{templateListings}}
	\addtocounter{table}{\value{templateTables}}
}

% END

% INICIO DE LAS PÁGINAS
\begin{document}

% PORTADA
% Template:     Informe/Reporte LaTeX
% Documento:    Definición de portadas
% Versión:      5.5.4 (16/09/2018)
% Codificación: UTF-8
%
% Autor: Pablo Pizarro R. @ppizarror
%        Facultad de Ciencias Físicas y Matemáticas
%        Universidad de Chile
%        pablo.pizarro@ing.uchile.cl, ppizarror.com
%
% Manual template: [http://latex.ppizarror.com/Template-Informe/]
% Licencia MIT:    [https://opensource.org/licenses/MIT/]

% Importa las configuraciones
% Template:     Informe/Reporte LaTeX
% Documento:    Configuraciones adicionales de portadas
% Versión:      5.2.0 (21/05/2018)
% Codificación: UTF-8
%
% Autor: Pablo Pizarro R. @ppizarror
%        Facultad de Ciencias Físicas y Matemáticas
%        Universidad de Chile
%        pablo.pizarro@ing.uchile.cl, ppizarror.com
%
% Manual template: [http://latex.ppizarror.com/Template-Informe/]
% Licencia MIT:    [https://opensource.org/licenses/MIT/]

% style15 [A]
\def\headerimageA {departamentos/uchile}            % Imagen en el header
\def\headerimagescaleA {0.4}                        % Escala de la imagen

% style16 [B]
\def\portraitbackgroundimageB {ejemplos/portada-1}  % Imagen de fondo
\def\portraitbackgroundcolorB {ocre}                % Color principal

% style17 [C]
\def\portraitimageC {img/ejemplos/test-image}       % Imagen de la portada
\def\portraitimageboxedC {true}                     % Imagen recuadrada
\def\portraitimageboxedwidthC {0.5}                 % Grosor línea recuadro
\def\portraitimagewidthC {8cm}                      % Ancho de la imagen en cm

% style18 [D]
\def\portraitimageD {img/ejemplos/test-image-wrap}  % Imagen de la portada
\def\portraitimageboxedD {false}                    % Imagen recuadrada
\def\portraitimageboxedwidthD {0.5}                 % Grosor línea recuadro
\def\portraitimagewidthD {4cm}                      % Ancho de la imagen en cm
 % !FILE <NL,NODIST>

% Se escribe el header de la portada
\newpage
\renewcommand{\thepage}{\nameportraitpage} % !NL

% Estilos de portada
\ifthenelse{\equal{\portraitstyle}{style1}}{
	\setpagemargincm{\pagemarginleft}{\firstpagemargintop}{\pagemarginright}{\pagemarginbottom}
	\pagestyle{fancy}

	% Escribe el header
	\fancyhf{}
	\fancyhead[L]{
		\nombreuniversidad ~ \\
		\nombrefacultad ~ \\
		\departamentouniversidad ~ \\
		\vspace{-0.43cm}
	}
	\fancyhead[R]{
		\includegraphics[scale=\imagendepartamentoescala]{\imagendepartamento}
	}

	% Título y tema a tratar
	~ \\
	\vfill
	\begin{center}
		\textcolor{\portraittitlecolor}{
			{\noindent \Huge{\titulodelinforme} \vspace{0.5cm}} ~ \\
			{\noindent \large{\temaatratar}}
		}
	\end{center}

	% Tabla de integrantes
	\vfill
	\noindent
	\begin{minipage}{1.0\textwidth}
		\begin{flushright}
			\tablaintegrantes
		\end{flushright}
	\end{minipage}
}{
\ifthenelse{\equal{\portraitstyle}{style2}}{
	\setpagemargincm{\pagemarginleft}{\firstpagemargintop}{\pagemarginright}{\pagemarginbottom}
	\pagestyle{fancy}

	% Escribe el header
	\fancyhf{}
	\fancyhead[L]{
		\nombreuniversidad ~ \\
		\nombrefacultad ~ \\
		\departamentouniversidad ~ \\
		\vspace{-0.43cm}
	}
	\fancyhead[R]{
		\includegraphics[scale=\imagendepartamentoescala]{\imagendepartamento}
	}

	% Nombre de curso y título
	~ \\
	\vfill
	\begin{center}
		{\noindent \LARGE{\nombredelcurso} \vspace{0.3cm}} ~ \\
		\vspace*{1.5cm}
		\textcolor{\portraittitlecolor}{
			{\centering \noindent \Huge{\titulodelinforme} \vspace{0.3cm}} ~ \\
			{\noindent \large{\temaatratar}}
		}
	\end{center}

	% Tabla de integrantes
	\vfill
	\noindent
	\begin{minipage}{1.0\textwidth}
		\begin{flushright}
			\tablaintegrantes
		\end{flushright}
	\end{minipage}
}{
\ifthenelse{\equal{\portraitstyle}{style3}}{
	\setpagemargincm{\pagemarginleft}{\firstpagemargintop}{\pagemarginright}{\pagemarginbottom}
	\pagestyle{fancy}

	% Escribe el header
	\fancyhf{}
	\fancyhead[L]{
		\nombreuniversidad ~ \\
		\nombrefacultad ~ \\
		\departamentouniversidad ~ \\
		\vspace{-0.43cm}
	}
	\fancyhead[R]{
		\includegraphics[scale=\imagendepartamentoescala]{\imagendepartamento}
	}

	% Curso y código - título informe
	~ \\
	\vfill
	\begin{center}
		\vspace*{-1.0cm}
		{\noindent \huge{\nombredelcurso} \vspace{0.3cm}} ~ \\
		{\noindent \large{Código del curso: \codigodelcurso}} ~ \\
		\vspace*{1.8cm}
		\textcolor{\portraittitlecolor}{
			{\noindent \Huge{\titulodelinforme} \vspace{0.3cm}} ~ \\
			{\noindent \large{\temaatratar}}
		}
	\end{center}

	% Tabla de integrantes
	\vfill
	\noindent
	\begin{minipage}{1.0\textwidth}
		\begin{flushright}
			\tablaintegrantes
		\end{flushright}
	\end{minipage}
}{
\ifthenelse{\equal{\portraitstyle}{style4}}{
	\setpagemargincm{\pagemarginleft}{\pagemargintop}{\pagemarginright}{\pagemarginbottom}
	\thispagestyle{empty}

	% Escribe el header
	\vspace*{-1.5cm}
	\noindent \includegraphics[width=1.75cm]{departamentos/uchile2}
	\hspace*{-0.15cm}
	\begin{tabular}{l}
		\small \scshape{\MakeUppercase{\nombreuniversidad}} ~ \\
		\small \scshape{\MakeUppercase{\nombrefacultad}} ~ \\
		\small \scshape{\MakeUppercase{\departamentouniversidad}} ~ \\
		\vspace*{1.25cm}\mbox{}
	\end{tabular}

	% Título portada - curso
	\vfill
	\begin{center}
		{\fontsize{22pt}{10pt} \selectfont
			\noindent \textcolor{\portraittitlecolor}{\titulodelinforme} \vspace*{0.35cm}} ~ \\
		{\noindent \fontsize{10pt}{5pt} \selectfont \textcolor{\portraittitlecolor}{\codigodelcurso\ - \nombredelcurso}} ~ \\
		\vspace*{3cm}
	\end{center}

	% Tabla de integrantes
	\vfill
	\noindent
	\begin{minipage}{1.0\textwidth}
		\begin{flushright}
			\tablaintegrantes
		\end{flushright}
	\end{minipage}
}{
\ifthenelse{\equal{\portraitstyle}{style5}}{
	\setpagemargincm{\pagemarginleft}{\pagemargintop}{\pagemarginright}{\pagemarginbottom}
	\thispagestyle{empty}

	% Escribe el header
	\includegraphics[width=1.5cm]{departamentos/uchile3}
	\hspace{-0.2cm}
	\begin{tabular}{l}
		\small \scshape{\MakeUppercase{\nombreuniversidad}} ~ \\
		\small \scshape{\MakeUppercase{\nombrefacultad}} ~ \\
		\small \scshape{\MakeUppercase{\departamentouniversidad}} ~ \\
		\vspace*{1cm}\mbox{}
	\end{tabular}

	% Título informe - tema
	\vfill
	\begin{center}
		\fontsize{8mm}{9mm}\selectfont
		\textcolor{\portraittitlecolor}{
			\noindent \titulodelinforme ~ \\
			\noindent \temaatratar ~ \\
		}
		\vspace*{1cm}
		\footnotesize{\codigodelcurso\ - \nombredelcurso} ~ \\
		\vspace*{1.4cm}
	\end{center}

	% Tabla de integrantes
	\vfill
	\begin{center}
		\noindent \normalsize{\tablaintegrantes}
	\end{center}
}{
\ifthenelse{\equal{\portraitstyle}{style6}}{
	\setpagemargincm{\pagemarginleft}{\pagemargintop}{\pagemarginright}{\pagemarginbottom}
	\thispagestyle{empty}

	% Escribe el header
	\begin{wrapfigure}{l}{0.3\textwidth}
		\vspace{-0.69cm}
		\noindent \hspace{-1.10cm} \includegraphics[scale=1.35]{departamentos/fcfm2}
	\end{wrapfigure}
	\hspace*{0.3cm}
	\noindent \textsc{\color{red} \hspace{-2.2cm} \departamentouniversidad} ~ \\
	\hspace*{0.3cm}
	\noindent \textsc{\color{dgray} \hspace{-1.6cm} \nombrefacultad} ~ \\
	\hspace*{0.3cm}
	\noindent \textsc{\color{dgray} \hspace{-1.6cm} \nombreuniversidad} ~ \\
	\hspace*{0.3cm}
	\noindent \textsc{\color{dgray} \hspace{-1.6cm} \codigodelcurso \nombredelcurso} ~ \\

	% Título informe - tema
	\vfill
	\begin{center}
		\vspace*{0.5cm}
		{\color{dgray} \Large \textbf{\MakeUppercase{\temaatratar}}} ~ \\
		\noindent \rule{\linewidth}{0.3mm} ~ \\
		\Huge \textup \bfseries \textsc{\textcolor{\portraittitlecolor}{\titulodelinforme}} ~ \\
		\noindent \rule{\linewidth}{0.3mm} ~ \\
	\end{center}
	\begin{minipage}{.5\textwidth}
		~
	\end{minipage}

	% Tabla de integrantes
	\vfill
	\begin{minipage}{1.0\textwidth}
		\begin{flushright}
			\noindent \tablaintegrantes
		\end{flushright}
	\end{minipage}
}{
\ifthenelse{\equal{\portraitstyle}{style7}}{
	\setpagemargincm{\pagemarginleft}{\pagemargintop}{\pagemarginright}{\pagemarginbottom}
	\thispagestyle{empty}

	% Escribe el header
	\begin{center}
		\vspace*{-1.5cm}
		\includegraphics[scale=\imagendepartamentoescala]{\imagendepartamento}
		\hspace*{-0.15cm}
		\begin{tabular}{l}
			\vspace*{0.26cm}\mbox{} ~ \\
			\small \textsc{\MakeUppercase{\nombreuniversidad}} ~ \\
			\small \textsc{\MakeUppercase{\nombrefacultad}} ~ \\
			\small \textsc{\MakeUppercase{\departamentouniversidad}} ~ \\
			\vspace*{1.25cm}\mbox{}
		\end{tabular}
	\end{center}

	% Título informe - tema
	\vfill
	\begin{center}
		\noindent \rule{\textwidth}{0.4mm} \\ \vspace{0.3cm}
		{\huge \textcolor{\portraittitlecolor}{\titulodelinforme} \vspace{0.2cm} ~ \\}
		\noindent \rule{\textwidth}{0.4mm} ~ \\ \vspace{0.40cm}
		{\large \textcolor{\portraittitlecolor}{\temaatratar} ~ \\}
	\end{center}

	% Tabla de integrantes
	\vfill
	\noindent
	\begin{minipage}{1.0\textwidth}
		\begin{flushright}
			\scshape{\tablaintegrantes}
		\end{flushright}
	\end{minipage}
}{
\ifthenelse{\equal{\portraitstyle}{style8}}{
	\setpagemargincm{\pagemarginleft}{\pagemargintop}{\pagemarginright}{\pagemarginbottom}
	\thispagestyle{empty}

	% Escribe el header
	\begin{center}
		\vspace*{-1.0cm}
		\begin{tabular}{c}
			\includegraphics[scale=\imagendepartamentoescala]{\imagendepartamento} \vspace{0.5cm} ~ \\
			\small \scshape{\MakeUppercase{\nombreuniversidad}} ~ \\
			\small \scshape{\MakeUppercase{\nombrefacultad}} ~ \\
			\small \scshape{\MakeUppercase{\departamentouniversidad}}
		\end{tabular}
	\end{center}

	% Título informe - tema
	\vfill
	\begin{center}
		\noindent \rule{\textwidth}{0.4mm} \\ \vspace{0.3cm}
		{\huge \textcolor{\portraittitlecolor}{\titulodelinforme} \vspace{0.2cm} ~ \\}
		\noindent \rule{\textwidth}{0.4mm} ~ \\ \vspace{0.40cm}
		{\large \textcolor{\portraittitlecolor}{\temaatratar} ~ \\}
	\end{center}

	% Tabla de integrantes
	\vfill
	\noindent
	\begin{minipage}{1.0\textwidth}
		\begin{flushright}
			\scshape{\tablaintegrantes}
		\end{flushright}
	\end{minipage}
}{
\ifthenelse{\equal{\portraitstyle}{style9}}{
	\setpagemargincm{\pagemarginleft}{\pagemargintop}{\pagemarginright}{\pagemarginbottom}
	\thispagestyle{empty}

	% Título del informe
	\noindent \includegraphics[scale=\imagendepartamentoescala]{\imagendepartamento}
	\vfill
	\begin{center}
		\noindent \rule{\textwidth}{0.4mm} \\ \vspace{0.3cm}
		{\huge \textcolor{\portraittitlecolor}{\titulodelinforme} \vspace{0.2cm} \\}
		\noindent \rule{\textwidth}{0.4mm} \\ \vspace{0.35cm}
		{\large \textcolor{\portraittitlecolor}{\temaatratar} \\}
	\end{center}

	% Nombre universidad - departamento
	\vfill
	\begin{center}
		\begin{tabular}{c}
			\small \scshape{\MakeUppercase{\nombreuniversidad}} ~ \\
			\small \scshape{\MakeUppercase{\nombrefacultad}} ~ \\
			\small \scshape{\MakeUppercase{\departamentouniversidad}}
		\end{tabular}
	\end{center}

	% Tabla de integrantes
	\vfill
	\begin{center}
		\indent \scshape{\tablaintegrantes}
	\end{center}
}{
\ifthenelse{\equal{\portraitstyle}{style10}}{
	\setpagemargincm{\pagemarginleft}{\pagemargintop}{\pagemarginright}{\pagemarginbottom}
	\thispagestyle{empty}

	% Header
	~ \\
	\vfill
	\begin{center}
		\ifthenelse{\equal{\nombreuniversidad}{\xspace}}{
			\noindent {\large \textsc{\departamentouniversidad}}
		}{
			\noindent {\large \textsc{\nombreuniversidad, \departamentouniversidad}}
		}
		\vspace{1.0cm}
	\end{center}

	% Título informe
	\vfill
	\begin{center}
		\noindent {\large \scshape{\nombredelcurso}} \vspace{0.5cm} ~ \\
		\noindent {\large \scshape{\codigodelcurso}} \vspace{0.5cm} ~ \\
		\noindent \rule{\textwidth}{0.4mm} \\ \vspace{0.3cm}
		{\huge \bfseries \textcolor{\portraittitlecolor}{\titulodelinforme} \vspace{0.2cm} \\}
		\noindent \rule{\textwidth}{0.4mm} \\ \vspace{2.5cm}
	\end{center}

	% Tabla de integrantes
	\vfill
	\begin{center}
		\indent \tablaintegrantes
	\end{center}
	\vfill
	~ \\
}{
\ifthenelse{\equal{\portraitstyle}{style11}}{
	\setpagemargincm{\pagemarginleft}{\pagemargintop}{\pagemarginright}{\pagemarginbottom}
	\thispagestyle{empty}

	% Header
	\begin{center}
		\vspace*{-1.0cm}
		\scshape{\nombreuniversidad} ~ \\
		\scshape{\nombrefacultad} ~ \\
		\scshape{\departamentouniversidad}
	\end{center}

	% Título informe
	\vfill
	\begin{center}
		{\setstretch{1.2} \fontsize{21pt}{22pt} \selectfont \textcolor{\portraittitlecolor}{\scshape{\titulodelinforme}} \vspace{0.5cm}} ~ \\
		{\fontsize{13pt}{10pt} \selectfont \textcolor{\portraittitlecolor}{\scshape{\temaatratar}}}
	\end{center}

	% Tabla de integrantes
	\vfill
	\begin{center}
		\indent \tablaintegrantes
	\end{center}
}{
\ifthenelse{\equal{\portraitstyle}{style12}}{
	\setpagemargincm{\pagemarginleft}{\pagemargintop}{\pagemarginright}{\pagemarginbottom}
	\thispagestyle{empty}

	% Imagen departamento
	\begin{center}
		\vspace*{-1.0cm}
		\includegraphics[scale=\imagendepartamentoescala]{\imagendepartamento}
	\end{center}

	% Título informe
	\vfill
	\begin{center}
		{\bf \Huge \scshape{\textcolor{\portraittitlecolor}{\titulodelinforme}} \vspace{0.3cm}} \\
		{\bf \Large \textcolor{\portraittitlecolor}{\temaatratar}}
	\end{center}

	% Tabla de integrantes
	\vfill
	\begin{flushright}
		\noindent \tablaintegrantes
	\end{flushright}

	% Footer
	\vspace{0.5cm}
	\noindent \rule{\textwidth}{0.4mm}
	\begin{center}
		\ifthenelse{\equal{\nombreuniversidad}{\xspace}}{
			\scshape{\nombrefacultad} \\
		}{
			\scshape{\nombreuniversidad, \nombrefacultad} \\
		}
		\scshape{\departamentouniversidad}
	\end{center}
}{
\ifthenelse{\equal{\portraitstyle}{style13}}{
	\setpagemargincm{\pagemarginleft}{\pagemargintop}{\pagemarginright}{\pagemarginbottom}
	\thispagestyle{empty}

	% Header
	\noindent
	\vspace*{-1.5cm}
	\begin{flushleft}
		\begin{minipage}{0.65\textwidth}
			\ifthenelse{\equal{\nombreuniversidad}{\xspace}}{
				{\fontsize{3.5mm}{0.5mm} \selectfont \noindent \textsf{\nombrefacultad}} ~ \\
			}{
				{\fontsize{3.5mm}{0.5mm} \selectfont \noindent \textsf{\nombreuniversidad, \nombrefacultad}} ~ \\
			}
			\noindent {\fontsize{3.0mm}{0.5mm} \selectfont \textsf{\departamentouniversidad} \vspace{-0.2cm}} ~ \\
			\noindent \textcolor{lgray}{\rule{\textwidth}{0.3mm}}
		\end{minipage}
	\end{flushleft}
	\vspace*{-2.15cm}
	\begin{flushright}
		\begin{minipage}{0.3\textwidth}
			\noindent \includegraphics[width=1.0\textwidth]{\imagendepartamento}
		\end{minipage}
	\end{flushright}

	% Título informe
	\vfill
	\begin{center}
		\begin{minipage}{0.9\textwidth}
			\begin{framed}
				\LARGE
				\vspace{1cm}
				\centering \textcolor{\portraittitlecolor}{\textbf{\titulodelinforme}}
				\vspace{1cm}
			\end{framed}
		\end{minipage}
	\end{center}

	% Tabla de integrantes
	\vfill
	\begin{flushright}
		\noindent \textsf{\tablaintegrantes}
	\end{flushright}
}{
\ifthenelse{\equal{\portraitstyle}{style14}}{
	\setpagemargincm{\pagemarginleft}{\pagemargintop}{\pagemarginright}{\pagemarginbottom}
	\thispagestyle{empty}

	% Header
	\noindent
	\begin{flushleft}
		\vspace*{-1.0cm}
		\noindent \includegraphics[scale=\imagendepartamentoescala]{\imagendepartamento} \\
	\end{flushleft}

	% Título del informe
	\vfill
	{\bf \huge \noindent \textcolor{\portraittitlecolor}{\textsf{\MakeUppercase{\titulodelinforme}} \vspace*{0.05cm}}} \\
	{\bf \large \noindent \textcolor{\portraittitlecolor}{\textsf{\MakeUppercase{\temaatratar}}}} \\

	% Tabla de integrantes
	\vfill
	\begin{flushright}
		\noindent \textsf{\tablaintegrantes}
	\end{flushright}
}{
\ifthenelse{\equal{\portraitstyle}{style15}}{
	\setpagemargincm{\pagemarginleft}{\pagemargintop}{\pagemarginright}{\pagemarginbottom}
	\thispagestyle{empty}
	\checkextravarexist{\headerimageA}{Defina la imagen extra de la portada en el archivo lib/page/portrait-config.tex (VERSION NORMAL) o bien en el bloque PORTADA (VERSION COMPACTA)}
	\checkextravarexist{\headerimagescaleA}{Defina la escala de la imagen extra de la portada en el archivo lib/page/portrait-config.tex (VERSION NORMAL) o bien en el bloque PORTADA (VERSION COMPACTA)}

	% Header
	\vspace*{-1.5cm}
	\noindent \begin{minipage}{0.8\textwidth}
		\noindent \begin{minipage}{0.22\textwidth}
			\includegraphics[scale=1.0]{departamentos/fcfm2} \\
		\end{minipage}
		\begin{minipage}{0.6\textwidth}
			\begin{flushleft}
				\textsc{
				\begin{tabular}{l}
					{\small \nombreuniversidad} ~ \\
					{\small \nombrefacultad} ~ \\
					{\small \departamentouniversidad}
				\end{tabular}
				}
			\end{flushleft}
		\end{minipage}
	\end{minipage}
	\noindent \begin{minipage}{0.2\textwidth}
		\begin{flushright}
			\ifthenelse{\isundefined{\headerimageA}}{}{
				\ifthenelse{\isundefined{\headerimagescaleA}}{}{
					\noindent \includegraphics[scale=\headerimagescaleA]{\headerimageA} \\
				}
			}
		\end{flushright}
	\end{minipage}

	% Título informe
	\vfill
	\begin{center}
		{\fontsize{25pt}{15pt} \selectfont \textcolor{\portraittitlecolor}{\textbf{\titulodelinforme}} \vspace{0.7cm}} \\
		{\Large \textcolor{\portraittitlecolor}{\temaatratar}}
	\end{center}

	% Tabla de integrantes
	\vfill
	\begin{center}
		\noindent \tablaintegrantes
	\end{center}
}{
\ifthenelse{\equal{\portraitstyle}{style16}}{
	\setpagemargincm{\pagemarginleft}{\pagemargintop}{\pagemarginright}{\pagemarginbottom}
	\checkextravarexist{\portraitbackgroundimageB}{Defina el fondo de la portada en el archivo lib/page/portrait-config.tex (VERSION NORMAL) o bien en el bloque PORTADA (VERSION COMPACTA)}
	\checkextravarexist{\portraitbackgroundcolorB}{Defina el color del bloque del titulo de la portada en el archivo lib/page/portrait-config.tex (VERSION NORMAL) o bien en el bloque PORTADA (VERSION COMPACTA)}

	% Imagen fondo, título informe
	\begingroup
		\thispagestyle{empty}
		\begin{tikzpicture}[remember picture,overlay]
			\node[inner sep=0pt] (background) at (current page.center) {\includegraphics[width=\paperwidth]{\portraitbackgroundimageB}};
			\draw (current page.center) node [fill=\portraitbackgroundcolorB!30!white,fill opacity=0.6,text opacity=1,inner sep=1cm]{\Huge\centering\bfseries\sffamily\parbox[c][][t]{\paperwidth}{
					\centering \textcolor{\portraittitlecolor}{\titulodelinforme} \\ [10pt]
					{\Large \temaatratar} \\ [25pt]
					{\huge \autordeldocumento}}};
		\end{tikzpicture}
		\vfill
	\endgroup
}{
\ifthenelse{\equal{\portraitstyle}{style17}}{
	\setpagemargincm{\pagemarginleft}{\firstpagemargintop}{\pagemarginright}{\pagemarginbottom}
	\pagestyle{fancy}
	\checkextravarexist{\portraitimageC}{[portrait-style17] Defina la imagen de la portada en el archivo lib/page/portrait-config.tex (VERSION NORMAL) o bien en el bloque PORTADA (VERSION COMPACTA)}
	\checkextravarexist{\portraitimageboxedC}{Defina si la imagen de la portada se encierra en un recuadro en el archivo lib/page/portrait-config.tex (VERSION NORMAL) o bien en el bloque PORTADA (VERSION COMPACTA)}
	\checkextravarexist{\portraitimageboxedwidthC}{Defina el grosor del recuadro de la imagen de la portada en el archivo lib/page/portrait-config.tex (VERSION NORMAL) o bien en el bloque PORTADA (VERSION COMPACTA)}
	\checkextravarexist{\portraitimagewidthC}{Defina los parametros de la imagen de la portada en el archivo lib/page/portrait-config.tex (VERSION NORMAL) o bien en el bloque PORTADA (VERSION COMPACTA)}

	% Header
	\fancyhf{}
	\fancyhead[L]{
		\nombreuniversidad ~ \\
		\nombrefacultad ~ \\
		\departamentouniversidad ~ \\
		\vspace{-0.43cm}
	}
	\fancyhead[R]{
		\includegraphics[scale=\imagendepartamentoescala]{\imagendepartamento}
	}

	% Título del informe
	~ \\
	\vfill
	\begin{center}
		\textcolor{\portraittitlecolor}{
			{\noindent \Huge{\titulodelinforme} \vspace{0.5cm}} ~ \\
			{\noindent \large{\temaatratar}}
		}
	\end{center}

	% Imagen de la portada
	~ \\
	\ifthenelse{\equal{\portraitimageboxedC}{true}}{
		\insertimageboxed{\portraitimageC}{width=\portraitimagewidthC}{\portraitimageboxedwidthC}{}
	}{
		\insertimage{\portraitimageC}{width=\portraitimagewidthC}{}
	}
	~ \\
	\vfill

	% Tabla de integrantes
	\noindent
	\begin{minipage}{1.0\textwidth}
		\begin{flushright}
			\tablaintegrantes
		\end{flushright}
	\end{minipage}
}{
\ifthenelse{\equal{\portraitstyle}{style18}}{
	\setpagemargincm{\pagemarginleft}{\firstpagemargintop}{\pagemarginright}{\pagemarginbottom}
	\pagestyle{fancy}
	\checkextravarexist{\portraitimageD}{[portrait-style17] Defina la imagen de la portada en el archivo lib/page/portrait-config.tex (VERSION NORMAL) o bien en el bloque PORTADA (VERSION COMPACTA)}
	\checkextravarexist{\portraitimageboxedD}{Defina si la imagen de la portada se encierra en un recuadro en el archivo lib/page/portrait-config.tex (VERSION NORMAL) o bien en el bloque PORTADA (VERSION COMPACTA)}
	\checkextravarexist{\portraitimageboxedwidthD}{Defina el grosor del recuadro de la imagen de la portada en el archivo lib/page/portrait-config.tex (VERSION NORMAL) o bien en el bloque PORTADA (VERSION COMPACTA)}
	\checkextravarexist{\portraitimagewidthD}{Defina los parametros de la imagen de la portada en el archivo lib/page/portrait-config.tex (VERSION NORMAL) o bien en el bloque PORTADA (VERSION COMPACTA)}

	% Header
	\fancyhf{}
	\fancyhead[L]{
		\nombreuniversidad ~ \\
		\nombrefacultad ~ \\
		\departamentouniversidad ~ \\
		\vspace{-0.43cm}
	}
	\fancyhead[R]{
		\includegraphics[scale=\imagendepartamentoescala]{\imagendepartamento}
	}

	% Imagen de la portada
	~ \\
	\ifthenelse{\equal{\portraitimageboxedD}{true}}{
		\insertimageboxed{\portraitimageD}{width=\portraitimagewidthD}{\portraitimageboxedwidthD}{}
	}{
		\insertimage{\portraitimageD}{width=\portraitimagewidthD}{}
	}

	% Título del informe
	\vfill
	\begin{center}
		\textcolor{\portraittitlecolor}{
			{\noindent \Huge{\titulodelinforme} \vspace{0.5cm}} ~ \\
			{\noindent \large{\temaatratar}}
		}
	\end{center}
	\vfill

	% Tabla de integrantes
	\noindent
	\begin{minipage}{1.0\textwidth}
		\begin{flushright}
			\tablaintegrantes
		\end{flushright}
	\end{minipage}
}{
\ifthenelse{\equal{\portraitstyle}{\bgtemplatetestcode}}{
	\setpagemargincm{\pagemarginleft}{\pagemargintop}{\pagemarginright}{\pagemarginbottom}
	\pagestyle{empty}
	\pagecolor{lbrown}
	\begin{center}
		\vspace*{-1.0cm}
		\scshape{\nombreuniversidad} ~ \\
		\scshape{\nombrefacultad} ~ \\
		\scshape{\departamentouniversidad}
	\end{center}
	~ \\
	\begin{center}
		\bgtemplatetestimg
	\end{center}
	\begin{center}
		\vspace*{-6cm}
		{\setstretch{1.2} \fontsize{25pt}{22pt} \selectfont \textcolor{\portraittitlecolor}{\scshape{\titulodelinforme}} \vspace{0.5cm}} \\
		{\fontsize{15pt}{10pt} \selectfont \textcolor{\portraittitlecolor}{\scshape{\temaatratar}}}
	\end{center}
	\vfill
	\begin{flushright}
		\noindent \tablaintegrantes
	\end{flushright}
	\newpage
	\pagecolor{white}
}{
	\throwbadconfigondoc{Estilo de portada incorrecto}{\portraitstyle}{style1 .. style18}}}}}}}}}}}}}}}}}}}
}

% Añade una página en blanco al imprimir por las dos caras
\ifthenelse{\equal{\addemptypagetwosides}{true}}{
	\newpage
	\null
	\thispagestyle{empty}
	\renewcommand{\thepage}{}
	\newpage}{
}


% CONFIGURACIÓN DE PÁGINA Y ENCABEZADOS
% Template:     Informe/Reporte LaTeX
% Documento:    Configuración de página
% Versión:      5.7.5 (02/10/2018)
% Codificación: UTF-8
%
% Autor: Pablo Pizarro R. @ppizarror
%        Facultad de Ciencias Físicas y Matemáticas
%        Universidad de Chile
%        pablo.pizarro@ing.uchile.cl, ppizarror.com
%
% Manual template: [http://latex.ppizarror.com/Template-Informe/]
% Licencia MIT:    [https://opensource.org/licenses/MIT/]

% -----------------------------------------------------------------------------
% Numeración de páginas
% -----------------------------------------------------------------------------
\newpage
\ifthenelse{\equal{\romanpageuppercase}{true}}{
	\pagenumbering{Roman}
}{
	\pagenumbering{roman}
}
\setcounter{page}{1}
\setcounter{footnote}{1}

% -----------------------------------------------------------------------------
% Márgenes de páginas y tablas
% -----------------------------------------------------------------------------
\setpagemargincm{\pagemarginleft}{\pagemargintop}{\pagemarginright}{\pagemarginbottom}
\def\arraystretch {\tablepaddingv} % Se ajusta el padding vertical de las tablas
\setlength{\tabcolsep}{\tablepaddingh em} % Se ajusta el padding horizontal de las tablas

% -----------------------------------------------------------------------------
% Se define el punto decimal
% -----------------------------------------------------------------------------
\ifthenelse{\equal{\pointdecimal}{true}}{
	\decimalpoint}{
}

% -----------------------------------------------------------------------------
% Definición de nombres de objetos
% -----------------------------------------------------------------------------
\renewcommand{\appendixname}{\nomltappendixsection} % Nombre del anexo (título)
\renewcommand{\appendixpagename}{\nameappendixsection} % Nombre del anexo en índice
\renewcommand{\appendixtocname}{\nameappendixsection} % Nombre del anexo en índice
\renewcommand{\contentsname}{\nomltcont} % Nombre del índice
\renewcommand{\figurename}{\nomltwfigure} % Nombre de la leyenda de las fig.
\renewcommand{\listfigurename}{\nomltfigure} % Nombre del índice de figuras
\renewcommand{\listtablename}{\nomlttable} % Nombre del índice de tablas
\renewcommand{\lstlistingname}{\nomltwsrc} % Nombre leyenda del código fuente
\renewcommand{\lstlistlistingname}{\nomltsrc} % Nombre índice código fuente
\renewcommand{\refname}{\namereferences} % Nombre de las referencias
\renewcommand{\tablename}{\nomltwtable} % Nombre de la leyenda de tablas

% -----------------------------------------------------------------------------
% Estilo de títulos
% -----------------------------------------------------------------------------
\sectionfont{\color{\titlecolor} \fontsizetitle \styletitle \selectfont}
\subsectionfont{\color{\subtitlecolor} \fontsizesubtitle \stylesubtitle \selectfont}
\subsubsectionfont{\color{\subsubtitlecolor} \fontsizesubsubtitle \stylesubsubtitle \selectfont}
\titleformat{\subsubsubsection}{\color{\ssstitlecolor} \normalfont \fontsizessstitle \stylessstitle}{\thesubsubsubsection}{1em}{}
\titlespacing*{\subsubsubsection}{0pt}{3.25ex plus 1ex minus .2ex}{1.5ex plus .2ex}

% -----------------------------------------------------------------------------
% Se crean los header-footer
% -----------------------------------------------------------------------------
\ifthenelse{\equal{\hfstyle}{style1}}{
	\pagestyle{fancy} \fancyhf{}
	\fancyhead[L]{\nouppercase{\rightmark}}
	\fancyhead[R]{\small \rm \thepage}
	\fancyfoot[L]{\small \rm \textit{\titulodelinforme}}
	\fancyfoot[R]{\small \rm \textit{\codigodelcurso \nombredelcurso}}
	\renewcommand{\headrulewidth}{0.5pt}
	\renewcommand{\footrulewidth}{0.5pt}
	\renewcommand{\sectionmark}[1]{\markboth{#1}{}}
}{
\ifthenelse{\equal{\hfstyle}{style2}}{
	\pagestyle{fancy} \fancyhf{}
	\fancyhead[L]{\nouppercase{\rightmark}}
	\fancyhead[R]{\small \rm \thepage}
	\fancyfoot[L]{\small \rm \textit{\titulodelinforme}}
	\fancyfoot[R]{\small \rm \textit{\codigodelcurso \nombredelcurso}}
	\renewcommand{\headrulewidth}{0.5pt}
	\renewcommand{\footrulewidth}{0pt}
	\renewcommand{\sectionmark}[1]{\markboth{#1}{}}
}{
\ifthenelse{\equal{\hfstyle}{style3}}{
	\pagestyle{fancy} \fancyhf{}
	\fancyhead[L]{
		\small \rm \textit{\codigodelcurso \nombredelcurso}
		\vspace{0.04cm}
	}
	\fancyhead[R]{
		\includegraphics[width=1.2cm]{\imagendepartamento}
		\vspace{-0.10cm}
	}
	\fancyfoot[C]{\thepage}
	\renewcommand{\headrulewidth}{0.5pt}
	\renewcommand{\footrulewidth}{0pt}
}{
\ifthenelse{\equal{\hfstyle}{style4}}{
	\pagestyle{fancy} \fancyhf{}
	\fancyhead[L]{\nouppercase{\rightmark}}
	\fancyhead[R]{}
	\fancyfoot[C]{\small \rm \thepage}
	\renewcommand{\headrulewidth}{0.5pt}
	\renewcommand{\footrulewidth}{0pt}
	\renewcommand{\sectionmark}[1]{\markboth{#1}{}}
}{
\ifthenelse{\equal{\hfstyle}{style5}}{
	\pagestyle{fancy} \fancyhf{}
	\fancyhead[L]{\codigodelcurso \nombredelcurso}
	\fancyhead[R]{\nouppercase{\rightmark}}
	\fancyfoot[L]{\departamentouniversidad, \nombreuniversidad}
	\fancyfoot[R]{\small \rm \thepage}
	\renewcommand{\headrulewidth}{0pt}
	\renewcommand{\footrulewidth}{0pt}
	\renewcommand{\sectionmark}[1]{\markboth{#1}{}}
}{
\ifthenelse{\equal{\hfstyle}{style6}}{
	\pagestyle{fancy} \fancyhf{}
	\fancyfoot[L]{\departamentouniversidad}
	\fancyfoot[C]{\thepage}
	\fancyfoot[R]{\nombreuniversidad}
	\renewcommand{\headrulewidth}{0pt}
	\renewcommand{\footrulewidth}{0pt}
	\setlength{\headheight}{49pt}
}{
\ifthenelse{\equal{\hfstyle}{style7}}{
	\pagestyle{fancy} \fancyhf{}
	\fancyfoot[C]{\thepage}
	\renewcommand{\headrulewidth}{0pt}
	\renewcommand{\footrulewidth}{0pt}
	\setlength{\headheight}{49pt}
}{
\ifthenelse{\equal{\hfstyle}{style8}}{
	\pagestyle{fancy} \fancyhf{}
	\fancyfoot[R]{\thepage}
	\renewcommand{\headrulewidth}{0pt}
	\renewcommand{\footrulewidth}{0pt}
	\setlength{\headheight}{49pt}
}{
\ifthenelse{\equal{\hfstyle}{style9}}{
	\pagestyle{fancy} \fancyhf{}
	\fancyhead[L]{\nouppercase{\rightmark}}
	\fancyhead[R]{}
	\fancyfoot[L]{\small \rm \textit{\titulodelinforme}}
	\fancyfoot[R]{\small \rm \thepage}
	\renewcommand{\headrulewidth}{0.5pt}
	\renewcommand{\footrulewidth}{0.5pt}
	\renewcommand{\sectionmark}[1]{\markboth{#1}{}}
}{
\ifthenelse{\equal{\hfstyle}{style10}}{
	\pagestyle{fancy} \fancyhf{}
	\fancyhead[L]{\nouppercase{\rightmark}}
	\fancyhead[R]{\small \rm \textit{\titulodelinforme}}
	\fancyfoot[L]{}
	\fancyfoot[R]{\small \rm \thepage}
	\renewcommand{\headrulewidth}{0.5pt}
	\renewcommand{\footrulewidth}{0.5pt}
	\renewcommand{\sectionmark}[1]{\markboth{#1}{}}
}{
\ifthenelse{\equal{\hfstyle}{style11}}{ % Similar a 1
	\pagestyle{fancy} \fancyhf{}
	\fancyhead[L]{\nouppercase{\rightmark}}
	\fancyhead[R]{\small \rm \thepage \nomnpageof \pageref{LastPage}}
	\fancyfoot[L]{\small \rm \textit{\titulodelinforme}}
	\fancyfoot[R]{\small \rm \textit{\codigodelcurso \nombredelcurso}}
	\renewcommand{\headrulewidth}{0.5pt}
	\renewcommand{\footrulewidth}{0.5pt}
	\renewcommand{\sectionmark}[1]{\markboth{#1}{}}
}{
\ifthenelse{\equal{\hfstyle}{style12}}{ % Similar a 6
	\pagestyle{fancy} \fancyhf{}
	\fancyfoot[L]{\departamentouniversidad}
	\fancyfoot[C]{\thepage \nomnpageof \pageref{LastPage}}
	\fancyfoot[R]{\nombreuniversidad}
	\renewcommand{\headrulewidth}{0pt}
	\renewcommand{\footrulewidth}{0pt}
	\setlength{\headheight}{49pt}
}{
\ifthenelse{\equal{\hfstyle}{style13}}{ % Similar a 3
	\pagestyle{fancy} \fancyhf{}
	\fancyhead[L]{
		\small \rm \textit{\codigodelcurso \nombredelcurso}
		\vspace{0.04cm}
	}
	\fancyhead[R]{
		\includegraphics[width=1.2cm]{\imagendepartamento}
		\vspace{-0.10cm}
	}
	\fancyfoot[C]{\thepage \nomnpageof \pageref{LastPage}}
	\renewcommand{\headrulewidth}{0.5pt}
	\renewcommand{\footrulewidth}{0pt}
}{
\ifthenelse{\equal{\hfstyle}{style14}}{ % Similar a 4
	\pagestyle{fancy} \fancyhf{}
	\fancyhead[L]{\nouppercase{\rightmark}}
	\fancyhead[R]{}
	\fancyfoot[C]{\small \rm \thepage \nomnpageof \pageref{LastPage}}
	\renewcommand{\headrulewidth}{0.5pt}
	\renewcommand{\footrulewidth}{0pt}
	\renewcommand{\sectionmark}[1]{\markboth{#1}{}}
}{
	\throwbadconfigondoc{Estilo de header-footer incorrecto}{\hfstyle}{style1 .. style14}}}}}}}}}}}}}}
}

% -----------------------------------------------------------------------------
% Muestra los números de línea
% -----------------------------------------------------------------------------
\ifthenelse{\equal{\showlinenumbers}{true}}{
	\linenumbers}{
}

% END

% RESUMEN O ABSTRACT
\begin{resumen}
	\lipsum[1] % Párrafo ejemplo, se puede borrar
\end{resumen}

% TABLA DE CONTENIDOS - ÍNDICE
% Template:     Informe/Reporte LaTeX
% Documento:    Índice
% Versión:      DEV
% Codificación: UTF-8
%
% Autor: Pablo Pizarro R. @ppizarror
%        Facultad de Ciencias Físicas y Matemáticas
%        Universidad de Chile
%        pablo.pizarro@ing.uchile.cl, ppizarror.com
%
% Manual template: [https://latex.ppizarror.com/Template-Informe/]
% Licencia MIT:    [https://opensource.org/licenses/MIT/]

% Sección inicio
\ifthenelse{\equal{\showindex}{true}}{
	
	% -------------------------------------------------------------------------
	% Crea nueva página y establece estilo de títulos
	% -------------------------------------------------------------------------
	\newpage
	\begingroup
	\sectionfont{\color{\indextitlecolor} \fontsizetitlei \styletitlei \selectfont}
	
	% -------------------------------------------------------------------------
	% Salta de página si está imprimiendo por ambas caras
	% -------------------------------------------------------------------------
	\ifthenelse{\equal{\addemptypagetwosides}{true}}{
		\checkoddpage
		\ifoddpage
		\else
			\newpage
			\null
			\thispagestyle{empty}
			\newpage
			\addtocounter{page}{-1}
		\fi}{
	}
	
	% -------------------------------------------------------------------------
	% Añade la entrada del índice a los marcadores del pdf
	% -------------------------------------------------------------------------
	\ifthenelse{\equal{\addindextobookmarks}{true}}{
		\belowpdfbookmark{\nomltcont}{contents}}{
	}
	\tocloftpagestyle{fancy}
	
	% -------------------------------------------------------------------------
	% Configuración del punto en índice
	% -------------------------------------------------------------------------
	\ifthenelse{\equal{\showdotontitles}{true}}{
		
		% Agrega los puntos
		\def\cftsecaftersnum {.}
		\def\cftsubsecaftersnum {.}
		\def\cftsubsubsecaftersnum {.}
		\def\cftsubsubsubsecaftersnum {.}
		
		% Modifica los márgenes
		\def\cftsecnumwidth {1.9em}
		\def\cftsubsecnumwidth {2.57em} % Incremento 0.67
		\renewcommand\cftsubsubsecnumwidth{3.35em} % Incremento 0.78
		
		\setlength{\cftsubsecindent}{1.91em}
		\setlength{\cftsubsubsecindent}{4.48em} % Incremento 2.57
		}{
	}

	% -------------------------------------------------------------------------
	% Configuración del punto en número de objetos
	% -------------------------------------------------------------------------
	\def\cftfigaftersnum {\charafterobjectindex\enspace}
	\def\cftsubfigaftersnum {\charafterobjectindex\enspace}
	\def\cfttabaftersnum {\charafterobjectindex\enspace}
	\def\cftlstlistingaftersnum {\charafterobjectindex\enspace}
	
	% -------------------------------------------------------------------------
	% Configuración carácter número de página
	% -------------------------------------------------------------------------
	\renewcommand{\cftdot}{\charnumpageindex}
	
	% -------------------------------------------------------------------------
	% Desactiva los números de línea
	% -------------------------------------------------------------------------
	\ifthenelse{\equal{\showlinenumbers}{true}}{
		\nolinenumbers}{
	}

	% -------------------------------------------------------------------------
	% Cambia tabulación índice de objetos para alinear con contenidos
	% -------------------------------------------------------------------------
	\ifthenelse{\equal{\objectindexindent}{true}}{
		\def\cftlstlistingindent {1.495em}
	}{
		\setlength{\cfttabindent}{0in}
		\setlength{\cftfigindent}{0in}
		\setlength{\cftsubfigindent}{0in}
		\setlength{\cftfigindent}{0in}
		\def\cftlstlistingindent {0.01em}
	}
	
	% -------------------------------------------------------------------------
	% Iguala tamaño del margen en números de objetos
	% -------------------------------------------------------------------------
	\ifthenelse{\equal{\equalmarginnumobject}{true}}{
		
		% Calcula tamaño del margen de los números en objetos del índice dependiendo de la configuración
		\ifthenelse{\equal{\showsectioncaption}{none}}{
			\def\cftdefautnumwidth {2.3em}
		}{
		\ifthenelse{\equal{\showsectioncaption}{sec}}{
			\def\cftdefautnumwidth {3.0em}
		}{
		\ifthenelse{\equal{\showsectioncaption}{ssec}}{
			\def\cftdefautnumwidth {3.7em}
		}{
		\ifthenelse{\equal{\showsectioncaption}{sssec}}{
			\def\cftdefautnumwidth {4.4em}
		}{
		\ifthenelse{\equal{\showsectioncaption}{ssssec}}{
			\def\cftdefautnumwidth {5.1em}
		}{
			\throwbadconfig{Valor configuracion incorrecto}{\showsectioncaption}{none,sec,ssec,sssec,ssssec}}}}}
		}
		
		% Configuración identado de títulos de objetos después del número
		\def\cftfignumwidth {\cftdefautnumwidth}
		\def\cftsubfignumwidth {\cftdefautnumwidth}
		\def\cfttabnumwidth {\cftdefautnumwidth}
		\def\cftlstlistingnumwidth {\cftdefautnumwidth}}{
		
	}

	% -------------------------------------------------------------------------
	% Genera las funciones para los índices
	% -------------------------------------------------------------------------
	\newcommand{\coregeneratefigureindex}{
		\iftotalfigures
			\ifthenelse{\equal{\indexnewpagef}{true}}{\newpage}{}
			\listoffigures
		\fi
	}
	\newcommand{\coregeneratetableindex}{
		\iftotaltables
			\ifthenelse{\equal{\indexnewpaget}{true}}{\newpage}{}
			\listoftables
		\fi
	}
	\newcommand{\coregeneratecodeindex}{
		\iftotallstlistings
			\ifthenelse{\equal{\indexnewpagec}{true}}{\newpage}{}
			\lstlistoflistings
		\fi
	}

	% -------------------------------------------------------------------------
	% Índice de contenidos
	% -------------------------------------------------------------------------
	\ifthenelse{\equal{\showindexofcontents}{true}}{
		\tableofcontents
	}{}
	
	% -------------------------------------------------------------------------
	% Índice de objetos
	% -------------------------------------------------------------------------
	\ifthenelse{\equal{\indexstyle}{ftc}}{
		\coregeneratefigureindex
		\coregeneratetableindex
		\coregeneratecodeindex
	}{
	\ifthenelse{\equal{\indexstyle}{f}}{
		\coregeneratefigureindex
	}{
	\ifthenelse{\equal{\indexstyle}{ft}}{
		\coregeneratefigureindex
		\coregeneratetableindex
	}{
	\ifthenelse{\equal{\indexstyle}{fc}}{
		\coregeneratefigureindex
		\coregeneratecodeindex
	}{
	\ifthenelse{\equal{\indexstyle}{fct}}{
		\coregeneratefigureindex
		\coregeneratecodeindex
		\coregeneratetableindex
	}{
	\ifthenelse{\equal{\indexstyle}{t}}{
		\coregeneratetableindex
	}{
	\ifthenelse{\equal{\indexstyle}{tf}}{
		\coregeneratetableindex
		\coregeneratefigureindex
	}{
	\ifthenelse{\equal{\indexstyle}{tfc}}{
		\coregeneratetableindex
		\coregeneratefigureindex
		\coregeneratecodeindex
	}{
	\ifthenelse{\equal{\indexstyle}{tc}}{
		\coregeneratetableindex
		\coregeneratecodeindex
	}{
	\ifthenelse{\equal{\indexstyle}{tcf}}{
		\coregeneratetableindex
		\coregeneratecodeindex
		\coregeneratefigureindex
	}{
	\ifthenelse{\equal{\indexstyle}{c}}{
		\coregeneratecodeindex
	}{
	\ifthenelse{\equal{\indexstyle}{ct}}{
		\coregeneratecodeindex
		\coregeneratetableindex
	}{
	\ifthenelse{\equal{\indexstyle}{ctf}}{
		\coregeneratecodeindex
		\coregeneratetableindex
		\coregeneratefigureindex
	}{
	\ifthenelse{\equal{\indexstyle}{cf}}{
		\coregeneratecodeindex
		\coregeneratefigureindex
	}{
	\ifthenelse{\equal{\indexstyle}{cft}}{
		\coregeneratecodeindex
		\coregeneratefigureindex
		\coregeneratetableindex
	}{
	\ifthenelse{\equal{\indexstyle}{}}{
	}{
		\throwbadconfig{Estilo desconocido del indice}{\indexstyle}{,f,ft,ftc,fc,fct,t,tf,tfc,tc,tcf,c,ct,ctf,cf,cft}}}}}}}}}}}}}}}}
	}

	% -------------------------------------------------------------------------
	% Termina el bloque de índice
	% -------------------------------------------------------------------------
	\endgroup
	
	% -------------------------------------------------------------------------
	% Se añade una página en blanco
	% -------------------------------------------------------------------------
	\newpage
	\ifthenelse{\equal{\addemptypagetwosides}{true}}{
		\vfill
		\checkoddpage
		\ifoddpage
			\newpage
			\null
			\thispagestyle{empty}
			\newpage
			\addtocounter{page}{-1}
		\else
		\fi}{
	}

}{}

% END % Índice, se puede borrar

% CONFIGURACIONES FINALES
% Template:     Informe/Reporte LaTeX
% Documento:    Configuraciones finales
% Versión:      DEV
% Codificación: UTF-8
%
% Autor: Pablo Pizarro R. @ppizarror
%        Facultad de Ciencias Físicas y Matemáticas
%        Universidad de Chile
%        pablo.pizarro@ing.uchile.cl, ppizarror.com
%
% Manual template: [https://latex.ppizarror.com/Template-Informe/]
% Licencia MIT:    [https://opensource.org/licenses/MIT/]

% -----------------------------------------------------------------------------
% Se reestablecen headers y footers
% -----------------------------------------------------------------------------
\markboth{}{}
\newpage

\ifthenelse{\equal{\disablehfrightmark}{false}}{
	\ifthenelse{\equal{\hfstyle}{style1}}{
		\fancyhead[L]{\nouppercase{\leftmark}}}{
	}
	\ifthenelse{\equal{\hfstyle}{style2}}{
		\fancyhead[L]{\nouppercase{\leftmark}}}{
	}
	\ifthenelse{\equal{\hfstyle}{style4}}{
		\fancyhead[L]{\nouppercase{\leftmark}}}{
	}
	\ifthenelse{\equal{\hfstyle}{style5}}{
		\ifthenelse{\equal{\hfwidthwrap}{true}}{
			\fancyhead[R]{
				\begin{minipage}[t]{\hfwidthtitle\linewidth}
					\begin{flushright}
						\nouppercase{\leftmark}
					\end{flushright}
				\end{minipage}
			}
		}{
			\fancyhead[R]{\nouppercase{\leftmark}}
		}}{
	}
	\ifthenelse{\equal{\hfstyle}{style9}}{
		\fancyhead[L]{\nouppercase{\leftmark}}}{
	}
	\ifthenelse{\equal{\hfstyle}{style10}}{
		\ifthenelse{\equal{\hfwidthwrap}{true}}{
			\fancyhead[L]{
				\begin{minipage}[t]{\hfwidthtitle\linewidth}
					\begin{flushleft}
						\nouppercase{\leftmark}
					\end{flushleft}
				\end{minipage}
			}
		}{
			\fancyhead[L]{\nouppercase{\leftmark}}
		}}{
	}
	\ifthenelse{\equal{\hfstyle}{style11}}{ % Similar a 1
		\fancyhead[L]{\nouppercase{\leftmark}}}{
	}
	\ifthenelse{\equal{\hfstyle}{style14}}{ % Similar a 4
		\fancyhead[L]{\nouppercase{\leftmark}}}{
	}
	\ifthenelse{\equal{\hfstyle}{style15}}{ % Similar a 1
		\fancyhead[L]{\nouppercase{\leftmark}}}{
	}
	}{
}

% -----------------------------------------------------------------------------
% Estilo de títulos - reestablece estilos por el índice
% -----------------------------------------------------------------------------
\sectionfont{\color{\titlecolor} \fontsizetitle \styletitle \selectfont}
\subsectionfont{\color{\subtitlecolor} \fontsizesubtitle \stylesubtitle \selectfont}
\subsubsectionfont{\color{\subsubtitlecolor} \fontsizesubsubtitle \stylesubsubtitle \selectfont}
\titleformat{\subsubsubsection}{\color{\ssstitlecolor} \normalfont \fontsizessstitle \stylessstitle}{\thesubsubsubsection}{1em}{}
\titlespacing*{\subsubsubsection}{0pt}{3.25ex plus 1ex minus .2ex}{1.5ex plus .2ex}

% -----------------------------------------------------------------------------
% Crea funciones para numerar objetos
% -----------------------------------------------------------------------------

% Numeración de la sección en los objetos CÓDIGO FUENTE
\ifthenelse{\equal{\showsectioncaptioncode}{none}}{
	\def\sectionobjectnumcode {}
}{
\ifthenelse{\equal{\showsectioncaptioncode}{sec}}{
	\def\sectionobjectnumcode {\thesection\sectioncaptiondelimiter}
}{
\ifthenelse{\equal{\showsectioncaptioncode}{ssec}}{
	\def\sectionobjectnumcode {\thesubsection\sectioncaptiondelimiter}
}{
\ifthenelse{\equal{\showsectioncaptioncode}{sssec}}{
	\def\sectionobjectnumcode {\thesubsubsection\sectioncaptiondelimiter}
}{
\ifthenelse{\equal{\showsectioncaptioncode}{ssssec}}{
	\def\sectionobjectnumcode {\thesubsubsubsection}
}{
	\throwbadconfig{Valor configuracion incorrecto}{\showsectioncaptioncode}{none,sec,ssec,sssec,ssssec}
}}}}}

% Numeración de la sección en los objetos ECUACIONES
\ifthenelse{\equal{\showsectioncaptioneqn}{none}}{
	\def\sectionobjectnumeqn {}
}{
\ifthenelse{\equal{\showsectioncaptioneqn}{sec}}{
	\def\sectionobjectnumeqn {\thesection\sectioncaptiondelimiter}
}{
\ifthenelse{\equal{\showsectioncaptioneqn}{ssec}}{
	\def\sectionobjectnumeqn {\thesubsection\sectioncaptiondelimiter}
}{
\ifthenelse{\equal{\showsectioncaptioneqn}{sssec}}{
	\def\sectionobjectnumeqn {\thesubsubsection\sectioncaptiondelimiter}
}{
\ifthenelse{\equal{\showsectioncaptioneqn}{ssssec}}{
	\def\sectionobjectnumeqn {\thesubsubsubsection}
}{
	\throwbadconfig{Valor configuracion incorrecto}{\showsectioncaptioneqn}{none,sec,ssec,sssec,ssssec}
}}}}}

% Numeración de la sección en los objetos FIGURAS
\ifthenelse{\equal{\showsectioncaptionfig}{none}}{
	\def\sectionobjectnumfig {}
}{
\ifthenelse{\equal{\showsectioncaptionfig}{sec}}{
	\def\sectionobjectnumfig {\thesection\sectioncaptiondelimiter}
}{
\ifthenelse{\equal{\showsectioncaptionfig}{ssec}}{
	\def\sectionobjectnumfig {\thesubsection\sectioncaptiondelimiter}
}{
\ifthenelse{\equal{\showsectioncaptionfig}{sssec}}{
	\def\sectionobjectnumfig {\thesubsubsection\sectioncaptiondelimiter}
}{
\ifthenelse{\equal{\showsectioncaptionfig}{ssssec}}{
	\def\sectionobjectnumfig {\thesubsubsubsection}
}{
	\throwbadconfig{Valor configuracion incorrecto}{\showsectioncaptionfig}{none,sec,ssec,sssec,ssssec}
}}}}}

% Numeración de la sección en los objetos TABLAS
\ifthenelse{\equal{\showsectioncaptiontab}{none}}{
	\def\sectionobjectnumtab {}
}{
\ifthenelse{\equal{\showsectioncaptiontab}{sec}}{
	\def\sectionobjectnumtab {\thesection\sectioncaptiondelimiter}
}{
\ifthenelse{\equal{\showsectioncaptiontab}{ssec}}{
	\def\sectionobjectnumtab {\thesubsection\sectioncaptiondelimiter}
}{
\ifthenelse{\equal{\showsectioncaptiontab}{sssec}}{
	\def\sectionobjectnumtab {\thesubsubsection\sectioncaptiondelimiter}
}{
\ifthenelse{\equal{\showsectioncaptiontab}{ssssec}}{
	\def\sectionobjectnumtab {\thesubsubsubsection}
}{
	\throwbadconfig{Valor configuracion incorrecto}{\showsectioncaptiontab}{none,sec,ssec,sssec,ssssec}
}}}}}

% -----------------------------------------------------------------------------
% Modifica numeración de objetos
% -----------------------------------------------------------------------------

% Código fuente, INCLUIR SECCIÓN
\ifthenelse{\equal{\captionnumcode}{arabic}}{
	\renewcommand{\thelstlisting}{\sectionobjectnumcode\arabic{lstlisting}}
}{
\ifthenelse{\equal{\captionnumcode}{alph}}{
	\renewcommand{\thelstlisting}{\sectionobjectnumcode\alph{lstlisting}}
}{
\ifthenelse{\equal{\captionnumcode}{Alph}}{
	\renewcommand{\thelstlisting}{\sectionobjectnumcode\Alph{lstlisting}}
}{
\ifthenelse{\equal{\captionnumcode}{roman}}{
	\renewcommand{\thelstlisting}{\sectionobjectnumcode\roman{lstlisting}}
}{
\ifthenelse{\equal{\captionnumcode}{Roman}}{
	\renewcommand{\thelstlisting}{\sectionobjectnumcode\Roman{lstlisting}}
}{
	\throwbadconfig{Tipo numero codigo fuente desconocido}{\captionnumcode}{arabic,alph,Alph,roman,Roman}}}}}
}

% Ecuaciones, INCLUIR SECCIÓN
\ifthenelse{\equal{\captionnumequation}{arabic}}{
	\renewcommand{\theequation}{\sectionobjectnumeqn\arabic{equation}}
}{
\ifthenelse{\equal{\captionnumequation}{alph}}{
	\renewcommand{\theequation}{\sectionobjectnumeqn\alph{equation}}
}{
\ifthenelse{\equal{\captionnumequation}{Alph}}{
	\renewcommand{\theequation}{\sectionobjectnumeqn\Alph{equation}}
}{
\ifthenelse{\equal{\captionnumequation}{roman}}{
	\renewcommand{\theequation}{\sectionobjectnumeqn\roman{equation}}
}{
\ifthenelse{\equal{\captionnumequation}{Roman}}{
	\renewcommand{\theequation}{\sectionobjectnumeqn\Roman{equation}}
}{
	\throwbadconfig{Tipo numero ecuacion desconocido}{\captionnumequation}{arabic,alph,Alph,roman,Roman}}}}}
}

% Figuras, INCLUIR SECCIÓN
\ifthenelse{\equal{\captionnumfigure}{arabic}}{
	\renewcommand{\thefigure}{\sectionobjectnumfig\arabic{figure}}
}{
\ifthenelse{\equal{\captionnumfigure}{alph}}{
	\renewcommand{\thefigure}{\sectionobjectnumfig\alph{figure}}
}{
\ifthenelse{\equal{\captionnumfigure}{Alph}}{
	\renewcommand{\thefigure}{\sectionobjectnumfig\Alph{figure}}
}{
\ifthenelse{\equal{\captionnumfigure}{roman}}{
	\renewcommand{\thefigure}{\sectionobjectnumfig\roman{figure}}
}{
\ifthenelse{\equal{\captionnumfigure}{Roman}}{
	\renewcommand{\thefigure}{\sectionobjectnumfig\Roman{figure}}
}{
	\throwbadconfig{Tipo numero figura desconocido}{\captionnumfigure}{arabic,alph,Alph,roman,Roman}}}}}
}

% Subfiguras, NO USAR SECCIONES YA QUE SON HIJAS DE FIGURA
\ifthenelse{\equal{\captionnumsubfigure}{arabic}}{
	\renewcommand{\thesubfigure}{\arabic{subfigure}}
}{
\ifthenelse{\equal{\captionnumsubfigure}{alph}}{
	\renewcommand{\thesubfigure}{\alph{subfigure}}
}{
\ifthenelse{\equal{\captionnumsubfigure}{Alph}}{
	\renewcommand{\thesubfigure}{\Alph{subfigure}}
}{
\ifthenelse{\equal{\captionnumsubfigure}{roman}}{
	\renewcommand{\thesubfigure}{\roman{subfigure}}
}{
\ifthenelse{\equal{\captionnumsubfigure}{Roman}}{
	\renewcommand{\thesubfigure}{\Roman{subfigure}}
}{
	\throwbadconfig{Tipo numero subfigura desconocido}{\captionnumsubfigure}{arabic,alph,Alph,roman,Roman}}}}}
}

% Tablas, INCLUIR SECCIÓN
\ifthenelse{\equal{\captionnumtable}{arabic}}{
	\renewcommand{\thetable}{\sectionobjectnumtab\arabic{table}}
}{
\ifthenelse{\equal{\captionnumtable}{alph}}{
	\renewcommand{\thetable}{\sectionobjectnumtab\alph{table}}
}{
\ifthenelse{\equal{\captionnumtable}{Alph}}{
	\renewcommand{\thetable}{\sectionobjectnumtab\Alph{table}}
}{
\ifthenelse{\equal{\captionnumtable}{roman}}{
	\renewcommand{\thetable}{\sectionobjectnumtab\roman{table}}
}{
\ifthenelse{\equal{\captionnumtable}{Roman}}{
	\renewcommand{\thetable}{\sectionobjectnumtab\Roman{table}}
}{
	\throwbadconfig{Tipo numero tabla desconocido}{\captionnumtable}{arabic,alph,Alph,roman,Roman}}}}}
}

% Subtablas, NO INCLUIR SECCIÓN YA QUE SON HIJAS DE LAS TABLAS
\ifthenelse{\equal{\captionnumsubtable}{arabic}}{
	\renewcommand{\thesubtable}{\arabic{subtable}}
}{
\ifthenelse{\equal{\captionnumsubtable}{alph}}{
	\renewcommand{\thesubtable}{\alph{subtable}}
}{
\ifthenelse{\equal{\captionnumsubtable}{Alph}}{
	\renewcommand{\thesubtable}{\Alph{subtable}}
}{
\ifthenelse{\equal{\captionnumsubtable}{roman}}{
	\renewcommand{\thesubtable}{\roman{subtable}}
}{
\ifthenelse{\equal{\captionnumsubtable}{Roman}}{
	\renewcommand{\thesubtable}{\Roman{subtable}}
}{
	\throwbadconfig{Tipo numero subtabla desconocido}{\captionnumsubtable}{arabic,alph,Alph,roman,Roman}}}}}
}

% -----------------------------------------------------------------------------
% Se reestablecen números de página y secciones
% -----------------------------------------------------------------------------

% Se usa número de páginas en arábigo si es que se tenía activado los númeos romanos
\ifthenelse{\equal{\predocpageromannumber}{true}}{
	\renewcommand{\thepage}{\arabic{page}}}{
}
\ifthenelse{\equal{\predocpageromannumber}{true}}{
	\setcounter{page}{1}}{
}
\setcounter{section}{0}
\setcounter{footnote}{0}

% -----------------------------------------------------------------------------
% Muestra los números de línea
% -----------------------------------------------------------------------------
\ifthenelse{\equal{\showlinenumbers}{true}}{
	\linenumbers}{
}

% END

% ======================= INICIO DEL DOCUMENTO =======================

% Template:     Informe/Reporte LaTeX
% Documento:    Archivo de ejemplo
% Versión:      5.5.7 (30/09/2018)
% Codificación: UTF-8
%
% Autor: Pablo Pizarro R. @ppizarror
%        Facultad de Ciencias Físicas y Matemáticas
%        Universidad de Chile
%        pablo.pizarro@ing.uchile.cl, ppizarror.com
%
% Manual template: [http://latex.ppizarror.com/Template-Informe/]
% Licencia MIT:    [https://opensource.org/licenses/MIT/]

% NUEVA SECCIÓN
% Las secciones se inician con \section, si se quiere una sección sin número se pueden usar las funciones \sectionanum (sección sin número) o la función \sectionanumnoi para crear el mismo título sin numerar y sin aparecer en el índice
\section{Informes con \LaTeX}

	% SUB-SECCIÓN
	% Las sub-secciones se inician con \subsection, si se quiere una sub-sección sin número se pueden usar las funciones \subsectionanum (nuevo subtítulo sin numeración) o la función \subsectionanumnoi para crear el mismo subtítulo sin numerar y sin aparecer en el índice
	\subsection{Una breve introducción}

		Este es un párrafo, puede contener múltiples \quotes{Expresiones} así como fórmulas o referencias \footnote{Las referencias se hacen utilizando la expresión \texttt{\textbackslash label}\{etiqueta\}.} a fórmulas como \eqref{eqn:identidad-imposible}. A continuación se muestra un ejemplo de inserción de imágenes o figuras (como la Figura \ref{img:testimage}) con el comando \href{http://latex.ppizarror.com/informe.html#hlp-imagen}{\texttt{\textbackslash insertimage}}:

		% Para insertar una imagen se puede usar la función \insertimage la cual toma un primer parámetro opcional para definir una etiqueta (dentro de los corchetes), luego toma la dirección de la imagen, sus parámetros (en este caso se definió la escala de 0.15) y una leyenda opcional
		\insertimage[\label{img:testimage}]{ejemplos/test-image.png}{scale=0.15}{Where are you? de \quotes{Internet}.}

		A continuación \footnote{Como se puede observar las funciones \texttt{\textbackslash insert...} añaden un párrafo automáticamente.} se muestra un ejemplo de inserción de ecuaciones simples con el comando \href{http://latex.ppizarror.com/informe.html#hlp-formulae}{\texttt{\textbackslash insertequation}}:

		% Se inserta una ecuación, el primer parámetro entre [] es opcional (permite identificar con una etiqueta para poder referenciarlo después con \ref), seguido de aquello se escribe la ecuación en modo bruto sin signos $
		\insertequation[\label{eqn:identidad-imposible}]{\pow{a}{k}=\pow{b}{k}+\pow{c}{k} \quad \forall k>2}

		% Se añade párrafo de prueba. Notar que no se requiere añadir un salto de línea después de insertar una ecuación
		\lipsum[75]

		% Los párrafos se pueden añadir con \newp, esta función se hizo para evitar errores y warnings por parte del compilador de LaTeX
		\newp Este es un nuevo párrafo insertado con el comando \href{http://latex.ppizarror.com/informe.html#hlp-parrafo}{\texttt{\textbackslash newp}}. Si no te gustan los comandos \texttt{\textbackslash newp}, \texttt{\textbackslash newpar} o \texttt{\textbackslash newparnl} simplemente puedes usar los salto de línea convencionales acompañado de \texttt{\textbackslash par}.

	% SUB-SECCIÓN
	\subsection{Añadiendo tablas}

		También puedes usar tablas, ¡Crearlas es muy fácil!. Puedes usar el plugin \href{https://www.ctan.org/tex-archive/support/excel2latex/}{Excel2Latex} \cite{ref2} de Excel para convertir las tablas a \LaTeX\xspace o bien utilizar el \quotes{creador de tablas online} \cite{ref3}.

		% Tabla generada con el plugin Excel2Latex
		\begin{table}[htbp]
			\centering
			\caption{Ejemplo de tablas.}
			\begin{tabular}{ccc}
				\hline
				\textbf{Columna 1} & \textbf{Columna 2} & \textbf{Columna 3} \bigstrut\\
				\hline
				$\omega$ & $\nu$ & $\delta$ \bigstrut[t]\\
				$\beta$ & $\gamma$ & $\epsilon$ \\
				$\Phi$ & $\Theta$ & $\varSigma$ \bigstrut[b]\\
				\hline
			\end{tabular}
			\label{tab:tabla-1}
		\end{table}


% NUEVA SECCIÓN
\newpage
\section{Aquí un nuevo tema}

	% SUB-SECCIÓN
	\subsection{Haciendo informes como un profesional}

		% Se inserta una imagen flotante en la izquierda del documento con \insertimageleft, al igual que las demás funciones, el primer parámetro es opcional, luego viene la ubicación de la imagen, seguido de la escala (un 30% del ancho de página) y por último su leyenda. Para insertar una imagen flotante en la derecha se utiliza \insertimageright usando los mismos parámetros
		\insertimageleft[\label{img:imagen-izquierda}]{ejemplos/test-image-wrap}{0.3}{Apolo flotando a la izquierda.}

		\lipsum[1]

		% Párrafos de ejemplo
		\newp \lipsum[115]
		\newp \lipsum[2]

		% Agrega una ecuación con leyenda
		\insertequationcaptioned[\label{eqn:formulasinsentido}]{\int_{a}^{b} f(x) \dd{x} = \fracnpartial{f(x)}{x}{\eta} \cdotp \textstyle \sum_{x=a}^{b} f(x)\cancelto{1+\frac{\epsilon}{k}}{(1+\Delta x)}}{Ecuación sin sentido.}

		% Aquí no es necesario usar \newp dado que todas las funciones \insert... añaden un párrafo nuevo por defecto
		\lipsum[115]

		\newp \lipsum[4]

	% Inserta un subtítulo sin número
	\subsection{Otros párrafos más normales}

		% Párrafos
		\lipsum[7]
		\newp \lipsum[2]

		% Se inserta una ecuación larga con el entorno gathered (1 solo número de ecuación)
		\insertgathered[\label{eqn:eqn-larga}]{
			\lpow{\Lambda}{f} = \frac{L\cdot f}{W} \cdot \frac{\pow{\lpow{Q}{e}}{2}}{8 \pow{\pi}{2} \pow{W}{4} g} + \sum_{i=1}^{l} \frac{f \cdot \big( M - d\big)}{l \cdot W} \cdot \frac{\pow{\big(\lpow{Q}{e}- i\cdot Q\big)}{2}}{8 \pow{\pi}{2} \pow{W}{4} g}\\
			Q_e = 2.5Q \cdot \int_{0}^{e} V(x) \dd{x} + \aasin{ \bigg(1+\frac{1}{1-e}\bigg) }
		}

		% Nuevo párrafo
		\lipsum[4]

		% Se inserta un multicols, con esto se pueden escribir párrafos en varias columnas
		\begin{multicols}{2}

			% Párrafo 1
			\lipsum[4]

			% Ecuación encerrada en una caja
			\insertequation[]{ \boxed{f(x) = \fracdpartial{u}{t}} }

			% Párrafo 2 del multicols
			\lipsum[1]

		\end{multicols}

	% SUB-SECCIÓN
	\subsection{Ejemplos de inserción de código fuente}

		% A continuación se crea una función auxiliar, esta es una herramienta extremadamente importante y muy útil. Esta función de ejemplo toma dos parámetros, uno es el lenguaje del código fuente, el segundo el identificador en el manual
		\newcommand{\insertsrcmanual}[2]{\href{http://latex.ppizarror.com/informe.html\#hlp-srccode\&srctype=#1}{#2}}

		El template permite la inserción de los siguientes lenguajes de programación de forma nativa: \insertsrcmanual{bash}{bash}, \insertsrcmanual{c}{C}, \insertsrcmanual{csharp}{C\#}, \insertsrcmanual{cpp}{C++}, \insertsrcmanual{docker}{DOCKER}, \insertsrcmanual{html5}{HTML5}, \insertsrcmanual{java}{Java}, \insertsrcmanual{js}{Javascript}, \insertsrcmanual{json}{JSON}, \insertsrcmanual{kotlin}{Kotlin}, \insertsrcmanual{latex}{LaTeX}, \insertsrcmanual{matlab}{Matlab}, \insertsrcmanual{perl}{Perl}, \insertsrcmanual{php}{PHP}, \insertsrcmanual{plaintext}{Texto plano}, \insertsrcmanual{pseudocode}{Pseudocódigo}, \insertsrcmanual{python}{Python}, \insertsrcmanual{ruby}{Ruby}, \insertsrcmanual{scala}{Scala}, \insertsrcmanual{sql}{SQL} y \insertsrcmanual{xml}{XML}. Para insertar un código fuente se debe usar el entorno \texttt{sourcecode}, o el entorno \texttt{sourcecodep} si es que se quiere utilizar parámetros adicionales. \newp

		A continuación se presenta un ejemplo de inserción de código fuente en Python (Código \ref{codigo-python}), Java (Código \ref{codigo-java}) y Matlab (Código \ref{codigo-matlab}):

% Se define el lenguaje del código. Cuidado: Los códigos en LaTeX son sensibles a las tabulaciones y espacios en blanco
\begin{sourcecode}[\label{codigo-python}]{python}{Ejemplo en Python.}
import numpy as np

def incmatrix(genl1, genl2):
	m = len(genl1)
	n = len(genl2)
	M = None # Comentario 1
	VT = np.zeros((n*m, 1), int) # Comentario 2
\end{sourcecode}

\begin{sourcecode}[\label{codigo-java}]{java}{Ejemplo en Java.}
import java.io.IOException;
import javax.servlet.*;

// Hola mundo
public class Hola extends GenericServlet {
	public void service(ServletRequest request, ServletResponse response)
	throws ServletException, IOException{
		response.setContentType("text/html");
		PrintWriter pw = response.getWriter();
		pw.println("Hola, mundo!");
		pw.close();
	}
}
\end{sourcecode}

\begin{sourcecode}[\label{codigo-matlab}]{matlab}{Ejemplo en Matlab.}
% Se crea gráfico
f = figure(1); hold on;
movegui(f, 'center');
xlabel('td/Tn'); ylabel('FAD=Umax/Uf0');
title('Espectro de pulso de desplazamiento');

for j = 1:length(BETA)
	fad = ones(1, NDATOS); % Arreglo para el FAD
	for i = 1:NDATOS
		[t, u_t, ~, ~] = main(BETA(j), r(i), M, K, F0, 0);
		fad(i) = max(abs(u_t)) / uf0;
	end
end
\end{sourcecode}

	% SUB-SECCIÓN
	\subsection{Añadir múltiples imágenes}

	A partir de la versión 5.0.0 existe el entorno \href{http://latex.ppizarror.com/informe.html#hlp-images}{\texttt{images}} que permite insertar múltiples imágenes de una manera muy sencilla, reemplazando así \footnote{Al usar cualquiera de estas funciones el template lanzará una alerta, se recomienda su desuso ya que en el futuro estas funciones podrían eliminarse del template.} a las funciones \texttt{\textbackslash insertdoubleimage}, \texttt{\textbackslash inserttripleimage}, etc. Para crear imágenes múltiples se deben usar instrucciones como:

\begin{sourcecode}{latex}{}
\begin{images}[\label{imagenmultiple}]{Ejemplo de imagen múltiple.}
	\addimage{ejemplos/test-image}{width=6.5cm}{Ciudad.}
	\addimage{ejemplos/test-image-wrap}{width=5cm}{Apolo.}
	\addimage{ejemplos/test-image}{width=12cm}{Ciudad más grande.}
\end{images}
\end{sourcecode}

	Obteniendo así:

	\begin{images}{Ejemplo de imagen múltiple.}
		\addimage{ejemplos/test-image}{width=6.5cm}{Ciudad.}
		\addimage{ejemplos/test-image-wrap}{width=5cm}{Apolo.}
		\addimage{ejemplos/test-image}{width=12cm}{Ciudad más grande.}
	\end{images}


% NUEVA SECCIÓN
% Inserta una sección sin número
\sectionanum{Más ejemplos}

	% Inserta un subtítulo sin número
	\subsectionanum{Listas y Enumeraciones}

		Hacer listas enumeradas con \LaTeX\ es muy fácil con el template \footnote{También puedes revisar el manual de las enumeraciones en \url{http://www.texnia.com/archive/enumitem.pdf}.}, para ello debes usar el comando \texttt{\textbackslash begin\{enumerate\}}, cada elemento comienza por \texttt{\textbackslash item}, resultando así:

		\begin{enumerate}
			\item Ítem 1
			\item Abracadabra
			\item Manzanas
		\end{enumerate}

		También se puede cambiar el tipo de enumeración, se pueden usar letras, números romanos, entre otros. Esto se logra cambiando el \textbf{label} del objeto \texttt{enumerate}. A continuación se muestra un ejemplo usando letras con el estilo \texttt{\textbackslash alph} \footnote{Con \texttt{\textbackslash Alph} las letras aparecen en mayúscula.}, números romanos con \texttt{\textbackslash roman} \footnote{Con \texttt{\textbackslash Roman} los números romanos salen en mayúscula.} o números griegos con \texttt{\textbackslash greek} \footnote{Una característica propia del template, con \texttt{\textbackslash Greek} las letras griegas están escritas en mayúscula.}:

		\begin{multicols}{3}
			\begin{enumeratebf}[label=\alph*) ] % Fuente en negrita
				\item Peras
				\item Manzanas
				\item Naranjas
			\end{enumeratebf}

			\begin{enumerate}[label=\greek*) ]
				\item Matemáticas
				\item Lenguaje
				\item Filosofía
			\end{enumerate}

			\begin{enumerate}[label=\roman*) ]
				\item Rojo
				\item Café
				\item Morado
			\end{enumerate}
		\end{multicols}

		Para hacer listas sin numerar con \LaTeX\ hay que usar el comando \texttt{\textbackslash begin\{itemize\}}, cada elemento empieza por \texttt{\textbackslash item}, resultando:

		\begin{multicols}{3}
			\begin{itemize}[label={--}]
				\item Peras
				\item Manzanas
				\item Naranjas
			\end{itemize}

			\begin{enumerate}[label={*}]
				\item Rojo
				\item Café
				\item Morado
			\end{enumerate}

			\begin{itemize}
				\item Árboles
				\item Pasto
				\item Flores
			\end{itemize}
		\end{multicols}

	% Inserta un subtítulo sin número
	\subsectionanum{Otros}

		Recuerda revisar el manual de todas las funciones y configuraciones de este template visitando el siguiente link: \url{http://latex.ppizarror.com/Template-Informe/}. Si necesitas una ayuda muy específica sobre el template, o si tienes alguna sugerencia, me puedes enviar un correo a \insertemail{pablo.pizarro@ing.uchile.cl}.


% REFERENCIAS (ESTILO BIBTEX), revisar configuración \stylecitereferences
\newpage % Salto de página
\begin{references}
	\bibitem{ref1}
	Template Informe en \LaTeX.
	\textit{¡Revisa el manual online de este template!} \\
	\url{http://latex.ppizarror.com/Template-Informe/}

	\bibitem{ref2}
	Excel2Latex.
	\textit{Importa de forma sencilla tus tablas de Excel a \LaTeX.} \\
	\url{https://www.ctan.org/tex-archive/support/excel2latex/}

	\bibitem{ref3}
	Overleaf.
	\textit{Uno de los mejores editores online para \LaTeX, renovado con su versión 2.0.} \\
	\href{https://es.overleaf.com}{\texttt{https://es.overleaf.com/}}
\end{references}


% ANEXO
\newpage
\begin{anexo}
	\section{Cálculos realizados}

		\subsection{Metodología}
			\lipsum[1]

			% Imagen, se numerará automáticamente con la letra del anexo según la configuración \appendixindepobjnum
			\insertimage[\label{img:anexo-2}]{ejemplos/test-image.png}{scale=0.2}{Imagen en anexo.}

		\subsection{Resultados}
			\lipsum[10]

			% Tablas
			\begin{table}[htbp]
				\centering
				\caption{Tabla de cálculo.}
				\begin{tabular}{ccc}
					\hline
					\textbf{Elemento} & $\epsilon_i$ & \boldmath{}\textbf{Valor}\unboldmath{} \bigstrut\\
					\hline
					A     & 10    & 3,14$\pi$ \bigstrut[t]\\
					B     & 20    & 6 \\
					C     & 30    & 7 \\
					\end{tabular}
				\label{tab:anexo-1}
			\end{table}

	\newpage
	\section{Más cálculos}

		% Párrafo
		\lipsum[1]

		% Nuevo párrafo con identación
		\newp \lipsum[4]

		% Tabla de encuestas
		\begin{table}[htbp]
			\centering
			\caption{Resultados encuesta.}
			\begin{tabular}{ccc}
				\hline
				\textbf{Herramienta} & \textbf{Nota} & \textbf{Recomendado} \bigstrut\\
				\hline
				\LaTeX & 100\% & Si $\checkmark$ \\
				Microsoft Word \textsuperscript{\textregistered} & 0\% & No $\frownie$\\
			\end{tabular}
			\label{tab:anexo-2}
		\end{table}

\end{anexo}
 % Ejemplo, se puede borrar

% FIN DEL DOCUMENTO
\end{document}
