% Configuración de referencias y citas
% -----------------------------------------------------------------------------
\ifthenelse{\equal{\stylecitereferences}{bibtex}}{
	\bibliographystyle{\bibtexstyle}
	\newlength{\bibitemsep}
	\setlength{\bibitemsep}{.2\baselineskip plus .05\baselineskip minus .05\baselineskip}
	\newlength{\bibparskip}\setlength{\bibparskip}{0pt}
	\let\oldthebibliography\thebibliography
	\renewcommand\thebibliography[1]{
		\oldthebibliography{#1}
		\setlength{\parskip}{\bibitemsep}
		\setlength{\itemsep}{\bibparskip}
	}
	\setlength{\bibitemsep}{\bibtexrefsep pt}
}{
\ifthenelse{\equal{\stylecitereferences}{custom}}{
	\coretemplatemessage{Usando estilo citas referencias custom, importar librerias y configuraciones posterior al llamado de template.tex en archivo principal}
}{
	\throwbadconfig{Estilo citas desconocido}{\stylecitereferences}{bibtex,custom}}
}

% Justificación de textos
% -----------------------------------------------------------------------------
\ifthenelse{\equal{\frametextjustified}{true}}{
	\apptocmd{\frame}{}{\justifying}{}
}{}
\newcommand{\justifytext}[1]{\parbox{\linewidth}{#1}}

% Word-break en citas
% -----------------------------------------------------------------------------
\makeatletter
\let\@cite@ofmt\@firstofone
\makeatother

% Configura footnotes
% -----------------------------------------------------------------------------
\let\oldfootnote\footnote
\renewcommand\footnote[1][]{\oldfootnote[frame,#1]}

% Corrige espaciamiento de itemize
% -----------------------------------------------------------------------------
\ifthenelse{\equal{\itemizedeleteleftmargin}{true}}{
	\setlist[itemize]{leftmargin=*}
	\setlist[enumerate]{leftmargin=*}
}{}

% Cambios generales en presentación
% -----------------------------------------------------------------------------
% \let\Tiny=\tiny % https://tex.stackexchange.com/q/58087/5764
% Desactiva color links en headline y footline
\makeatletter
	\renewcommand\insertshorttitle[1][]{%
		\beamer@setupshort{#1}%
		\let\thanks=\@gobble%
		\ifnum\c@page=1%
		\hyperlinkpresentationend{\beamer@insertshort{\usebeamercolor*[fg]{title in head/foot}\beamer@shorttitle}}%
		\else%
		\hyperlinkpresentationstart{\beamer@insertshort{\usebeamercolor*[fg]{title in head/foot}\beamer@shorttitle}}%
		\fi%
	}
	\newcommand\disablebeamercolorlinks{\def\HyColor@UseColor##1{}}
\makeatletter
\addtobeamertemplate{headline}{\disablebeamercolorlinks{}}{}
\addtobeamertemplate{footline}{\disablebeamercolorlinks{}}{}
% Modifica color de fondo de la página
\setbeamercolor{background canvas}{bg=\pagescolor}

% Configura los bloques
% -----------------------------------------------------------------------------
% Normal
\addtobeamertemplate{block begin}{}{
	\setlength\abovedisplayskip{0pt}%
	\vspace{\dimexpr-0.7em + \blockpaddingtop em} % Padding superior bloque
	\begin{adjustwidth}{%
		\dimexpr-0.4em + \blockpaddingleft em % Padding izquierdo bloque
	}{
		\dimexpr-0.4em + \blockpaddingright em % Padding derecho bloque
	}
}
\addtobeamertemplate{block end}{
	\end{adjustwidth}
	\vspace{\dimexpr-0.6em + \blockpaddingbottom em} % Padding inferior bloque
}{
	\vspace{\dimexpr-1.07em + \blockmarginbottom em} % Margen inferior bloque
}
% Alerta
\addtobeamertemplate{block alerted begin}{}{
	\setlength\abovedisplayskip{0pt}%
	\vspace{\dimexpr-0.7em + \blockpaddingtop em} % Padding superior alerta
	\begin{adjustwidth}{%
			\dimexpr-0.4em + \blockpaddingleft em % Padding izquierdo alerta
		}{
			\dimexpr-0.4em + \blockpaddingright em % Padding derecho alerta
		}
	}
\addtobeamertemplate{block alerted end}{
	\end{adjustwidth}
	\vspace{\dimexpr-0.6em + \blockpaddingbottom em} % Padding inferior alerta
}{
	\vspace{\dimexpr-1.07em + \blockmarginbottom em} % Margen inferior alerta
}
% Ejemplo
\addtobeamertemplate{block example begin}{}{
	\setlength\abovedisplayskip{0pt}%
	\vspace{\dimexpr-0.7em + \blockpaddingtop em} % Padding superior ejemplo
	\begin{adjustwidth}{%
			\dimexpr-0.4em + \blockpaddingleft em % Padding izquierdo ejemplo
		}{
			\dimexpr-0.4em + \blockpaddingright em % Padding derecho ejemplo
		}
	}
\addtobeamertemplate{block example end}{
	\end{adjustwidth}
	\vspace{\dimexpr-0.6em + \blockpaddingbottom em} % Padding inferior ejemplo
}{
	\vspace{\dimexpr-1.07em + \blockmarginbottom em} % Margen inferior ejemplo
}

% Definición de entornos beamer
% -----------------------------------------------------------------------------
\newenvironment<>{blockjustified}[1]{%
	\begin{block}#2{#1}\justifying
}{%
	\end{block}
}
\newenvironment<>{alertblockjustified}[1]{%
	\begin{alertblock}#2{#1}\justifying
}{%
	\end{alertblock}
}
\newenvironment<>{exampleblockjustified}[1]{%
	\begin{exampleblock}#2{#1}\justifying
}{%
	\end{exampleblock}
}

% Reinicia número de subfiguras y subtablas
% -----------------------------------------------------------------------------
\makeatletter
	\@addtoreset{subfigure}{framenumber}
	\@addtoreset{subtable}{framenumber}
\makeatother

% END