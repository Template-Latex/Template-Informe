% Template:     Template Presentación LaTeX
% Documento:    Funciones del template
% Versión:      DEV
% Codificación: UTF-8
%
% Autor: Pablo Pizarro R.
%        Facultad de Ciencias Físicas y Matemáticas
%        Universidad de Chile
%        pablo@ppizarror.com
%
% Manual template: [https://latex.ppizarror.com/presentacion]
% Licencia MIT:    [https://opensource.org/licenses/MIT]

\def\COREtabularframecolumnselected {1}
\def\COREtabularframebarheight {0.05}
\def\COREtabularframecontentheight {0.65}
\def\COREtabularframecontentwidthleft {0.4}
\def\COREtabularframecontentwidthright {0.6}
\def\COREtabularframecontentleftcentered {false}
\def\COREtabularframecontentrightcentered {false}
\def\COREtabularinit {false}
\newcounter{tabularcolnumber}
\setcounter{tabularcolnumber}{0}

% Función que configura el entorno tabular frames
%	#1	Altura de head en porcentaje (0-1)
%	#2	Altura de content en porcentaje (0-1)
%	#3	Ancho de la barra izquierda de contenidos (coloreada)
%	#4	Ancho de la barra derecha de contenidos
\newcommand{\tabularframecfg}[4]{
	\emptyvarerr{\tabularframecfg}{#1}{Altura de head no definido}
	\emptyvarerr{\tabularframecfg}{#2}{Altura de content no definido}
	\emptyvarerr{\tabularframecfg}{#3}{Ancho de la barra izquierda de contenidos}
	\emptyvarerr{\tabularframecfg}{#4}{Ancho de la barra derecha de contenidos}
	\def\COREtabularframebarheight {#1}
	\def\COREtabularframecontentheight {#2}
	\def\COREtabularframecontentwidthleft {#3}
	\def\COREtabularframecontentwidthright {#4}
}

% Función que configura la alineación del entorno tabular frames
%	#1	Columna izquierda contenidos centrada (true/false)
%	#2	Columna derecha contenidos centrada (true/false)
\newcommand{\tabularframecfgalign}[2]{
	\emptyvarerr{\tabularframecfgalign}{#1}{Centrado columna izquierda no definido}
	\emptyvarerr{\tabularframecfgalign}{#2}{Centrado columna derecha no definido}
	\corecheckbooleanvar{#1}
	\corecheckbooleanvar{#2}
	\def\COREtabularframecontentleftcentered {#1}
	\def\COREtabularframecontentrightcentered {#2}
}

% Función que configura el entorno tabular frames
%	#1	Número de la columna destacada (desde 1)
\newenvironment{tabularframehead}[1]{%
	\emptyvarerr{\tabularframehead}{#1}{Numero de columna destacada no definido}
	\def\COREtabularframecolumnselected {#1}
	\def\COREtabularinit {true}
	\setcounter{tabularcolnumber}{0}%
	\vspace{\dimexpr-\COREtabularframebarheight\textheight-0.5\footheight}
	\begin{columns}[t,onlytextwidth]%
	}{
	\end{columns}
	\setcounter{tabularcolnumber}{-1}
}

% Agrega una columna al tabular head
%	#1	Tamaño de la columna
%	#2	Texto
\newcommand{\addtfheadcolumn}[2]{%
	\ifthenelse{\equal{\COREtabularinit}{true}}{}{%
		\throwwarning{Funcion \noexpand\addtfheadcolumn no puede usarse fuera del entorno \noexpand\tabularframehead}\stop}%
	\emptyvarerr{\addtfheadcolumn}{#1}{Tamano de la columna no definido}%
	\emptyvarerr{\addtfheadcolumn}{#2}{Texto de la columna no definido}%
	\stepcounter{tabularcolnumber}%
	% Obtiene el tamaño de la columna
	\column{\dimexpr#1\linewidth}%
	\ifthenelse{\equal{\thetabularcolnumber}{\COREtabularframecolumnselected}}{%
		\transparent{1.0}%
	}{%
		\transparent{\opacitytabularframe}%
	}
	\usebeamercolor[fg]{frametitle}\colorbox{bg}{\begin{minipage}[t][\COREtabularframebarheight\textheight][t]{\dimexpr\linewidth-15\fboxrule}%
		\usebeamercolor[fg]{frametitle}{%
			\verticallycentertext{#2}%
		}%
	\end{minipage}}%
}

% Inicia el contenido de un tabular frame
%	#1	Panel izquierdo
%	#2	Panel derecho
\newcommand{\tabularframecontent}[2]{%
	\ifthenelse{\equal{\thetabularcolnumber}{-1}}{}{%
		\throwwarning{Funcion \noexpand\addtfheadcolumn no puede usarse fuera del entorno \noexpand\tabularframehead}\stop}%
	\emptyvarerr{\tabularframecontent}{#1}{Columna izquierda no definida}%
	\emptyvarerr{\tabularframecontent}{#2}{Columna derecha no definida}%
	\vspace{-2.5\fboxrule}
	\begin{columns}[t,onlytextwidth]%
		\column{\COREtabularframecontentwidthleft\linewidth}%
		\usebeamercolor[fg]{block body}\colorbox{bg}{\begin{minipage}[t][\COREtabularframecontentheight \textheight][t]{\dimexpr\textwidth-15\fboxrule}%
			\ifthenelse{\equal{\COREtabularframecontentleftcentered}{false}}{%
				#1%
			}{%
				\verticallycentertext{#1}%
			}
		\end{minipage}}%
		\column{\COREtabularframecontentwidthright\linewidth}%
		\centering
		{\begin{minipage}[t][\COREtabularframecontentheight \textheight][t]{\dimexpr\textwidth-10\fboxrule}%
			\ifthenelse{\equal{\COREtabularframecontentrightcentered}{false}}{%
				\vspace{-10\fboxrule}%
				#2%
			}{%
				\vspace{-\baselineskip}
				\verticallycentertext{#2}%
			}
		\end{minipage}}%
	\end{columns}%
	\def\COREtabularinit {false}%
}