% Template:     Informe/Reporte LaTeX
% Documento:    Configuraciones del template
% Versión:      4.3.6 (30/07/2017)
% Codificación: UTF-8
%
% Autor: Pablo Pizarro R.
%        Facultad de Ciencias Físicas y Matemáticas
%        Universidad de Chile
%        pablo.pizarro@ing.uchile.cl, ppizarror.com
%
% Manual template: [http://latex.ppizarror.com/Template-Informe/]
% Licencia MIT:    [https://opensource.org/licenses/MIT/]

% CONFIGURACIONES GENERALES
\def\addemptypagetwosides {false}  % Añade pags. en blanco al imprimir a 2 caras
\def\anumsecaddtocounter {false}   % Función para insertar títulos anum. aumenta contador
\def\defaultinterline {1.0}        % Interlineado por defecto [pt]
\def\defaultnewlinesize {11.0}     % Tamaño del salto de línea [pt]
\def\fontdocument {lmodern}        % Tipografía (lmodern,arial,helvet)
\def\numberedequation {true}       % Ecuaciones con \insert... numeradas
\def\pointdecimal {false}          % Decimales con punto en vez de coma
\def\romanpageuppercase {false}    % Páginas en número romano en mayúsculas
\def\showdotontitles {true}        % Punto al final de cada número de título/sub(sub)título
\def\tablepadding {1.0}            % Ancho de celda de las tablas

% ESTILO PORTADA Y HEADER-FOOTER
\def\hfstyle {style1}              % Estilo del header-footer (6 estilos distintos)
\def\portraitstyle {style1}        % Estilo de la portada (4 estilos distintos)

% MÁRGENES DE PÁGINA
\def\firstpagemargintop {3.8}      % Margen superior página portada [cm]
\def\pagemarginbottom {2.7}        % Margen inferior página [cm]
\def\pagemarginleft {2.54}         % Margen izquierdo página [cm]
\def\pagemarginright {2.54}        % Margen derecho página [cm]
\def\pagemargintop {3.0}           % Margen superior página [cm]

% CONFIGURACIÓN DE LAS LEYENDAS - CAPTION
\def\captionlessmarginimage {0.1}  % Margen sup/inf de fig si no hay leyenda [cm]
\def\captionlrmargin {2.0}         % Márgenes izq/der de la leyenda [cm]
\def\captionalignment {justified}  % Alineación leyenda: justified,centered,left,right
\def\captiontbmarginfigure {9.35}  % Margen sup/inf de la leyenda en figuras [pt]
\def\captiontbmargintable {7.0}    % Margen sup/inf de la leyenda en tablas [pt]
\def\captiontextbold {false}       % Etiquetas (Código,Figura,Tabla) en negrita
\def\codecaptiontop {true}         % Leyenda arriba del código fuente
\def\figurecaptiontop {false}      % Leyenda arriba de las imágenes
\def\showsectiononcaption {false}  % Muestra el número de sección en las leyendas
\def\tablecaptiontop {true}        % Leyenda arriba de las tablas

% CONFIGURACIÓN DEL ÍNDICE
\def\indexdepth {3}                % Profundidad máxima del índice
\def\indextitlemargin {7.0}        % Margen título en índice \insertindextitle [pt]
\def\showdotonobjectindex {false}  % Punto en cada número de figura, tabla o código
\def\showdotpagenumindex {true}    % Muestra puntos entre objeto y número de página
\def\showindex {true}              % Muestra el índice
\def\showindexofcode {true}        % Muestra la lista de códigos fuente
\def\showindexofcontents {true}    % Muestra la lista de contenidos
\def\showindexoffigures {true}     % Muestra la lista de figuras
\def\showindexoftables {true}      % Muestra la lista de tablas

% CONFIGURACIÓN DE LOS COLORES DEL DOCUMENTO
\def\captioncolor {black}          % Color de la etiqueta (Código,Figura,Tabla)
\def\captiontextcolor {black}      % Color de la leyenda
\def\citecolor {black}             % Color del número de las referencias o citas
\def\highlightcolor {yellow}       % Color del subrayado con \hl
\def\indextitlecolor {black}       % Color de los títulos del índice
\def\linkcolor {black}             % Color de los links del doc.
\def\maintextcolor {black}         % Color principal del texto
\def\portraittitlecolor {black}    % Color de los títulos de la portada
\def\showborderonlinks {false}     % Color de un links por un recuadro de color
\def\subsubtitlecolor {black}      % Color de los sub-subtítulos
\def\subtitlecolor {black}         % Color de los subtítulos
\def\tablelinecolor {black}        % Color de las líneas de las tablas
\def\titlecolor {black}            % Color de los títulos
\def\urlcolor {magenta}            % Color de los enlaces web (\url,\href)

% CONFIGURACIÓN DE FIGURAS
\def\defaultimagefolder {images/}  % Carpeta raíz de las imágenes
\def\marginbottomimages {-0.2}     % Margen inferior figura [cm]
\def\marginfloatimages {-13.0}     % Margen sup. fig. float \insertimageleft/right [pt]
\def\margintopimages {0.0}         % Margen superior figura [cm]

% ANEXO, CITAS, REFERENCIAS
\def\apaciterefsep {9}             % Separación entre referencias [pt] {apacite}
\def\appendixindepobjnum {true}    % Anexo usa número objetos de forma independiente
\def\bibtexrefsep {9}              % Separación entre referencias [pt] {bibtex}
\def\natbibrefsep {5}              % Separación entre referencias [pt] {natbib}
\def\referencenumsection {false}   % Sección de referencias numerada
\def\sectionappendixlastchar {.}   % Caracter entre el n° de la sec. de anexo y el título
\def\showappendixsecindex {false}  % Muestra el título de la sec. de anexos en el índice
\def\showappendixsectitle {false}  % Muestra el título de la sec. de anexo en el informe
\def\stylecitereferences {bibtex}  % Estilo citas y referencias (bibtex,apacite,natbib)
\def\twocolumnreferences {false}   % Referencias en dos columnas

% ESTILO Y TAMAÑO DE TÍTULOS
\def\fontsizesubsubtitle {\large}  % Tamaño sub-subtítulos
\def\fontsizesubtitle {\Large}     % Tamaño subtítulos
\def\fontsizetitle {\huge}         % Tamaño títulos
\def\fontsizetitlei {\huge}        % Tamaño títulos en el índice
\def\stylesubsubtitle {\bfseries}  % Estilo sub-subtítulos
\def\stylesubtitle {\bfseries}     % Estilo subtítulos
\def\styletitle {\bfseries}        % Estilo títulos
\def\styletitlei {\bfseries}       % Estilo títulos en el índice

% NOMBRE DE OBJETOS
\def\nameabstract {Resumen}           % Nombre del resumen-abstract
\def\nameappendixsection{Anexos}      % Nombre de la sección de anexos/apéndices
\def\nameportraitpage {Portada}       % Etiqueta página de la portada
\def\namereferences {Referencias}     % Nombre de la sección de referencias
\def\nomltappendixsection {Anexo}     % Etiqueta sección en anexo/apéndices
\def\nomltcont {Índice de Contenidos} % Nombre del índice de contenidos
\def\nomltfigure {Lista de Figuras}   % Nombre del índice de la lista de figuras
\def\nomltsrc {Lista de Códigos}      % Nombre del índice de la lista de código
\def\nomlttable {Lista de Tablas}     % Nombre del índice de la lista de tablas
\def\nomltwfigure {Figura}            % Etiqueta leyenda de las figuras
\def\nomltwsrc {Código}               % Etiqueta leyenda del código fuente
\def\nomltwtable {Tabla}              % Etiqueta leyenda de las tablas

% OPCIONES DEL PDF COMPILADO
\def\cfgbookmarksopenlevel {2}     % Nivel de los marcadores a mostrar (2: subsecciones)
\def\cfgpdfbookmarkopen {true}     % Expande los marcadores hasta el nivel configurado
\def\cfgpdfcenterwindow {true}     % Centra la ventana del lector al abrir el pdf
\def\cfgpdfcopyright {}            % Establece el copyright del documento
\def\cfgpdfdisplaydoctitle {true}  % Muestra el título del informe como título del pdf
\def\cfgpdffitwindow {false}       % Ajusta la ventana del lector al tamaño del pdf
\def\cfgpdfmenubar {true}          % Muestra el menú del lector al abrir el pdf
\def\cfgpdftoolbar {true}          % Muestra la barra de herramientas del lector pdf
\def\cfgshowbookmarkmenu {true}    % Muestra el menú de los marcadores en el lector pdf
