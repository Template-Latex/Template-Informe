% Template:     Informe/Reporte LaTeX
% Documento:    Configuración de página
% Versión:      4.7.4 (04/04/2018)
% Codificación: UTF-8
%
% Autor: Pablo Pizarro R.
%        Facultad de Ciencias Físicas y Matemáticas
%        Universidad de Chile
%        pablo.pizarro@ing.uchile.cl, ppizarror.com
%
% Manual template: [http://latex.ppizarror.com/Template-Informe/]
% Licencia MIT:    [https://opensource.org/licenses/MIT/]

% Numeración de páginas
\newpage
\ifthenelse{\equal{\romanpageuppercase}{true}}{
	\pagenumbering{Roman}
}{
	\pagenumbering{roman}
}
\setcounter{page}{1}
\setcounter{footnote}{1}

% Márgenes de páginas y tablas
\setpagemargincm{\pagemarginleft}{\pagemargintop}{\pagemarginright}{\pagemarginbottom}
\def\arraystretch{\tablepadding} % Se ajusta el padding de las tablas

% Se define el punto decimal
\ifthenelse{\equal{\pointdecimal}{true}}{
	\decimalpoint}{
}

% Definición de nombres de objetos
\renewcommand{\appendixname}{\nomltappendixsection} % Nombre del anexo (título)
\renewcommand{\appendixpagename}{\nameappendixsection} % Nombre del anexo en índice
\renewcommand{\appendixtocname}{\nameappendixsection} % Nombre del anexo en índice
\renewcommand{\contentsname}{\nomltcont}  % Nombre del índice
\renewcommand{\figurename}{\nomltwfigure} % Nombre de la leyenda de las fig.
\renewcommand{\listfigurename}{\nomltfigure} % Nombre del índice de figuras
\renewcommand{\listtablename}{\nomlttable} % Nombre del índice de tablas
\renewcommand{\lstlistingname}{\nomltwsrc} % Nombre leyenda del código fuente
\renewcommand{\lstlistlistingname}{\nomltsrc} % Nombre índice código fuente
\renewcommand{\refname}{\namereferences} % Nombre de las referencias
\renewcommand{\tablename}{\nomltwtable} % Nombre de la leyenda de tablas

% Estilo de títulos
\sectionfont{\color{\titlecolor} \fontsizetitle \styletitle \selectfont}
\subsectionfont{\color{\subtitlecolor} \fontsizesubtitle \stylesubtitle \selectfont}
\subsubsectionfont{\color{\subsubtitlecolor} \fontsizesubsubtitle \stylesubsubtitle \selectfont}

% Se crean los header-footer
\ifthenelse{\equal{\hfstyle}{style1}}{
	\pagestyle{fancy} \fancyhf{}
	\fancyhead[L]{\nouppercase{\rightmark}}
	\fancyhead[R]{\small \rm \thepage}
	\fancyfoot[L]{\small \rm \textit{\titulodelinforme}}
	\fancyfoot[R]{\small \rm \textit{\codigodelcurso \nombredelcurso}}
	\renewcommand{\headrulewidth}{0.5pt}
	\renewcommand{\footrulewidth}{0.5pt}
	\renewcommand{\sectionmark}[1]{\markboth{#1}{}}
}{
\ifthenelse{\equal{\hfstyle}{style2}}{
	\pagestyle{fancy} \fancyhf{}
	\fancyhead[L]{\nouppercase{\rightmark}}
	\fancyhead[R]{\small \rm \thepage}
	\fancyfoot[L]{\small \rm \textit{\titulodelinforme}}
	\fancyfoot[R]{\small \rm \textit{\codigodelcurso \nombredelcurso}}
	\renewcommand{\headrulewidth}{0.5pt}
	\renewcommand{\footrulewidth}{0pt}
	\renewcommand{\sectionmark}[1]{\markboth{#1}{}}
}{
\ifthenelse{\equal{\hfstyle}{style3}}{
	\pagestyle{fancy} \fancyhf{}
	\fancyhead[L]{
		\small \rm \textit{\codigodelcurso \nombredelcurso}
		\vspace{0.04cm}
	}
	\fancyhead[R]{
		\includegraphics[width=1.2cm]{departamentos/fcfm2}
		\vspace{-0.10cm}
	}
	\fancyfoot[C]{\thepage}
	\renewcommand{\headrulewidth}{0.5pt}
	\renewcommand{\footrulewidth}{0pt}
}{
\ifthenelse{\equal{\hfstyle}{style4}}{
	\pagestyle{fancy} \fancyhf{}
	\fancyhead[L]{\nouppercase{\rightmark}}
	\fancyhead[R]{}
	\fancyfoot[C]{\small \rm \thepage}
	\renewcommand{\headrulewidth}{0.5pt}
	\renewcommand{\footrulewidth}{0pt}
	\renewcommand{\sectionmark}[1]{\markboth{#1}{}}
}{
\ifthenelse{\equal{\hfstyle}{style5}}{
	\pagestyle{fancy} \fancyhf{}
	\fancyhead[L]{\codigodelcurso \nombredelcurso}
	\fancyhead[R]{\nouppercase{\rightmark}}
	\fancyfoot[L]{\departamentouniversidad, \nombreuniversidad}
	\fancyfoot[R]{\small \rm \thepage}
	\renewcommand{\headrulewidth}{0pt}
	\renewcommand{\footrulewidth}{0pt}
	\renewcommand{\sectionmark}[1]{\markboth{#1}{}}
}{
\ifthenelse{\equal{\hfstyle}{style6}}{
	\pagestyle{fancy} \fancyhf{}
	\fancyfoot[L]{\departamentouniversidad}
	\fancyfoot[C]{\thepage}
	\fancyfoot[R]{\nombreuniversidad}
	\renewcommand{\headrulewidth}{0pt}
	\renewcommand{\footrulewidth}{0pt}
	\setlength{\headheight}{49pt}
}{
\ifthenelse{\equal{\hfstyle}{style7}}{
	\pagestyle{fancy} \fancyhf{}
	\fancyfoot[C]{\thepage}
	\renewcommand{\headrulewidth}{0pt}
	\renewcommand{\footrulewidth}{0pt}
	\setlength{\headheight}{49pt}
}{
\ifthenelse{\equal{\hfstyle}{style8}}{
	\pagestyle{fancy} \fancyhf{}
	\fancyfoot[R]{\thepage}
	\renewcommand{\headrulewidth}{0pt}
	\renewcommand{\footrulewidth}{0pt}
	\setlength{\headheight}{49pt}
}{
	\throwbadconfigondoc{Estilo de header-footer incorrecto}{\hfstyle}{style1 .. style8}}}}}}}}
}

% Muestra los números de línea
\ifthenelse{\equal{\showlinenumbers}{true}}{
	\linenumbers}{
}
