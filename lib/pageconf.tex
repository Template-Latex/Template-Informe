% Template:     Informe/Reporte LaTeX
% Documento:    Configuración de página
% Versión:      4.3.6 (30/07/2017)
% Codificación: UTF-8
%
% Autor: Pablo Pizarro R.
%        Facultad de Ciencias Físicas y Matemáticas
%        Universidad de Chile
%        pablo.pizarro@ing.uchile.cl, ppizarror.com
%
% Manual template: [http://latex.ppizarror.com/Template-Informe/]
% Licencia MIT:    [https://opensource.org/licenses/MIT/]

% Numeración de páginas
\newpage
\ifthenelse{\equal{\romanpageuppercase}{true}}{
	\pagenumbering{Roman}
}{
	\pagenumbering{roman}
}
\setcounter{page}{1}
\setcounter{footnote}{1}

% Márgenes de páginas y tablas
\setpagemargincm{\pagemarginleft}{\pagemargintop}{\pagemarginright}{\pagemarginbottom}
\def\arraystretch{\tablepadding} % Se ajusta el padding de las tablas

% Se define el punto decimal
\ifthenelse{\equal{\pointdecimal}{true}}{
	\decimalpoint}{
}

% Definición de nombres de objetos
\renewcommand{\appendixname}{\nomltappendixsection} % Nom. del anexo en etiq. de título
\renewcommand{\appendixpagename}{\nameappendixsection} % Nombre del anexo en índice
\renewcommand{\appendixtocname}{\nameappendixsection} % Nombre del anexo en índice
\renewcommand{\contentsname}{\nomltcont}  % Nombre del índice
\renewcommand{\figurename}{\nomltwfigure} % Nombre de la leyenda de las fig.
\renewcommand{\listfigurename}{\nomltfigure} % Nombre del índice de figuras
\renewcommand{\listtablename}{\nomlttable} % Nombre del índice de tablas
\renewcommand{\lstlistingname}{\nomltwsrc} % Nombre leyenda del código fuente
\renewcommand{\lstlistlistingname}{\nomltsrc} % Nombre índice código fuente
\renewcommand{\refname}{\namereferences} % Nombre de las referencias
\renewcommand{\tablename}{\nomltwtable} % Nombre de la leyenda de tablas

% Numeración de objetos
\ifthenelse{\equal{\showsectiononcaption}{true}}{
	\counterwithin{equation}{section}   % Añade número de sección a las ecuaciones
	\counterwithin{figure}{section}     % Añade número de sección a las figuras
	\counterwithin{lstlisting}{section} % Añade número de sección a los códigos
	\counterwithin{table}{section}      % Añade número de sección a las tablas
}{}

% Se crean los header-footer
\ifthenelse{\equal{\hfstyle}{style1}}{
	\pagestyle{fancy} \fancyhf{}
	\fancyhead[L]{\nouppercase{\rightmark}}
	\fancyhead[R]{\small \rm \thepage}
	\fancyfoot[L]{\small \rm \textit{\titulodelinforme}}
	\fancyfoot[R]{\small \rm \textit{\codigodelcurso \nombredelcurso}}
	\renewcommand{\footrulewidth}{0.5pt}
	\renewcommand{\headrulewidth}{0.5pt}
	\renewcommand{\sectionmark}[1]{\markboth{#1}{}}
}{
\ifthenelse{\equal{\hfstyle}{style2}}{
	\pagestyle{fancy} \fancyhf{}
	\fancyfoot[C]{\thepage}
	\renewcommand{\headrulewidth}{0pt}
	\renewcommand{\footrulewidth}{0pt}
	\setlength{\headheight}{49pt}
}{
\ifthenelse{\equal{\hfstyle}{style3}}{
	\pagestyle{fancy} \fancyhf{}
	\fancyfoot[L]{\departamentouniversidad}
	\fancyfoot[C]{\thepage}
	\fancyfoot[R]{\nombreuniversidad}
	\renewcommand{\headrulewidth}{0pt}
	\renewcommand{\footrulewidth}{0pt}
	\setlength{\headheight}{49pt}
}{
\ifthenelse{\equal{\hfstyle}{style4}}{
	\pagestyle{fancy} \fancyhf{}
	\fancyfoot[R]{\thepage}
	\renewcommand{\headrulewidth}{0pt}
	\renewcommand{\footrulewidth}{0pt}
	\setlength{\headheight}{49pt}
}{
\ifthenelse{\equal{\hfstyle}{style5}}{
	\pagestyle{fancy} \fancyhf{}
	\fancyhead[L]{\codigodelcurso \nombredelcurso}
	\fancyhead[R]{\nouppercase{\rightmark}}
	\fancyfoot[L]{\departamentouniversidad, \nombreuniversidad}
	\fancyfoot[R]{\small \rm \thepage}
	\renewcommand{\footrulewidth}{0pt}
	\renewcommand{\headrulewidth}{0pt}
	\renewcommand{\sectionmark}[1]{\markboth{#1}{}}
}{
\ifthenelse{\equal{\hfstyle}{style6}}{
	\pagestyle{fancy} \fancyhf{}
	\fancyhead[L]{\nouppercase{\rightmark}}
	\fancyhead[R]{}
	\fancyfoot[C]{\small \rm \thepage}
	\renewcommand{\footrulewidth}{0pt}
	\renewcommand{\headrulewidth}{0.5pt}
	\renewcommand{\sectionmark}[1]{\markboth{#1}{}}
}{
	\throwbadconfigondoc{Estilo de header-footer incorrecto}{\hfstyle}{style1,style2,style3,style4,style5,style6}}}}}}
}

% Profundidad del índice
\setcounter{tocdepth}{\indexdepth} % Se ajusta la profundidad del índice
