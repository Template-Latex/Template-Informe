\begin{sourcecode}{cobol}{Ejemplo en Cobol.}
*> Terminator period ("implicit termination")
IF invalid-record
    IF no-more-records
        NEXT SENTENCE
    ELSE
        READ record-file
            AT END SET no-more-records TO TRUE.

*> Scope terminators ("explicit termination")
IF invalid-record
    IF no-more-records
        CONTINUE
    ELSE
        READ record-file
            AT END SET no-more-records TO TRUE
        END-READ
    END-IF
END-IF
\end{sourcecode}

\begin{sourcecode}{basic}{Ejemplo en Basic.}
REM QuickBASIC example

REM Forward declaration - allows the main code to call a
REM    subroutine that is defined later in the source code
DECLARE SUB PrintSomeStars (StarCount!)

REM Main program follows
INPUT "What is your name: ", UserName$
PRINT "Hello "; UserName$
DO
   INPUT "How many stars do you want: ", NumStars
   CALL PrintSomeStars(NumStars)
   DO
      INPUT "Do you want more stars? ", Answer$
   LOOP UNTIL Answer$ <> ""
   Answer$ = LEFT$(Answer$, 1)
LOOP WHILE UCASE$(Answer$) = "Y"
PRINT "Goodbye "; UserName$
END

REM subroutine definition
SUB PrintSomeStars (StarCount)
   REM This procedure uses a local variable called Stars$
   Stars$ = STRING$(StarCount, "*")
   PRINT Stars$
END SUB
\end{sourcecode}

\begin{sourcecode}{postscript}{Ejemplo en PostScript.}
%!PS
 /Courier             % name the desired font
 20 selectfont        % choose the size in points and establish 
                      % the font as the current one
 72 500 moveto        % position the current point at 
                      % coordinates 72, 500 (the origin is at the 
                      % lower-left corner of the page)
 (Hello world!) show  % stroke the text in parentheses
 showpage             % print all on the page
\end{sourcecode}

\begin{sourcecode}{gnuplot}{Ejemplo en Gnuplot.}
set terminal epslatex   
set output 'mplot.tex'   
set xlabel "Avg. No. of demand duration (slot) "   
set ylabel "Acceptence rate (%)"   
set grid xtics ytics   
set key right bottom   

set style line 1 lw 1 lc 3 pt 7  
set style line 2 lw 1 lc 1 pt 5  
set style line 3 lw 1 lc 0 pt 9  
set style line 4 lw 1 lc 4 pt 3  

plot "AcceptanceRate_Ser.txt" using 2:4:5:6 title "NoMig" with errorlines linestyle 1,\ 
     "AcceptanceRate_Ser.txt" using 2:7:8:9 title "FlowMig" with errorlines linestyle 2 ,\ 
      "AcceptanceRate_Ser.txt" using 2:10:11:12 title "VMMig" with errorlines linestyle 3, \ 
     "AcceptanceRate_Ser.txt" using 2:13:14:15 title "NoRis" with errorlines linestyle 4
\end{sourcecode}

\begin{sourcecode}{go}{Ejemplo en Go.}
// Package main is, tautologically, the main package.
package main

import "fmt"

type s struct{}

func main() {
	fmt.Println("Hello, world!")
}
\end{sourcecode}

\begin{sourcecode}{swift}{Ejemplo en Swift.}
// This function passes the result of the first closure or function to another and returns its result.
func b(closure a: () -> Int, anotherClosure: (Int) -> Int) -> Int {
    return anotherClosure(a())
}

// With no trailing closures
a(closure: {return 1}, anotherClosure: {x in return x + 1})

// With 1 trailing closure
a(closure: {return 1}) {x in return x + 1})

// With 2 trailing closures
a {return 1} anotherClosure: {x in return x + 1}
\end{sourcecode}

\begin{sourcecode}{mathematica}{Ejemplo en Mathematica.}
Block[
 {region=DiscretizeRegion[Polygon[{{0,0},{-1/2,Sqrt[3]/2},{1/2,Sqrt[3]/2}}]]},
 ContourPlot[
  2 Cos[4 Pi x] Sin[(4 Pi y)/Sqrt[3]] - Sin[(8 Pi y)/Sqrt[3]],
  {x,y} %*$\in$*) region,
  PlotPoints ->70,
  Contours ->10,
  AspectRatio ->Automatic,
  FrameLabel ->{"x","y"},
  PlotLabel ->"Excited state of the equilateral triangle"
 ]
]
\end{sourcecode}

\begin{sourcecode}{racket}{Ejemplo en Racket.}
#lang racket/gui

;; let's play a guessing game

(define frame (new frame% [label "Guess"]))

(define secret (random 5))
(define ((check i) btn evt)
  (define found? (if (= i secret) "Yes" "No"))
  (message-box "?" found?)
  (when (= i secret)
    (send frame show #false)))

(for ([i (in-range 5)])
   (new button%
        [label (~a i)]
        [parent frame]
        [callback (check i)]))

(send frame show #t)
\end{sourcecode}

Testeamos las ecuaciones:

\insertequation[\label{CI2.1}]{Epic}
\insertequation[\label{D.1}]{Epic}
\insertequation[\label{T.1}]{Epic}
\insertequation[\label{T.2}]{Epic}

\begin{sourcecode}{pseudocode}{Algoritmo transporte de calor en sistemas geotermales.}
Inicializar temperatura $T^0$ y velocidad $\textbf{u}^0$ según ecuaciones $\coderef{CI2.1}$.
PARA k = 1, ..., n
	Resolver la ecuación de Darcy (ec. $\codeeqref{D.1}$) con Backwards Euler para obtener $\Psi^{pr}$
	Calcular  $u^{pr} = \frac{\partial \Psi}{\partial z}$ y $w^{pr} = -\frac{\partial \Psi}{\partial x}$
	Resolver la ecuación de Transporte (ec. $\coderef{T.2}$) con Backwards Euler para obtener $T^{pr}$
	Resolver la ecuación de Darcy (ec. $\coderef{D.1}$) con Backwards Euler para obtener $\Psi^{k+1}$
	Calcular $u^{k+1} = \frac{\partial \Psi}{\partial z}$ y $w^{k+1} = -\frac{\partial \Psi}{\partial x}$
	Resolver la ecuación de Transporte (ec. $\coderef{T.2}$) con Crank Nicolson para obtener $T^{k+1}$
	Usar la condición de CFL para obtener $dt$
FIN
\end{sourcecode}


\begin{sourcecode}{plaintext}{}
Epicˢᵘᵖᵉʳˢᶜʳᶦᵖᵗ¹²³¹²αάγΓδΐόϋ
\end{sourcecode}

\begin{sourcecode}[\label{ejemplo-tcl}]{tcl}{Código en TCL.}
	proc file'hexdump filename {
		set fp [open $filename]
		fconfigure $fp -translation binary
		set n 0
		while {![eof $fp]} {
			set bytes [read $fp 16]
			regsub -all {[^ -þ]} $bytes . ascii
			puts [format "%04X %-48s %-16s" $n [hexdump $bytes] $ascii]
			incr n 16
		}
		close $fp
	}
	
	proc hexdump string {
		binary scan $string H* hex
		regexp -all -inline .. $hex
	}
\end{sourcecode}

\begin{sourcecode}{rust}{Ejemplo en Rust.}
use std::rc::Rc;

/// upside-down tree with a designated position (the *stack pointer*)
/// and *nodes* of type `A`.
#[derive(Clone, Debug)]
pub struct TreeStack<A> {
	parent: Option<(usize, Rc<TreeStack<A>>)>,
	value: A,
	children: Vec<Option<Rc<TreeStack<A>>>>,
}

impl<A> TreeStack<A> {
	/// Creates a new `TreeStack<A>` with root label `a`.
	pub fn new(a: A) -> Self {
		TreeStack { value: a, children: Vec::new(), parent: None }
	}
	
	/// Applies a function `FnMut(&A) -> B` to every node in a `TreeStack<A>`.
	pub fn map<F, B>(&self, f: &mut F) -> TreeStack<B>
	where F: FnMut(&A) -> B,
	{
		let new_value = f(&self.value);
		let new_parent = match self.parent {
			Some((i, ref p)) => Some((i, Rc::new(p.map(f)))),
			None => None,
		};
		let new_children = self.children
		.iter()
		.map(|o| o.clone().map(|v| Rc::new(v.map(f))))
		.collect();
		TreeStack {
			parent: new_parent,
			value: new_value,
			children: new_children
		}
	}
}
\end{sourcecode}

\begin{sourcecode}[]{opensees}{Programa en TCL+OpenSees.}
# --------------------------------------------------------------------------------------------------
# Example 6. 2D RC Frame
#		Silvia Mazzoni & Frank McKenna, 2006
# nonlinearBeamColumn element, inelastic fiber section
#

# SET UP ----------------------------------------------------------------------------
wipe;				# clear memory of all past model definitions
model BasicBuilder -ndm 2 -ndf 3;	# Define the model builder, ndm=#dimension, ndf=#dofs
set dataDir Data;			# set up name of data directory (can remove this)
file mkdir $dataDir; 			# create data directory
set GMdir "../GMfiles/";			# ground-motion file directory
source LibUnits.tcl;			# define units
source DisplayPlane.tcl;		# procedure for displaying a plane in model
source DisplayModel2D.tcl;		# procedure for displaying 2D perspective of model
source BuildRCrectSection.tcl;		# procedure for definining RC fiber section

# define GEOMETRY -------------------------------------------------------------
# define structure-geometry paramters
set LCol [expr 14*$ft];		# column height
set LBeam [expr 24*$ft];		# beam length
set NStory 3;			# number of stories above ground level -------------- you can change this.
set NBay 3;			# number of bays (max 9) ------------------------------you can change this.

# define NODAL COORDINATES
for {set level 1} {$level <=[expr $NStory+1]} {incr level 1} {
	set Y [expr ($level-1)*$LCol];
	for {set pier 1} {$pier <= [expr $NBay+1]} {incr pier 1} {
		set X [expr ($pier-1)*$LBeam];
		set nodeID [expr $level*10+$pier]
		node $nodeID $X $Y;		# actually define node
	}
}


# determine support nodes where ground motions are input, for multiple-support excitation
set iSupportNode ""
set level 1
for {set pier 1} {$pier <= [expr $NBay+1]} {incr pier 1} {
	set nodeID [expr $level*10+$pier]
	lappend iSupportNode $nodeID
}


# BOUNDARY CONDITIONS
fixY 0.0 1 1 0;		# pin all Y=0.0 nodes

# calculated MODEL PARAMETERS, particular to this model
puts "Number of Stories: $NStory Number of bays: $NBay"
# Set up parameters that are particular to the model for displacement control
set IDctrlNode [expr ($NStory+1)*10+1];		# node where displacement is read for displacement control
set IDctrlDOF 1;		# degree of freedom of displacement read for displacement control
set LBuilding [expr $NStory*$LCol];	# total building height

# Define SECTIONS -------------------------------------------------------------
set SectionType FiberSection;		# options: Elastic FiberSection

# define section tags:
set ColSecTag 1
set BeamSecTag 2

# Section Properties:
set HCol [expr 24*$in];		# square-Column width
set BCol $HCol
set HBeam [expr 42*$in];		# Beam depth -- perpendicular to bending axis
set BBeam [expr 24*$in];		# Beam width -- parallel to bending axis

if {$SectionType == "Elastic"} {
	# material properties:
	set fc 4000*$psi;			# concrete nominal compressive strength
	set Ec [expr 57*$ksi*pow($fc/$psi,0.5)];	# concrete Young's Modulus
	# column section properties:
	set AgCol [expr $HCol*$BCol];		# rectuangular-Column cross-sectional area
	set IzCol [expr 0.5*1./12*$BCol*pow($HCol,3)];	# about-local-z Rect-Column gross moment of inertial
	# beam sections:
	set AgBeam [expr $HBeam*$BBeam];		# rectuangular-Beam cross-sectional area
	set IzBeam [expr 0.5*1./12*$BBeam*pow($HBeam,3)];	# about-local-z Rect-Beam cracked moment of inertial
	
	section Elastic $ColSecTag $Ec $AgCol $IzCol 
	section Elastic $BeamSecTag $Ec $AgBeam $IzBeam 
	
} elseif {$SectionType == "FiberSection"} {
	# MATERIAL parameters 
	source LibMaterialsRC.tcl;	# define library of Reinforced-concrete Materials
	
	# FIBER SECTION properties 
	# Column section geometry:
	set cover [expr 2.5*$in];	# rectangular-RC-Column cover
	set numBarsTopCol 8;		# number of longitudinal-reinforcement bars on top layer
	set numBarsBotCol 8;		# number of longitudinal-reinforcement bars on bottom layer
	set numBarsIntCol 6;		# TOTAL number of reinforcing bars on the intermediate layers
	set barAreaTopCol [expr 1.*$in*$in];	# longitudinal-reinforcement bar area
	set barAreaBotCol [expr 1.*$in*$in];	# longitudinal-reinforcement bar area
	set barAreaIntCol [expr 1.*$in*$in];	# longitudinal-reinforcement bar area
	
	set numBarsTopBeam 6;		# number of longitudinal-reinforcement bars on top layer
	set numBarsBotBeam 6;		# number of longitudinal-reinforcement bars on bottom layer
	set numBarsIntBeam 2;		# TOTAL number of reinforcing bars on the intermediate layers
	set barAreaTopBeam [expr 1.*$in*$in];	# longitudinal-reinforcement bar area
	set barAreaBotBeam [expr 1.*$in*$in];	# longitudinal-reinforcement bar area
	set barAreaIntBeam [expr 1.*$in*$in];	# longitudinal-reinforcement bar area
	
	set nfCoreY 20;		# number of fibers in the core patch in the y direction
	set nfCoreZ 20;		# number of fibers in the core patch in the z direction
	set nfCoverY 20;		# number of fibers in the cover patches with long sides in the y direction
	set nfCoverZ 20;		# number of fibers in the cover patches with long sides in the z direction
	# rectangular section with one layer of steel evenly distributed around the perimeter and a confined core.
	BuildRCrectSection $ColSecTag $HCol $BCol $cover $cover $IDconcCore  $IDconcCover $IDSteel $numBarsTopCol $barAreaTopCol $numBarsBotCol $barAreaBotCol $numBarsIntCol $barAreaIntCol  $nfCoreY $nfCoreZ $nfCoverY $nfCoverZ
	BuildRCrectSection $BeamSecTag $HBeam $BBeam $cover $cover $IDconcCore  $IDconcCover $IDSteel $numBarsTopBeam $barAreaTopBeam $numBarsBotBeam $barAreaBotBeam $numBarsIntBeam $barAreaIntBeam  $nfCoreY $nfCoreZ $nfCoverY $nfCoverZ
	
} else {
	puts "No section has been defined"
	return -1
}


# define ELEMENTS
# set up geometric transformations of element
#   separate columns and beams, in case of P-Delta analysis for columns
set IDColTransf 1; # all columns
set IDBeamTransf 2; # all beams
set ColTransfType Linear ;			# options, Linear PDelta Corotational 
geomTransf $ColTransfType $IDColTransf  ; 	# only columns can have PDelta effects (gravity effects)
geomTransf Linear $IDBeamTransf

# Define Beam-Column Elements
set np 5;	# number of Gauss integration points for nonlinear curvature distribution-- np=2 for linear distribution ok
# columns
set N0col 100;	# column element numbers
set level 0
for {set level 1} {$level <=$NStory} {incr level 1} {
	for {set pier 1} {$pier <= [expr $NBay+1]} {incr pier 1} {
		set elemID [expr $N0col  + $level*10 +$pier]
		set nodeI [expr  $level*10 + $pier]
		set nodeJ  [expr  ($level+1)*10 + $pier]
		element nonlinearBeamColumn $elemID $nodeI $nodeJ $np $ColSecTag $IDColTransf;		# columns
	}
}
# beams
set N0beam 200;	# beam element numbers
set M0 0
for {set level 2} {$level <=[expr $NStory+1]} {incr level 1} {
	for {set bay 1} {$bay <= $NBay} {incr bay 1} {
		set elemID [expr $N0beam + $level*10 +$bay]
		set nodeI [expr  $M0 + $level*10 + $bay]
		set nodeJ  [expr  $M0 + $level*10 + $bay+1]
		element nonlinearBeamColumn $elemID $nodeI $nodeJ $np $BeamSecTag $IDBeamTransf;	# beams
	}
}

# Define GRAVITY LOADS, weight and masses
# calculate dead load of frame, assume this to be an internal frame (do LL in a similar manner)
# calculate distributed weight along the beam length
set GammaConcrete [expr 150*$pcf];   		# Reinforced-Concrete floor slabs
set Tslab [expr 6*$in];			# 6-inch slab
set Lslab [expr 2*$LBeam/2]; 			# assume slab extends a distance of $LBeam1/2 in/out of plane
set Qslab [expr $GammaConcrete*$Tslab*$Lslab]; 
set QdlCol [expr $GammaConcrete*$HCol*$BCol];	# self weight of Column, weight per length
set QBeam [expr $GammaConcrete*$HBeam*$BBeam];	# self weight of Beam, weight per length
set QdlBeam [expr $Qslab + $QBeam]; 	# dead load distributed along beam.
set WeightCol [expr $QdlCol*$LCol];  		# total Column weight
set WeightBeam [expr $QdlBeam*$LBeam]; 	# total Beam weight

# assign masses to the nodes that the columns are connected to 
# each connection takes the mass of 1/2 of each element framing into it (mass=weight/$g)
set iFloorWeight ""
set WeightTotal 0.0
set sumWiHi 0.0;		# sum of storey weight times height, for lateral-load distribution
for {set level 2} {$level <=[expr $NStory+1]} {incr level 1} { ;		
	set FloorWeight 0.0
	if {$level == [expr $NStory+1]}  {
		set ColWeightFact 1;		# one column in top story
	} else {
		set ColWeightFact 2;		# two columns elsewhere
	}
	for {set pier 1} {$pier <= [expr $NBay+1]} {incr pier 1} {;
		if {$pier == 1 || $pier == [expr $NBay+1]} {
			set BeamWeightFact 1;	# one beam at exterior nodes
		} else {;
			set BeamWeightFact 2;	# two beams elewhere
		}
		set WeightNode [expr $ColWeightFact*$WeightCol/2 + $BeamWeightFact*$WeightBeam/2]
		set MassNode [expr $WeightNode/$g];
		set nodeID [expr $level*10+$pier]
		mass $nodeID $MassNode 0.0 0.0 0.0 0.0 0.0;			# define mass
		set FloorWeight [expr $FloorWeight+$WeightNode];
	}
	lappend iFloorWeight $FloorWeight
	set WeightTotal [expr $WeightTotal+ $FloorWeight]
	set sumWiHi [expr $sumWiHi+$FloorWeight*($level-1)*$LCol];		# sum of storey weight times height, for lateral-load distribution
}
set MassTotal [expr $WeightTotal/$g];						# total mass

# LATERAL-LOAD distribution for static pushover analysis
# calculate distribution of lateral load based on mass/weight distributions along building height
# Fj = WjHj/sum(WiHi)  * Weight   at each floor j
set iFj ""
for {set level 2} {$level <=[expr $NStory+1]} {incr level 1} { ;	
	set FloorWeight [lindex $iFloorWeight [expr $level-1-1]];
	set FloorHeight [expr ($level-1)*$LCol];
	set NodeFactor [expr $NBay+1];
	lappend iFj [expr $FloorWeight*$FloorHeight/$sumWiHi/$NodeFactor*$WeightTotal];		# per node per floor
}
# create node and load vectors for lateral-load distribution in static analysis
set iFPush ""
set iNodePush ""
for {set level 2} {$level <=[expr $NStory+1]} {incr level 1} {
	set FPush [lindex $iFj [expr $level-1-1]];		# lateral load coefficient
	for {set pier 1} {$pier <= [expr $NBay+1]} {incr pier 1} {
		set nodeID [expr $level*10+$pier]
		lappend iNodePush $nodeID
		lappend iFPush $FPush
	}
}

# Define RECORDERS -------------------------------------------------------------
set FreeNodeID [expr ($NStory+1)*10+($NBay+1)];					# ID: free node
set SupportNodeFirst [lindex $iSupportNode 0];						# ID: first support node
set SupportNodeLast [lindex $iSupportNode [expr [llength $iSupportNode]-1]];			# ID: last support node
set FirstColumn [expr $N0col+1*10+1];							# ID: first column
recorder Node -file $dataDir/DFree.out -time -node $FreeNodeID  -dof 1 2 3 disp;				# displacements of free node
recorder Node -file $dataDir/DBase.out -time -nodeRange $SupportNodeFirst $SupportNodeLast -dof 1 2 3 disp;	# displacements of support nodes


# Define DISPLAY -------------------------------------------------------------
DisplayModel2D NodeNumbers

# define GRAVITY -------------------------------------------------------------
# GRAVITY LOADS # define gravity load applied to beams and columns -- eleLoad applies loads in local coordinate axis
pattern Plain 101 Linear {
	for {set level 1} {$level <=$NStory} {incr level 1} {
		for {set pier 1} {$pier <= [expr $NBay+1]} {incr pier 1} {
			set elemID [expr $N0col  + $level*10 +$pier]
			eleLoad -ele $elemID -type -beamUniform 0 -$QdlCol; 	# COLUMNS
		}
	}
	for {set level 2} {$level <=[expr $NStory+1]} {incr level 1} {
		for {set bay 1} {$bay <= $NBay} {incr bay 1} {
			set elemID [expr $N0beam + $level*10 +$bay]
			eleLoad -ele $elemID  -type -beamUniform -$QdlBeam; 	# BEAMS
		}
	}
}
# Gravity-analysis parameters -- load-controlled static analysis
set Tol 1.0e-8;			# convergence tolerance for test
variable constraintsTypeGravity Plain;		# default;
if {  [info exists RigidDiaphragm] == 1} {
	if {$RigidDiaphragm=="ON"} {
		variable constraintsTypeGravity Lagrange;	#  large model: try Transformation
	};	# if rigid diaphragm is on
};	# if rigid diaphragm exists
constraints $constraintsTypeGravity ;     		# how it handles boundary conditions
numberer RCM;			# renumber dof's to minimize band-width (optimization), if you want to
system BandGeneral ;		# how to store and solve the system of equations in the analysis (large model: try UmfPack)
test NormDispIncr $Tol 6 ; 		# determine if convergence has been achieved at the end of an iteration step
algorithm Newton;			# use Newton's solution algorithm: updates tangent stiffness at every iteration
set NstepGravity 10;  		# apply gravity in 10 steps
set DGravity [expr 1./$NstepGravity]; 	# first load increment;
integrator LoadControl $DGravity;	# determine the next time step for an analysis
analysis Static;			# define type of analysis static or transient
analyze $NstepGravity;		# apply gravity

# ------------------------------------------------- maintain constant gravity loads and reset time to zero
loadConst -time 0.0

puts "Model Built"
\end{sourcecode}

\begin{sourcecode}[]{opensees}{Otro de opensees}
# SET UP MODEL ------------------------------------------------------------------------------------
wipe

# Define model builder
# --------------------
model basic -ndm 2 -ndf 3
source MomentCurvature.tcl;         # Llama a la rutina de Momento-Curvatura

# Define materials for nonlinear columns
# ------------------------------------------
# CONCRETE
# Core and cover concrete (unconfined)

set fc -25.0;      # MPa
set eco -0.0025; 
set fcu -0.00;     # MPa
set ecu -0.005;
set lambda 0.00001;
set ft [expr 0.62*pow(25,0.5)]; #MPa
set Ecs 1e10;      #MPa
# uniaxialMaterial Concrete02 $matTag $fpc $epsc0 $fpcu $epsU $lambda $ft $Ets
uniaxialMaterial Concrete02   1  $fc $eco $fcu $ecu $lambda $ft $Ecs ;

# STEEL
# Reinforcing steel 

set fy 420.0;     # MPa
set E 2e5;        # MPa
set b 0.01;

#uniaxialMaterial Steel02 $matTag $Fy $E $b $R0 $cR1 $cR2 <$a1 $a2 $a3 $a4 $sigInit>
uniaxialMaterial Steel02  2  $fy $E $b 15.0 0.10 0.15;

# Define cross-section for nonlinear columns
# ------------------------------------------

# set some paramaters
set b  350.0;   #mm 
set h  700.0;   #mm
set b2 700.0;   #mm
set d  645.0;   #mm
set dp  55.0;   #mm
set tf 110.0;   #mm
set As  1050.0;  #mm²  area de cada barra
set Asp 1050.0;  #mm²

# some variables derived from the parameters
set y1 [expr $h/2.0];
set z1 [expr $b/2.0];
set z2 [expr $b2/2.0];

section Fiber 1 {
	
	# Create the concrete fibers
	#Beam web
	#patch rect $matTag $numSubdivY $numSubdivZ $yI $zI $yJ $zJ
	patch rect 1 59 1 [expr -$y1+$tf] [expr -$z1] [expr $y1] [expr $z1];
	
	#Beam flange 
	patch rect 1 11 1 [expr -$y1] [expr -$z2] [expr -$y1+$tf] [expr $z2];
	
	# Create the reinforcing fibers (left, right)
	#layer straight $matTag $numFiber $areaFiber $yStart $zStart $yEnd $zEnd
	layer straight 2 2 $As [expr $y1-$dp] [expr $z1-$dp] [expr $y1-$dp] [expr -$z1+$dp];
	layer straight 2 2 $As [expr -$y1+$dp] [expr $z1-$dp] [expr -$y1+$dp] [expr -$z1+$dp];
	
}   

# Estimate yield curvature
# (Assuming no axial load and only top and bottom steel)
set epsy [expr $fy/$E]  ;# steel yield strain
set Ky [expr $epsy/(0.7*$d)];

# Print estimate to standard output
puts "Estimated yield curvature: $Ky"

# Set axial load 
set P -10e3;  # [N] 

set mu 26;      # Target ductility for analysis
set numIncr 500;    # Number of analysis increments

# Call the section analysis procedure
MomentCurvature 1 $P [expr $Ky*$mu] $numIncr

puts "ANALISIS FINALIZADO"

# SET UP MODEL ------------------------------------------------------------------------------------
wipe                                                                                
model BasicBuilder -ndm 2 -ndf 3;   # Model builder, ndm=#dims, ndf=#dofs, /2D 3GDL por nudo                                                                            
set dataDir Data_P2_2c;             # Define una carpeta para guardar los archivos de salida
file mkdir $dataDir;                # Crea la carpeta
source lib_DisplayPlane.tcl;        # Opcional Rutina que despliega el modelo
source lib_DisplayModel2D.tcl;      # Opcional Rutina que dibuja 

#Creamos las constantes del problema 
# Unidades m, kN
set L 2.0;       #m
set A 0.1;       #m2 
set I 8e-3;
set E 2e10;
set k 10;      #kN*m
set theta [expr -0.01*3.1416/180];

puts "Constantes ...OK"

#Coordenadas de nodos en m
#node $nodeTag (ndm $coords) <-mass (ndf $massValues)>

node 1 0.0 0.0; 
node 2 [expr $L*sin($theta) ] $L;
node 3 [expr 2*$L*sin($theta) ] [expr 2*$L];
node 4 [expr 3*$L*sin($theta) ] [expr 3*$L];
node 5 [expr 4*$L*sin($theta) ] [expr 4*$L];

#Resortes
node 11 0.0 0.0; 
node 12 [expr $L*sin($theta) ] $L;
node 13 [expr 2*$L*sin($theta) ] [expr 2*$L];
node 14 [expr 3*$L*sin($theta) ] [expr 3*$L];

puts "Nodos ...OK"

#Condiciones de borde 
fix 1 1 1 1;
fix 11 1 1 0;

#Resortes
equalDOF 2 12 1 2;
equalDOF 3 13 1 2;
equalDOF 4 14 1 2;

puts "Condiciones de borde ...OK"

#Materiales 
uniaxialMaterial Elastic 1 $k;

#Matriz de transformación geométrica (NO LINEALIDAD GEOMÉTRICA)
geomTransf Linear 1;          # No considera no linealidad geométrica
geomTransf PDelta 2;          # Considera efectos P-delta
geomTransf Corotational 3;    # Considera efectos P-delta y P-Delta

#Creación de elementos
# Barras Rígidas
element elasticBeamColumn 1 11 2 $A $E $I 3;
element elasticBeamColumn 2 12 3 $A $E $I 3;
element elasticBeamColumn 3 13 4 $A $E $I 3;
element elasticBeamColumn 4 14 5 $A $E $I 3;

#Resortes
element zeroLength 11 1 11 -mat 1 -dir 3;  # La dirección es 6 (3D) o 3 (2D) funciona con los 2
element zeroLength 12 2 12 -mat 1 -dir 3;  
element zeroLength 13 3 13 -mat 1 -dir 3;
element zeroLength 14 4 14 -mat 1 -dir 3;

puts "Materiales ...OK"

# DISPLAY -----------------------------------------------------------------------------------------
DisplayModel2D NodeNumbers

#Creación del Patrón de cargas
set P -1.0;

pattern Plain 1 Linear {
	load 5 0.0 $P 0.0;
} 

#RECORDERS
recorder Node -file $dataDir/NodeReact.out -time -node 1 -dof 1 2 3 reaction;
recorder Node -file $dataDir/Node2Disp.out -time -node 2 -dof 1 2 3 disp;
recorder Node -file $dataDir/Node3Disp.out -time -node 3 -dof 1 2 3 disp;
recorder Node -file $dataDir/Node4Disp.out -time -node 4 -dof 1 2 3 disp;
recorder Node -file $dataDir/Node5Disp.out -time -node 5 -dof 1 2 3 disp;

# Display Deformed Shape: (Opcional) Dibuja la deformada
set ViewScale 0.1;               # amplify display of deformed shape
DisplayModel2D DeformedShape $ViewScale ;   # display deformed shape, the scaling factor needs to be adjusted for each model

#Definición del análisis
#integrator DisplacementControl $node $dof $incr <$numIter $<math>\Delta U \text{min} </math> $<math>\Delta U \text{max}</math>>
set dU -0.001;
integrator DisplacementControl 5 2 $dU;
# test de tolerancia
set tol 1.0e-15;                # tolerance 
set iter 5000;                 # max iterations
set pFlag 0;                   # No desplega info de cada iteración
set nType 2;                   # Tipo de 
test EnergyIncr $tol $iter $pFlag $nType;
algorithm KrylovNewton; 
numberer Plain;
constraints Plain;
system SparseGeneral;
analysis Static;
analyze 4100;

puts "Analysis...   OK"
puts "End Time: [getTime]"

# --------------------------------------------------------------------------------------------------
# Example 1. portal frame in 2D
# dynamic earthquake analysis of Portal Frame, with gravity.
# all units are in kip, inch, second
# elasticBeamColumn ELEMENT
#		Silvia Mazzoni & Frank McKenna, 2006
#
#    ^Y
#    |
#    3_________(3)________4       __ 
#    |                                    |          | 
#    |                                    |          |
#    |                                    |          |
#  (1)                                 (2)       LCol
#    |                                    |          |
#    |                                    |          |
#    |                                    |          |
#  =1=                               =2=      _|_  -------->X
#    |----------LBeam------------|
#

# SET UP ----------------------------------------------------------------------------
wipe;						# clear opensees model
model basic -ndm 2 -ndf 3;				# 2 dimensions, 3 dof per node
file mkdir data; 					# create data directory

# define GEOMETRY -------------------------------------------------------------
# nodal coordinates:
node 1 0 0;					# node#, X Y
node 2 504 0
node 3 0 432
node 4 504 432 

# Single point constraints -- Boundary Conditions
fix 1 1 1 1; 			# node DX DY RZ
fix 2 1 1 1; 			# node DX DY RZ
fix 3 0 0 0
fix 4 0 0 0

# nodal masses:
mass 3 5.18 0. 0.;					# node#, Mx My Mz, Mass=Weight/g.
mass 4 5.18 0. 0.

# Define ELEMENTS -------------------------------------------------------------
# define geometric transformation: performs a linear geometric transformation of beam stiffness and resisting force from the basic system to the global-coordinate system
geomTransf Linear 1;  		# associate a tag to transformation

# connectivity: (make A very large, 10e6 times its actual value)
element elasticBeamColumn 1 1 3 3600000000 4227 1080000 1;	# element elasticBeamColumn $eleTag $iNode $jNode $A $E $Iz $transfTag
element elasticBeamColumn 2 2 4 3600000000 4227 1080000 1
element elasticBeamColumn 3 3 4 5760000000 4227 4423680 1

# Define RECORDERS -------------------------------------------------------------
recorder Node -file Data/DFree.out -time -node 3 4 -dof 1 2 3 disp;			# displacements of free nodes
recorder Node -file Data/DBase.out -time -node 1 2 -dof 1 2 3 disp;			# displacements of support nodes
recorder Node -file Data/RBase.out -time -node 1 2 -dof 1 2 3 reaction;		# support reaction
recorder Drift -file Data/Drift.out -time -iNode 1 2 -jNode 3 4 -dof 1  -perpDirn 2 ;	# lateral drift
recorder Element -file Data/FCol.out -time -ele 1 2 globalForce;			# element forces -- column
recorder Element -file Data/FBeam.out -time -ele 3 globalForce;			# element forces -- beam

# define GRAVITY -------------------------------------------------------------
pattern Plain 1 Linear {
	eleLoad -ele 3 -type -beamUniform -7.94 ; # distributed superstructure-weight on beam
}
constraints Plain;     				# how it handles boundary conditions
numberer Plain;					# renumber dof's to minimize band-width (optimization), if you want to
system BandGeneral;				# how to store and solve the system of equations in the analysis
test NormDispIncr 1.0e-8 6 ; 				# determine if convergence has been achieved at the end of an iteration step
algorithm Newton;					# use Newton's solution algorithm: updates tangent stiffness at every iteration
integrator LoadControl 0.1;				# determine the next time step for an analysis, # apply gravity in 10 steps
analysis Static					# define type of analysis static or transient
analyze 10;					# perform gravity analysis
loadConst -time 0.0;				# hold gravity constant and restart time

# DYNAMIC ground-motion analysis -------------------------------------------------------------
# create load pattern
set accelSeries "Series -dt 0.01 -filePath BM68elc.acc -factor 1";	# define acceleration vector from file (dt=0.01 is associated with the input file gm)
pattern UniformExcitation 2 1 -accel $accelSeries;		# define where and how (pattern tag, dof) acceleration is applied
rayleigh 0. 0. 0. [expr 2*0.02/pow([eigen 1],0.5)];		# set damping based on first eigen mode

# create the analysis
wipeAnalysis;					# clear previously-define analysis parameters
constraints Plain;     				# how it handles boundary conditions
numberer Plain;					# renumber dof's to minimize band-width (optimization), if you want to
system BandGeneral;				# how to store and solve the system of equations in the analysis
test NormDispIncr 1.0e-8 10;				# determine if convergence has been achieved at the end of an iteration step
algorithm Newton;					# use Newton's solution algorithm: updates tangent stiffness at every iteration
integrator Newmark 0.5 0.25 ;			# determine the next time step for an analysis
analysis Transient;					# define type of analysis: time-dependent
analyze 1000 0.02;					# apply 1000 0.02-sec time steps in analysis


puts "Done!"

set coverY [expr $HSec/2.0];		# The distance from the section z-axis to the edge of the cover concrete -- outer edge of cover concrete
set coverZ [expr $BSec/2.0];		# The distance from the section y-axis to the edge of the cover concrete -- outer edge of cover concrete
set coreY [expr $coverY-$coverH];		# The distance from the section z-axis to the edge of the core concrete --  edge of the core concrete/inner edge of cover concrete
set coreZ [expr $coverZ-$coverB];		# The distance from the section y-axis to the edge of the core concrete --  edge of the core concrete/inner edge of cover concrete
set numBarsInt [expr $numBarsIntTot/2];	# number of intermediate bars per side

# Define the fiber section
section fiberSec $id {
	# Define the core patch
	patch quadr $coreID $nfCoreZ $nfCoreY -$coreY $coreZ -$coreY -$coreZ $coreY -$coreZ $coreY $coreZ
	
	# Define the four cover patches
	patch quadr $coverID 2 $nfCoverY -$coverY $coverZ -$coreY $coreZ $coreY $coreZ $coverY $coverZ
	patch quadr $coverID 2 $nfCoverY -$coreY -$coreZ -$coverY -$coverZ $coverY -$coverZ $coreY -$coreZ
	patch quadr $coverID $nfCoverZ 2 -$coverY $coverZ -$coverY -$coverZ -$coreY -$coreZ -$coreY $coreZ
	patch quadr $coverID $nfCoverZ 2 $coreY $coreZ $coreY -$coreZ $coverY -$coverZ $coverY $coverZ	
	
	# define reinforcing layers
	layer straight $steelID $numBarsInt $barAreaInt  -$coreY $coreZ $coreY $coreZ;	# intermediate skin reinf. +z
	layer straight $steelID $numBarsInt $barAreaInt  -$coreY -$coreZ $coreY -$coreZ;	# intermediate skin reinf. -z
	layer straight $steelID $numBarsTop $barAreaTop $coreY $coreZ $coreY -$coreZ;	# top layer reinfocement
	layer straight $steelID $numBarsBot $barAreaBot  -$coreY $coreZ  -$coreY -$coreZ;	# bottom layer reinforcement
	
};	# end of fibersection definition

wipe;				# clear memory of all past model definitions
model BasicBuilder -ndm 3 -ndf 6;	# Define the model builder, ndm=#dimension, ndf=#dofs
set dataDir Data;			# set up name of data directory
file mkdir $dataDir; 			# create data directory
set GMdir "../GMfiles";		# ground-motion file directory
source LibUnits.tcl;			# define units
source DisplayPlane.tcl;		# procedure for displaying a plane in model
source DisplayModel3D.tcl;		# procedure for displaying 3D perspectives of model
source BuildRCrectSection.tcl;		# procedure for definining RC fiber section

# ------ frame configuration
set NStory 2;			# number of stories above ground level
set NBay 2;			# number of bays in X direction
set NBayZ 2;			# number of bays in Z direction
puts "Number of Stories in Y: $NStory Number of bays in X: $NBay Number of bays in Z: $NBayZ"
set NFrame [expr $NBayZ + 1];	# actually deal with frames in Z direction, as this is an easy extension of the 2d model

# define GEOMETRY -------------------------------------------------------------
# define structure-geometry paramters
set LCol [expr 14*$ft];		# column height (parallel to Y axis)
set LBeam [expr 24*$ft];		# beam length (parallel to X axis)
set LGird [expr 24*$ft];		# girder length (parallel to Z axis)

# define NODAL COORDINATES
set Dlevel 10000;	# numbering increment for new-level nodes
set Dframe 100;	# numbering increment for new-frame nodes
for {set frame 1} {$frame <=[expr $NFrame]} {incr frame 1} {
	set Z [expr ($frame-1)*$LGird];
	for {set level 1} {$level <=[expr $NStory+1]} {incr level 1} {
		set Y [expr ($level-1)*$LCol];
		for {set pier 1} {$pier <= [expr $NBay+1]} {incr pier 1} {
			set X [expr ($pier-1)*$LBeam];
			set nodeID [expr $level*$Dlevel+$frame*$Dframe+$pier]
			node $nodeID $X $Y $Z;		# actually define node
		}
	}
}

# rigid diaphragm nodes
set RigidDiaphragm ON ;		# options: ON, OFF. specify this before the analysis parameters are set the constraints are handled differently.
set Xa [expr ($NBay*$LBeam)/2];		# mid-span coordinate for rigid diaphragm
set Za [expr ($NFrame-1)*$LGird/2];
set iMasterNode ""
for {set level 2} {$level <=[expr $NStory+1]} {incr level 1} {
	set Y [expr ($level-1)*$LCol];
	# rigid-diaphragm nodes in center of each diaphram
	set MasterNodeID [expr 9900+$level]
	node $MasterNodeID $Xa $Y $Za;		# master nodes for rigid diaphragm
	fix $MasterNodeID 0  1  0  1  0  1;		# constrain other dofs that don't belong to rigid diaphragm control
	lappend iMasterNode $MasterNodeID
	set perpDirn 2;				# perpendicular to plane of rigid diaphragm
	for {set frame 1} {$frame <=[expr $NFrame]} {incr frame 1} {
		for {set pier 1} {$pier <= [expr $NBay+1]} {incr pier 1} {
			set nodeID [expr $level*$Dlevel+$frame*$Dframe+$pier]
			rigidDiaphragm $perpDirn $MasterNodeID $nodeID; 	# define Rigid Diaphram,
		}
	}
}

# determine support nodes where ground motions are input, for multiple-support excitation
set iSupportNode ""
for {set frame 1} {$frame <=[expr $NFrame]} {incr frame 1} {
	set level 1
	for {set pier 1} {$pier <= [expr $NBay+1]} {incr pier 1} {
		set nodeID [expr $level*$Dlevel+$frame*$Dframe+$pier]
		lappend iSupportNode $nodeID
	}
}

# BOUNDARY CONDITIONS
fixY 0.0  1 1 1 0 1 0;		# pin all Y=0.0 nodes

# calculated MODEL PARAMETERS, particular to this model
# Set up parameters that are particular to the model for displacement control
set IDctrlNode [expr int(($NStory+1)*$Dlevel+(1*$Dframe)+1)];		# node where displacement is read for displacement control
set IDctrlDOF 1;					# degree of freedom of displacement read for displacement control
set LBuilding [expr $NStory*$LCol];			# total building height

# Define SECTIONS -------------------------------------------------------------
set SectionType FiberSection;		# options: Elastic FiberSection

# define section tags:
set ColSecTag 1
set BeamSecTag 2
set GirdSecTag 3
set ColSecTagFiber 4
set BeamSecTagFiber 5
set GirdSecTagFiber 6
set SecTagTorsion 70

# Section Properties:
set HCol [expr 18*$in];		# square-Column width
set BCol $HCol
set HBeam [expr 24*$in];		# Beam depth -- perpendicular to bending axis
set BBeam [expr 18*$in];		# Beam width -- parallel to bending axis
set HGird [expr 24*$in];		# Girder depth -- perpendicular to bending axis
set BGird [expr 18*$in];		# Girder width -- parallel to bending axis

if {$SectionType == "Elastic"} {
	# material properties:
	set fc 4000*$psi;			# concrete nominal compressive strength
	set Ec [expr 57*$ksi*pow($fc/$psi,0.5)];	# concrete Young's Modulus
	set nu 0.2;			# Poisson's ratio
	set Gc [expr $Ec/2./[expr 1+$nu]];  	# Torsional stiffness Modulus
	set J $Ubig;			# set large torsional stiffness
	# column section properties:
	set AgCol [expr $HCol*$BCol];		# rectuangular-Column cross-sectional area
	set IzCol [expr 0.5*1./12*$BCol*pow($HCol,3)];	# about-local-z Rect-Column gross moment of inertial
	set IyCol [expr 0.5*1./12*$HCol*pow($BCol,3)];	# about-local-z Rect-Column gross moment of inertial
	# beam sections:
	set AgBeam [expr $HBeam*$BBeam];		# rectuangular-Beam cross-sectional area
	set IzBeam [expr 0.5*1./12*$BBeam*pow($HBeam,3)];	# about-local-z Rect-Beam cracked moment of inertial
	set IyBeam [expr 0.5*1./12*$HBeam*pow($BBeam,3)];	# about-local-y Rect-Beam cracked moment of inertial
	# girder sections:
	set AgGird [expr $HGird*$BGird];		# rectuangular-Girder cross-sectional area
	set IzGird [expr 0.5*1./12*$BGird*pow($HGird,3)];	# about-local-z Rect-Girder cracked moment of inertial
	set IyGird [expr 0.5*1./12*$HGird*pow($BGird,3)];	# about-local-y Rect-Girder cracked moment of inertial
	
	section Elastic $ColSecTag $Ec $AgCol $IzCol $IyCol $Gc $J
	section Elastic $BeamSecTag $Ec $AgBeam $IzBeam $IyBeam $Gc $J
	section Elastic $GirdSecTag $Ec $AgGird $IzGird $IyGird $Gc $J
	
	set IDconcCore  1;		# material numbers for recorder (this stressstrain recorder will be blank, as this is an elastic section)
	set IDSteel  2;			# material numbers for recorder (this stressstrain recorder will be blank, as this is an elastic section)
	
} elseif {$SectionType == "FiberSection"} {
	# MATERIAL parameters 
	source LibMaterialsRC.tcl;	# define library of Reinforced-concrete Materials
	# FIBER SECTION properties 
	# Column section geometry:
	set cover [expr 2.5*$in];	# rectangular-RC-Column cover
	set numBarsTopCol 8;		# number of longitudinal-reinforcement bars on top layer
	set numBarsBotCol 8;		# number of longitudinal-reinforcement bars on bottom layer
	set numBarsIntCol 6;		# TOTAL number of reinforcing bars on the intermediate layers
	set barAreaTopCol [expr 1.*$in*$in];	# longitudinal-reinforcement bar area
	set barAreaBotCol [expr 1.*$in*$in];	# longitudinal-reinforcement bar area
	set barAreaIntCol [expr 1.*$in*$in];	# longitudinal-reinforcement bar area
	
	set numBarsTopBeam 6;		# number of longitudinal-reinforcement bars on top layer
	set numBarsBotBeam 6;		# number of longitudinal-reinforcement bars on bottom layer
	set numBarsIntBeam 2;		# TOTAL number of reinforcing bars on the intermediate layers
	set barAreaTopBeam [expr 1.*$in*$in];	# longitudinal-reinforcement bar area
	set barAreaBotBeam [expr 1.*$in*$in];	# longitudinal-reinforcement bar area
	set barAreaIntBeam [expr 1.*$in*$in];	# longitudinal-reinforcement bar area
	
	set numBarsTopGird 6;		# number of longitudinal-reinforcement bars on top layer
	set numBarsBotGird 6;		# number of longitudinal-reinforcement bars on bottom layer
	set numBarsIntGird 2;		# TOTAL number of reinforcing bars on the intermediate layers
	set barAreaTopGird [expr 1.*$in*$in];	# longitudinal-reinforcement bar area
	set barAreaBotGird [expr 1.*$in*$in];	# longitudinal-reinforcement bar area
	set barAreaIntGird [expr 1.*$in*$in];	# longitudinal-reinforcement bar area
	
	set nfCoreY 12;		# number of fibers in the core patch in the y direction
	set nfCoreZ 12;		# number of fibers in the core patch in the z direction
	set nfCoverY 8;		# number of fibers in the cover patches with long sides in the y direction
	set nfCoverZ 8;		# number of fibers in the cover patches with long sides in the z direction
	# rectangular section with one layer of steel evenly distributed around the perimeter and a confined core.
	BuildRCrectSection $ColSecTagFiber $HCol $BCol $cover $cover $IDconcCore  $IDconcCover $IDSteel $numBarsTopCol $barAreaTopCol $numBarsBotCol $barAreaBotCol $numBarsIntCol $barAreaIntCol  $nfCoreY $nfCoreZ $nfCoverY $nfCoverZ
	BuildRCrectSection $BeamSecTagFiber $HBeam $BBeam $cover $cover $IDconcCore  $IDconcCover $IDSteel $numBarsTopBeam $barAreaTopBeam $numBarsBotBeam $barAreaBotBeam $numBarsIntBeam $barAreaIntBeam  $nfCoreY $nfCoreZ $nfCoverY $nfCoverZ
	BuildRCrectSection $GirdSecTagFiber $HGird $BGird $cover $cover $IDconcCore  $IDconcCover $IDSteel $numBarsTopGird $barAreaTopGird $numBarsBotGird $barAreaBotGird $numBarsIntGird $barAreaIntGird  $nfCoreY $nfCoreZ $nfCoverY $nfCoverZ
	
	# assign torsional Stiffness for 3D Model
	uniaxialMaterial Elastic $SecTagTorsion $Ubig
	section Aggregator $ColSecTag $SecTagTorsion T -section $ColSecTagFiber
	section Aggregator $BeamSecTag $SecTagTorsion T -section $BeamSecTagFiber
	section Aggregator $GirdSecTag $SecTagTorsion T -section $GirdSecTagFiber
} else {
	puts "No section has been defined"
	return -1
}
set GammaConcrete [expr 150*$pcf];
set QdlCol [expr $GammaConcrete*$HCol*$BCol];	# self weight of Column, weight per length
set QBeam [expr $GammaConcrete*$HBeam*$BBeam];	# self weight of Beam, weight per length
set QGird [expr $GammaConcrete*$HGird*$BGird];	# self weight of Gird, weight per length

# --------------------------------------------------------------------------------------------------------------------------------
# define ELEMENTS
# set up geometric transformations of element
#   separate columns and beams, in case of P-Delta analysis for columns
set IDColTransf 1; # all columns
set IDBeamTransf 2; # all beams
set IDGirdTransf 3; # all girds
set ColTransfType Linear ;		# options for columns: Linear PDelta  Corotational 
geomTransf $ColTransfType  $IDColTransf  0 0 1;			# orientation of column stiffness affects bidirectional response.
geomTransf Linear $IDBeamTransf 0 0 1
geomTransf Linear $IDGirdTransf 1 0 0

# Define Beam-Column Elements
set numIntgrPts 5;	# number of Gauss integration points for nonlinear curvature distribution
# columns
set N0col [expr 10000-1];	# column element numbers
set level 0
for {set frame 1} {$frame <=[expr $NFrame]} {incr frame 1} {
	for {set level 1} {$level <=$NStory} {incr level 1} {
		for {set pier 1} {$pier <= [expr $NBay+1]} {incr pier 1} {
			set elemID [expr $N0col  +$level*$Dlevel + $frame*$Dframe+$pier]
			set nodeI [expr  $level*$Dlevel + $frame*$Dframe+$pier]
			set nodeJ  [expr  ($level+1)*$Dlevel + $frame*$Dframe+$pier]
			element nonlinearBeamColumn $elemID $nodeI $nodeJ $numIntgrPts $ColSecTag $IDColTransf;		# columns
		}
	}
}


# beams -- parallel to X-axis
set N0beam 1000000;	# beam element numbers
for {set frame 1} {$frame <=[expr $NFrame]} {incr frame 1} {
	for {set level 2} {$level <=[expr $NStory+1]} {incr level 1} {
		for {set bay 1} {$bay <= $NBay} {incr bay 1} {
			set elemID [expr $N0beam +$level*$Dlevel + $frame*$Dframe+ $bay]
			set nodeI [expr $level*$Dlevel + $frame*$Dframe+ $bay]
			set nodeJ  [expr $level*$Dlevel + $frame*$Dframe+ $bay+1]
			element nonlinearBeamColumn $elemID $nodeI $nodeJ $numIntgrPts $BeamSecTag $IDBeamTransf;	# beams
		}
	}
}

# girders -- parallel to Z-axis
set N0gird 2000000;	# gird element numbers
for {set frame 1} {$frame <=[expr $NFrame-1]} {incr frame 1} {
	for {set level 2} {$level <=[expr $NStory+1]} {incr level 1} {
		for {set bay 1} {$bay <= $NBay+1} {incr bay 1} {
			set elemID [expr $N0gird + $level*$Dlevel +$frame*$Dframe+ $bay]
			set nodeI [expr   $level*$Dlevel + $frame*$Dframe+ $bay]
			set nodeJ  [expr  $level*$Dlevel + ($frame+1)*$Dframe+ $bay]
			element nonlinearBeamColumn $elemID $nodeI $nodeJ $numIntgrPts $GirdSecTag $IDGirdTransf;		# Girds
		}
	}
}

# --------------------------------------------------------------------------------------------------------------------------------
# Define GRAVITY LOADS, weight and masses
# calculate dead load of frame, assume this to be an internal frame (do LL in a similar manner)
# calculate distributed weight along the beam length
#set GammaConcrete [expr 150*$pcf];   		# Reinforced-Concrete floor slabs, defined above
set Tslab [expr 6*$in];			# 6-inch slab
set Lslab [expr $LGird/2]; 			# slab extends a distance of $LGird/2 in/out of plane
set DLfactor 1.0;				# scale dead load up a little
set Qslab [expr $GammaConcrete*$Tslab*$Lslab*$DLfactor]; 
set QdlBeam [expr $Qslab + $QBeam]; 	# dead load distributed along beam (one-way slab)
set QdlGird $QGird; 			# dead load distributed along girder
set WeightCol [expr $QdlCol*$LCol];  		# total Column weight
set WeightBeam [expr $QdlBeam*$LBeam]; 	# total Beam weight
set WeightGird [expr $QdlGird*$LGird]; 	# total Beam weight

# assign masses to the nodes that the columns are connected to 
# each connection takes the mass of 1/2 of each element framing into it (mass=weight/$g)
set iFloorWeight ""
set WeightTotal 0.0
set sumWiHi 0.0;		# sum of storey weight times height, for lateral-load distribution

for {set frame 1} {$frame <=[expr $NFrame]} {incr frame 1} {
	if {$frame == 1 || $frame == $NFrame}  {
		set GirdWeightFact 1;		# 1x1/2girder on exterior frames
	} else {
		set GirdWeightFact 2;		# 2x1/2girder on interior frames
	}
	for {set level 2} {$level <=[expr $NStory+1]} {incr level 1} { ;		
		set FloorWeight 0.0
		if {$level == [expr $NStory+1]}  {
			set ColWeightFact 1;		# one column in top story
		} else {
			set ColWeightFact 2;		# two columns elsewhere
		}
		for {set pier 1} {$pier <= [expr $NBay+1]} {incr pier 1} {;
			if {$pier == 1 || $pier == [expr $NBay+1]} {
				set BeamWeightFact 1;	# one beam at exterior nodes
			} else {;
				set BeamWeightFact 2;	# two beams elewhere
			}
			set WeightNode [expr $ColWeightFact*$WeightCol/2 + $BeamWeightFact*$WeightBeam/2 + $GirdWeightFact*$WeightGird/2]
			set MassNode [expr $WeightNode/$g];
			set nodeID [expr $level*$Dlevel+$frame*$Dframe+$pier]
			mass $nodeID $MassNode 0. $MassNode 0. 0. 0.;			# define mass
			set FloorWeight [expr $FloorWeight+$WeightNode];
		}
		lappend iFloorWeight $FloorWeight
		set WeightTotal [expr $WeightTotal+ $FloorWeight]
		set sumWiHi [expr $sumWiHi+$FloorWeight*($level-1)*$LCol];		# sum of storey weight times height, for lateral-load distribution
	}
}
set MassTotal [expr $WeightTotal/$g];						# total mass

# --------------------------------------------------------------------------------------------------------------------------------
# LATERAL-LOAD distribution for static pushover analysis
# calculate distribution of lateral load based on mass/weight distributions along building height
# Fj = WjHj/sum(WiHi)  * Weight   at each floor j
set iFj ""
for {set level 2} {$level <=[expr $NStory+1]} {incr level 1} { ;	
	set FloorWeight [lindex $iFloorWeight [expr $level-1-1]];
	set FloorHeight [expr ($level-1)*$LCol];
	lappend iFj [expr $FloorWeight*$FloorHeight/$sumWiHi*$WeightTotal];		# per floor
}
set iNodePush $iMasterNode;		# nodes for pushover/cyclic, vectorized
set iFPush $iFj;				# lateral load for pushover, vectorized

# Define RECORDERS -------------------------------------------------------------
set FreeNodeID [expr $NFrame*$Dframe+($NStory+1)*$Dlevel+($NBay+1)];					# ID: free node
set SupportNodeFirst [lindex $iSupportNode 0];						# ID: first support node
set SupportNodeLast [lindex $iSupportNode [expr [llength $iSupportNode]-1]];			# ID: last support node
set FirstColumn [expr $N0col  + 1*$Dframe+1*$Dlevel +1];							# ID: first column
recorder Node -file $dataDir/DFree.out -time -node $FreeNodeID  -dof 1 2 3 disp;				# displacements of free node
recorder Node -file $dataDir/DBase.out -time -nodeRange $SupportNodeFirst $SupportNodeLast -dof 1 2 3 disp;	# displacements of support nodes
recorder Node -file $dataDir/RBase.out -time -nodeRange $SupportNodeFirst $SupportNodeLast -dof 1 2 3 reaction;	# support reaction
recorder Drift -file $dataDir/DrNode.out -time -iNode $SupportNodeFirst  -jNode $FreeNodeID   -dof 1 -perpDirn 2;	# lateral drift
recorder Element -file $dataDir/Fel1.out -time -ele $FirstColumn localForce;					# element forces in local coordinates
recorder Element -file $dataDir/ForceEle1sec1.out -time -ele $FirstColumn section 1 force;			# section forces, axial and moment, node i
recorder Element -file $dataDir/DefoEle1sec1.out -time -ele $FirstColumn section 1 deformation;			# section deformations, axial and curvature, node i
recorder Element -file $dataDir/ForceEle1sec$numIntgrPts.out -time -ele $FirstColumn section $numIntgrPts force;			# section forces, axial and moment, node j
recorder Element -file $dataDir/DefoEle1sec$numIntgrPts.out -time -ele $FirstColumn section $numIntgrPts deformation;		# section deformations, axial and curvature, node j
set yFiber [expr $HCol/2-$cover];								# fiber location for stress-strain recorder, local coords
set zFiber [expr $BCol/2-$cover];								# fiber location for stress-strain recorder, local coords
recorder Element -file $dataDir/SSconcEle1sec1.out -time -ele $FirstColumn section $numIntgrPts fiber $yFiber $zFiber $IDconcCore  stressStrain;	# steel fiber stress-strain, node i
recorder Element -file $dataDir/SSreinfEle1sec1.out -time -ele $FirstColumn section $numIntgrPts fiber $yFiber $zFiber $IDSteel  stressStrain;	# steel fiber stress-strain, node i

# Define DISPLAY -------------------------------------------------------------
DisplayModel3D DeformedShape ;	 # options: DeformedShape NodeNumbers ModeShape

# GRAVITY -------------------------------------------------------------
# define GRAVITY load applied to beams and columns -- eleLoad applies loads in local coordinate axis
pattern Plain 101 Linear {
	for {set frame 1} {$frame <=[expr $NFrame]} {incr frame 1} {
		for {set level 1} {$level <=$NStory} {incr level 1} {
			for {set pier 1} {$pier <= [expr $NBay+1]} {incr pier 1} {
				set elemID [expr $N0col  + $level*$Dlevel +$frame*$Dframe+$pier]
				eleLoad -ele $elemID -type -beamUniform 0. 0. -$QdlCol; 	# COLUMNS		}
		}
	}
	for {set frame 1} {$frame <=[expr $NFrame]} {incr frame 1} {
		for {set level 2} {$level <=[expr $NStory+1]} {incr level 1} {
			for {set bay 1} {$bay <= $NBay} {incr bay 1} {
				set elemID [expr $N0beam + $level*$Dlevel +$frame*$Dframe+ $bay]
				eleLoad -ele $elemID  -type -beamUniform -$QdlBeam 0.; 	# BEAMS
			}
		}
	}
	for {set frame 1} {$frame <=[expr $NFrame-1]} {incr frame 1} {
		for {set level 2} {$level <=[expr $NStory+1]} {incr level 1} {
			for {set bay 1} {$bay <= $NBay+1} {incr bay 1} {
				set elemID [expr $N0gird + $level*$Dlevel +$frame*$Dframe+ $bay]
				eleLoad -ele $elemID  -type -beamUniform -$QdlGird 0.;	# GIRDS
			}
		}
	}
	
}
puts goGravity
# Gravity-analysis parameters -- load-controlled static analysis
set Tol 1.0e-8;			# convergence tolerance for test
variable constraintsTypeGravity Plain;		# default;
if {  [info exists RigidDiaphragm] == 1} {
	if {$RigidDiaphragm=="ON"} {
		variable constraintsTypeGravity Lagrange;	#  large model: try Transformation
	};	# if rigid diaphragm is on
};	# if rigid diaphragm exists
constraints $constraintsTypeGravity ;     		# how it handles boundary conditions
numberer RCM;			# renumber dof's to minimize band-width (optimization), if you want to
system BandGeneral ;		# how to store and solve the system of equations in the analysis (large model: try UmfPack)
test EnergyIncr $Tol 6 ; 		# determine if convergence has been achieved at the end of an iteration step
algorithm Newton;			# use Newton's solution algorithm: updates tangent stiffness at every iteration
set NstepGravity 10;  		# apply gravity in 10 steps
set DGravity [expr 1./$NstepGravity]; 	# first load increment;
integrator LoadControl $DGravity;	# determine the next time step for an analysis
analysis Static;			# define type of analysis static or transient
analyze $NstepGravity;		# apply gravity

# ------------------------------------------------- maintain constant gravity loads and reset time to zero
loadConst -time 0.0

# -------------------------------------------------------------
puts "Model Built"
\end{sourcecode}

\begin{sourcecode}[]{cuda}{Un poco de cuda.}
__global__ void foo(){
}

__global__ void addKernel(int *c, const int *a, const int *b){
	int i = threadIdx.x;
	c[i] = a[i] + b[i];
}

foo<<<n,m>>>();
\end{sourcecode}

\begin{sourcecode}[]{bash}{Un poco de bash.}
# Muestra toda la información de la batería
function battr-info {
	wrks-scripts
	data=$(ioreg -l -w0 |grep Capacity)
	python2.7 battery_info.py $data
	cd - >> config/.empty
}

function qtest-java {
	if [ -z "${1}" ]; then
	echo-err 'Nombre fuente no definido'
	else
	vim $1
	javac -encoding ISO-8859-1 $1 $2 $3
	blankspace=""
	first=$1
	first=${first/.java/$blankspace}
	java $first
	fi
}

PATH=$PATH:"/Library/Frameworks/Python.framework/Versions/2.7/bin:${PATH}"
PATH=$PATH:"/Applications/Utilities/Lynxlet.app/Contents/Resources/lynx/bin"
export PATH

alias ga='git add '
dig +short myip.opendns.com @resolver1.opendns.com
gcc "$@" -o $first

::claramente esto no funcionará::
sudo x, call y
rem esto es un comentario en windows
history -c rm a, ls -d killall e mv | grep | awk python 'hola'
vim uwu printf 'have you seen him' -z
git doge ssh cp cd
\end{sourcecode}

\clearpage
\begin{sourcecode}[]{c}{Ejemplo en C.}
#include <stdio.h>
int main(){
	int i, j, rows;
	
	printf("Enter number of rows: ");
	scanf("%d",&rows);
	
	for(i=1; i<=rows; ++i){
		for(j=1; j<=i; ++j){
			printf("* ");
		}
		printf("\n");
	}
	return 0;
}
\end{sourcecode}

\begin{sourcecode}[]{csharp}{Ejemplo en C\#.}
/*
* C# Program to Get a Number and Display the Sum of the Digits 
*/
using System;
using System.Collections.Generic;
using System.Linq;
using System.Text;

namespace Program
{
	class Program
	{
		static void Main(string[] args)
		{
			int num, sum = 0, r;
			Console.WriteLine("Enter a Number : ");
			num = int.Parse(Console.ReadLine());
			while (num != 0)
			{
				r = num % 10;
				num = num / 10;
				sum = sum + r;
			}
			Console.WriteLine("Sum of Digits of the Number : "+sum);
			Console.ReadLine();
			
		}
	}
}
\end{sourcecode}

\clearpage
\begin{sourcecode}{cpp}{Suma en C++.}
#include <iostream>
using namespace std;

int main()
{
	int n, sum = 0;
	
	cout << "Enter a positive integer: ";
	cin >> n;
	
	for (int i = 1; i <= n; ++i) {
		sum += i;
	}
	
	cout << "Sum = " << sum;
	return 0;
}
\end{sourcecode}

\begin{sourcecode}{docker}{Docker.}
version: '2'
services:
web:
build: .
ports:
- "5000:5000"
volumes:
- .:/code
- logvolume01:/var/log
links:
- redis
redis:
image: redis
volumes:
logvolume01: {}
\end{sourcecode}

\begin{sourcecode}{plaintext}{Resultado del análiis con TEFAME.}
TEFAME - Toolbox para Elemento Finitos y Analisis
Matricial de Estructuras en MATLAB

Propiedades de entrada modelo:

Nodos: 
Numero de nodos: 4 
Coordenadas nodo N1: 0 0
Coordenadas nodo N2: 800 0
Coordenadas nodo N3: 400 400
Coordenadas nodo N4: 400 800

Elementos: 
Numero de elementos: 6 
Elemento E1:	Largo: 800         Area: 20        Eo: 200000    
Elemento E2:	Largo: 565.6854    Area: 20        Eo: 200000    
Elemento E3:	Largo: 565.6854    Area: 20        Eo: 200000    
Elemento E4:	Largo: 894.4272    Area: 20        Eo: 200000    
Elemento E5:	Largo: 400         Area: 20        Eo: 200000    
Elemento E6:	Largo: 894.4272    Area: 20        Eo: 200000    

Resultados del analisis:

Desplazamientos nodos: 
Desplazamientos nodo N1: 0 0
Desplazamientos nodo N2: 0.016 0
Desplazamientos nodo N3: 0.008 -0.013
Desplazamientos nodo N4: 0.053 -0.016

Reacciones: 
Reacciones nodo N1: -80 -20
Reacciones nodo N2: 0 140
Reacciones nodo N3: 0 0
Reacciones nodo N4: 0 0

Esfuerzos Elementos: 
Elemento E1: -78.4273       TRACCION
Elemento E2: 23.836         COMPRESION
Elemento E3: 23.836         COMPRESION
Elemento E4: -41.2047       TRACCION
Elemento E5: 33.7093        COMPRESION
Elemento E6: 137.6807       COMPRESION
\end{sourcecode}

\begin{sourcecode}[\label{codigo-html5}]{html5}{Ejemplo en HTML5.}
<!DOCTYPE html>
<html>
<head>
	<title>Página</title>
</head>
<body>
	<style>
		.titulo {
			color: #ff0000;
		}
	</style>
	<div class="titulo">Hola</div>
</body>
</html>
\end{sourcecode}

\begin{sourcecode}[\label{codigo-matlab}]{matlab}{Ejemplo en Matlab.}
% Se crea gráfico
f = figure(1);
hold on;
movegui(f, 'center');
xlabel('td/Tn'); ylabel('FAD=Umax/Uf0');
title('Espectro de pulso de desplazamiento');

for j = 1:length(BETA)
	fad = ones(1, NDATOS); % Arreglo para el FAD, uno para cada r
	for i = 1:NDATOS
		[t, u_t, ~, ~] = main(BETA(j), r(i), M, K, F0, 0);
		fad(i) = max(abs(u_t)) / uf0;
	end
mx = find(fad == max(fad(:)));
fprintf('BETA=%.2f, MAX: FAD=%.3f, TD/TN=%.3f\n', BETA(j), fad(mx), tdtn(mx));
plot(tdtn, fad, 'DisplayName', strcat('\beta=', sprintf('%.2f', BETA(j))));
end	
\end{sourcecode}

\begin{sourcecode}[]{xml}{Ejemplo xml.}
<?xml version="1.0" encoding="utf-8"?>
<xs:schema attributeFormDefault="unqualified" elementFormDefault="qualified"
xmlns:xs="http://www.w3.org/2001/XMLSchema">
	<xs:element name="points">
		<xs:complexType>
			<xs:sequence>
				<xs:element maxOccurs="unbounded" name="point">
					<xs:complexType>
						<xs:attribute name="x" type="xs:unsignedShort" use="required" />
						<xs:attribute name="y" type="xs:unsignedShort" use="required" />
					</xs:complexType>
				</xs:element>
			</xs:sequence>
		</xs:complexType>
	</xs:element>
</xs:schema>
\end{sourcecode}

\begin{sourcecode}[\label{codigo-python}]{python}{Ejemplo en Python.}
import numpy as np

def incmatrix(genl1, genl2):
m = len(genl1)
n = len(genl2)
M = None # Comentario 1
VT = np.zeros((n*m, 1), int) # Comentario 2
\end{sourcecode}

\begin{sourcecode}[\label{codigo-java}]{java}{Ejemplo en Java.}
import java.io.IOException; 
import javax.servlet.*;

// Hola mundo
public class Hola extends GenericServlet {
	public void service(ServletRequest request, ServletResponse response)
	throws ServletException, IOException{
		response.setContentType("text/html");
		PrintWriter pw = response.getWriter();
		pw.println("Hola, mundo!");
		pw.close();
	}
}
\end{sourcecode}

\begin{sourcecode}{js}{Ejemplo en Javascript.}
$.urlParam = function (name) {
	let results = new RegExp('[\?&]' + name + '=([^&#]*)').exec(window.location.href);
	if (results == null) {
		return null;
	} else {
		return decodeURI(results[1]) || 0;
	}
};
\end{sourcecode}

\begin{sourcecode}{json}{Un arreglo en JSON.}
{"menu": {
	"id": "file",
	"value": "File",
	"popup": {
		"menuitem": [
		{"value": "New", "onclick": "CreateNewDoc()"},
		{"value": "Open", "onclick": "OpenDoc()"},
		{"value": "Close", "onclick": "CloseDoc()"}
		]
	}
}}
\end{sourcecode}

\begin{sourcecode}{latex}{Imágenes múltiples.}
\begin{images}[\label{imagenmultiple}]{Ejemplo de imagen múltiple.}
	\addimage{ejemplos/test-image}{width=6.5cm}{Ciudad}
	\addimage{ejemplos/test-image-wrap}{width=5cm}{Apolo}
	\addimage{ejemplos/test-image}{width=12cm}{Ciudad más grande}
\end{images}
\end{sourcecode}

\clearpage
\begin{sourcecode}[\label{ejemplito-perl}]{perl}{Algo de perl.}
#!/usr/bin/perl
use strict;
use warnings;

# first, create your message
use Email::MIME;
my $message = Email::MIME->create(
  header_str => [
    From    => 'you@example.com',
    To      => 'friend@example.com',
    Subject => 'Happy birthday!',
  ],
  attributes => {
    encoding => 'quoted-printable',
    charset  => 'ISO-8859-1',
  },
  body_str => "Happy birthday to you!\n",
);

# send the message
use Email::Sender::Simple qw(sendmail);
sendmail($message);
\end{sourcecode}

\clearpage
\begin{sourcecode}{php}{Ejemplo php.}
<?php
$target_dir = "uploads/";
$target_file = $target_dir . basename($_FILES["fileToUpload"]["name"]);
$uploadOk = 1;
$imageFileType = strtolower(pathinfo($target_file,PATHINFO_EXTENSION));
// Check if image file is a actual image or fake image
if(isset($_POST["submit"])) {
    $check = getimagesize($_FILES["fileToUpload"]["tmp_name"]);
    if($check !== false) {
        echo "File is an image - " . $check["mime"] . ".";
        $uploadOk = 1;
    } else {
        echo "File is not an image.";
        $uploadOk = 0;
    }
}
?>
\end{sourcecode}

\clearpage
\begin{sourcecode}[]{ruby}{Ejemplo con ruby.}
class DataFile < ActiveRecord::Base
    attr_accessor :upload

  def self.save_file(upload)   

    file_name = upload['datafile'].original_filename  if  (upload['datafile'] !='')    
    file = upload['datafile'].read    

    file_type = file_name.split('.').last
    new_name_file = Time.now.to_i
    name_folder = new_name_file
    new_file_name_with_type = "#{new_name_file}." + file_type

    image_root = "#{RAILS_CAR_IMAGES}"


    Dir.mkdir(image_root + "#{name_folder}");
      File.open(image_root + "#{name_folder}/" + new_file_name_with_type, "wb")  do |f|  
        f.write(file) 
      end

  end
end
\end{sourcecode}

\clearpage
\begin{sourcecode}{sql}{Merge two tables.}
SELECT ChargeNum, CategoryID, SUM(Hours)
FROM KnownHours
GROUP BY ChargeNum, CategoryID
UNION ALL
SELECT ChargeNum, 'Unknown' AS CategoryID, SUM(Hours)
FROM UnknownHours
GROUP BY ChargeNum
\end{sourcecode}

\begin{sourcecode}{scala}{Código en scala.}
object Test {
	def main(args: Array[String]) {
		var a = 0;
		// for loop execution with a range
		for( a <- 1 to 10){
			println( "Value of a: " + a );
		}
	}
}
\end{sourcecode}

\begin{sourcecode}{kotlin}{Kotlin en acción.}
/* Block comment */
package hello
import kotlin.collections.* // line comment

/**
* Doc comment here for `SomeClass`
* @see Iterator#next()
*/
@Deprecated("Deprecated class")
private class MyClass<out T : Iterable<T>>(var prop1 : Int) {
	fun foo(nullable : String?, r : Runnable, f : () -> Int, 
	fl : FunctionLike, dyn: dynamic) {
		println("length\nis ${nullable?.length} \e")
		val ints = java.util.ArrayList<Int?>(2)
		ints[0] = 102 + f() + fl()
		val myFun = { -> "" };
		var ref = ints.size
		ints.lastIndex + globalCounter
		ints.forEach lit@ {
			if (it == null) return@lit
			println(it + ref)
		}
		dyn.dynamicCall()
		dyn.dynamicProp = 5
	}
	
	val test = """
		hello
		world
		kotlin
	"""
	override fun hashCode(): Int {
		return super.hashCode() * 31
	}
}
fun Int?.bar() {
	if (this != null) {
		println(message = toString())
	}
	else {
		println(this.toString())
	}
}
var globalCounter : Int = 5
get = field
abstract class Abstract {
}
object Obj
enum class E { A, B }
interface FunctionLike {
	operator fun invoke() = 1
}
\end{sourcecode}

\begin{sourcecodep}{xml}{firstnumber=100, basicstyle={\fontsize{7}{10}\selectfont\ttfamily}}{}
<?xml version="1.0" encoding="utf-8"?>
<xs:schema attributeFormDefault="unqualified" elementFormDefault="qualified"
   xmlns:xs="http://www.w3.org/2001/XMLSchema">
  <xs:element name="points">
    <xs:complexType>
      <xs:sequence>
        <xs:element maxOccurs="unbounded" name="point">
          <xs:complexType>
            <xs:attribute name="x" type="xs:unsignedShort" use="required" />
            <xs:attribute name="y" type="xs:unsignedShort" use="required" />
          </xs:complexType>
        </xs:element>
      </xs:sequence>
    </xs:complexType>
  </xs:element>
</xs:schema>
\end{sourcecodep}

\begin{sourcecode}{pseudocode}{Algoritmo.}
input: int N, int D % Comentario
output: int // Comentario 2
begin # key
	LET $\gets$ 0 /* comentario 1 */
	while N $\geq$ D /** comentario 2 */
		N $\gets$ N - D
		res $\gets$ res + 1      
	end
	return res
end    
\end{sourcecode}

\clearpage
\begin{sourcecode}{glsl}{Noise shader.}
#ifdef GL_ES
precision mediump float;
#endif

uniform vec2 u_resolution;
uniform vec2 u_mouse;
uniform float u_time;

// 2D Random
float random (in vec2 st) {
	return fract(sin(dot(st.xy,
	vec2(12.9898,78.233)))
	* 43758.5453123);
}

// 2D Noise based on Morgan McGuire @morgan3d
// https://www.shadertoy.com/view/4dS3Wd
float noise (in vec2 st) {
	vec2 i = floor(st);
	vec2 f = fract(st);
	
	// Four corners in 2D of a tile
	float a = random(i);
	float b = random(i + vec2(1.0, 0.0));
	float c = random(i + vec2(0.0, 1.0));
	float d = random(i + vec2(1.0, 1.0));
	
	// Smooth Interpolation
	
	// Cubic Hermine Curve.  Same as SmoothStep()
	vec2 u = f*f*(3.0-2.0*f);
	// u = smoothstep(0.,1.,f);
	
	// Mix 4 coorners percentages
	return mix(a, b, u.x) +
	(c - a)* u.y * (1.0 - u.x) +
	(d - b) * u.x * u.y;
}

void main() {
	vec2 st = gl_FragCoord.xy/u_resolution.xy;
	
	// Scale the coordinate system to see
	// some noise in action
	vec2 pos = vec2(st*5.0);
	
	// Use the noise function
	float n = noise(pos);
	
	gl_FragColor = vec4(vec3(n), 1.0);
}
\end{sourcecode}

\begin{sourcecode}[]{matlab}{Ejemplo en Matlab.}
% Se crea gráfico
%% hola
f = figure(1);
hold on;
movegui(f, 'center');
xlabel('td/Tn'); ylabel('FAD=Umax/Uf0');
title('Espectro de pulso de desplazamiento');

for j = 1:length(BETA)
	fad = ones(1, NDATOS); % Arreglo para el FAD, uno para cada r
	for i = 1:NDATOS
		[t, u_t, ~, ~] = main(BETA(j), r(i), M, K, F0, 0);
		fad(i) = max(abs(u_t)) / uf0;
	end
mx = find(fad == max(fad(:)));
fprintf('BETA=%.2f, MAX: FAD=%.3f, TD/TN=%.3f\n', BETA(j), fad(mx), tdtn(mx));
plot(tdtn, fad, 'DisplayName', strcat('\beta=', sprintf('%.2f', BETA(j))));
end	
\end{sourcecode}

\begin{sourcecode}{r}{Ejemplo en r.}
#
# Genera una correlacion de los generos con todas sus variables
#
corr_data_genre <- function(data, genre) {
	num_genre <- data[data$genre == genre, -c(1:4)]
	corr_genre <- cor(num_genre)
	corrplot(corr_genre, type = "lower", order = "hclust", tl.col = "black", tl.srt = 30, main = toupper(genre), tl.cex = 0.7, mar=c(0,0,1,0))
}

#
# Crea matriz de generos
#
create_track_genre_df <- function(df, title) {
	# Obtiene la lista de generos ordenados
	genre <- data.frame(genre=unique(df$genre))
	track_ids <- unique(df$track_id)
	df_genre <- df[, -c(2:3, 5:15)]
	df_genre <- df_genre[order(df_genre$track_id), ]
	track_genres <- data.frame(track_id=track_ids)
	
	# Crea un data frame indicando para cada tema si tiene cada genero o no
	df_full <- data.frame(track_id=df_genre$track_id,
	'_'=df_genre$genre==genre[1, 1],
	'Opera'=df_genre$genre==genre[1, 1],
	'A_Capella'=df_genre$genre==genre[2, 1],
	'Alternative'=df_genre$genre==genre[3, 1],
	'Blues'=df_genre$genre==genre[4, 1],
	'Dance'=df_genre$genre==genre[5, 1],
	'Pop'=df_genre$genre==genre[6, 1],
	'Electronic'=df_genre$genre==genre[7, 1],
	'RyB'=df_genre$genre==genre[8, 1],
	'Children_Music'=df_genre$genre==genre[9, 1],
	'Folk'=df_genre$genre==genre[10, 1],
	'Anime'=df_genre$genre==genre[11, 1],
	'Rap'=df_genre$genre==genre[12, 1],
	'Classical'=df_genre$genre==genre[13, 1],
	'Reggae'=df_genre$genre==genre[14, 1],
	'Hip_Hop'=df_genre$genre==genre[15, 1],
	'Comedy'=df_genre$genre==genre[16, 1],
	'Country'=df_genre$genre==genre[17, 1],
	'Reggaeton'=df_genre$genre==genre[18, 1],
	'Ska'=df_genre$genre==genre[19, 1],
	'Indie'=df_genre$genre==genre[20, 1],
	'Rock'=df_genre$genre==genre[21, 1],
	'Soul'=df_genre$genre==genre[22, 1],
	'Soundtrack'=df_genre$genre==genre[23, 1],
	'Jazz'=df_genre$genre==genre[24, 1],
	'World'=df_genre$genre==genre[25, 1],
	'Movie'=df_genre$genre==genre[26, 1])
	cols <- sapply(df_full, is.logical)
	df_full[,cols] <- lapply(df_full[,cols], as.numeric)
	
	# Fusiona
	track_genres_df <- aggregate(.~track_id, df_full[,-c(2, 27)], sum)
	
	# Crea un grafico de correlacion
	track_genres_df_noid <- track_genres_df[, -c(1)]
	corr_genre <- cor(track_genres_df_noid)
	corrplot(corr_genre, type = "lower", order = "hclust", tl.col = "black", tl.srt = 30, main = title, tl.cex = 0.7, mar=c(0,0,1,0))
	
	return(track_genres_df)
}

# Crea histograma con todas las variables del dataset
create_hist_plot <- function(df, probability) {
	select_plot <- function(data, tag){
		hist(data,
		col="#00AFBB",
		border="#015f66",
		ylab=NULL,
		xlab=NULL,
		prob=probability,
		main=toupper(tag),
		breaks=50
		)
	}
	par(mfrow=c(3,4))
	select_plot(df$popularity, "popularity")
	select_plot(df$acousticness, "acousticness")
	select_plot(datos_num$danceability, "danceability")
	select_plot(datos_num$duration_ms, "duration_ms")
	select_plot(datos_num$energy, "energy")
	if (probability){
		title(ylab="Probability")
	}else{
		title(ylab="Frequence") 
	}
	select_plot(datos_num$instrumentalness, "instrumentalness")
	select_plot(datos_num$liveness, "liveness")
	select_plot(datos_num$loudness, "loudness")
	select_plot(datos_num$speechiness, "speechiness")
	select_plot(datos_num$tempo, "tempo")
	select_plot(datos_num$valence, "valence")
}

#
# Genera un histograma de varios generos con una variable
#
create_hist_genre <- function(df, genres, var){
	gi <- 0
	colors <- c()
	genr <- c()
	for (g in genres) {
		df_g <- df[df$genre == g, var] # elige el data
		color <- rgb(runif(1), runif(1), runif(1), 0.5)
		if (gi == 0) {
			hist(df_g[[1]], col=color, xlab="", ylab="", main="", border=color, breaks=50)
		} else {
			hist(df_g[[1]], col=color, border=color, breaks=50, add=T, xlab=var)
		}
		colors <- c(colors, color)
		gi <- gi + 1
		genr <- c(genr, g)
	}
	title(main=toupper(var), sub="", xlab=var, ylab="Frecuency")
	op <- par(cex = 0.75)
	legend('topleft', genr, fill = colors, ncol=2)
}

#
# Genera un histograma de varios generos con una variable
#
create_prob_genre <- function(df, genres, var, pos, ncols, fsize){
	if (pos == "l") {
		legp <- "topleft"
	} else if(pos == "c") {
		legp <- "top"
	} else if(pos == "r") {
		legp <- "topright"
	} else if (pos == "bl") {
		legp <- "bottomleft"
	} else if(pos == "bc") {
		legp <- "bottom"
	} else if(pos == "br") {
		legp <- "bottomright"
	} else {
		legp <- "center"
	}
	gi <- 0
	tranp <- rgb(0, 0, 0, 0) # color transparente
	colors <- c()
	genr <- c()
	for (g in genres) {
		df_g <- df[df$genre == g, var] # elige el data
		color <- rgb(runif(1), runif(1), runif(1), 1)
		if (gi == 0) {
			hist(df_g[[1]], col=tranp, xlab="", ylab="", main="", border=tranp, prob = TRUE, breaks=50)
			lines(density(df_g[[1]]), col=color, xlab="", ylab="", main="", lw=2)
		} else {
			hist(df_g[[1]], col=tranp, border=tranp, breaks=50, prob = TRUE, add=T, xlab=var)
			lines(density(df_g[[1]]), col=color, lw=2)
		}
		colors <- c(colors, color)
		gi <- gi + 1
		genr <- c(genr, g)
	}
	title(main=toupper(var), sub="", xlab=var, ylab="Density")
	grid()
	op <- par(cex = fsize)
	legend(legp, genr, fill = colors, ncol=ncols)
}

# Tamaño normal
create_prob_genre_normal <- function(df, genres, var, pos) {
	create_prob_genre(df, genres, var, pos, 2, 0.75)
}

# Tamaño pequeño
create_prob_genre_small <- function(df, genres, var, pos) {
	create_prob_genre(df, genres, var, pos, 4, 0.6)
}

# Tamaño grande
create_prob_genre_large <- function(df, genres, var, pos) {
	create_prob_genre(df, genres, var, pos, 1, 1)
}
\end{sourcecode}

\begin{sourcecode}{css}{Código CSS.}
.fecha-estilo {
	color: #819198;
	font-size: 15px;
	position: relative;
	bottom: 0;
}

.btn {
	-moz-box-shadow: 0 0 5px 0 rgba(0, 0, 0, 0.75);
	-webkit-box-shadow: 0 0 5px 0 rgba(0, 0, 0, 0.75);
	background-color: rgba(100, 100, 100, 0.4);
	border-radius: 0.3rem;
	border: 1px solid rgba(255, 255, 255, 0.2);
	box-shadow: 0 0 5px 0 rgba(0, 0, 0, 0.75);
	color: rgba(255, 255, 255, 1);
	display: inline-block;
	margin-bottom: 1rem;
	margin-left: 0;
	margin-right: 0.5rem;
	opacity: 1.0;
	outline: none;
	transition: color 0.2s, background-color 0.2s, border-color 0.2s;
}

.btn:hover {
	opacity: 1.0;
}

.btn-pro {
	-moz-box-shadow: 0 0 5px 0 rgba(10, 10, 10, 0.75);
	-webkit-box-shadow: 0 0 5px 0 rgba(10, 10, 10, 0.75);
	background-color: rgba(210, 210, 210, 0.85);
	border-radius: 0.3rem;
	border: 1px solid rgba(255, 255, 255, 0.5);
	box-shadow: 0 0 5px 0 rgba(10, 10, 10, 0.75);
	color: rgba(20, 20, 20, 1);
	display: inline-block;
	font-weight: bolder;
	margin-bottom: 1rem;
	margin-left: 0;
	margin-right: 0.5rem;
	opacity: 0.9;
	outline: none;
	transition: color 0.2s, background-color 0.2s, border-color 0.2s;
}
\end{sourcecode}

\begin{sourcecode}{haskell}{Ejemplo en Haskell.}
-- Type annotation (optional)
fib :: Int -> Integer

-- With self-referencing data
fib n = fibs !! n
where fibs = 0 : scanl (+) 1 fibs
-- 0,1,1,2,3,5,...

-- Same, coded directly
fib n = fibs !! n
where fibs = 0 : 1 : next fibs
next (a : t@(b:_)) = (a+b) : next t

-- Similar idea, using zipWith
fib n = fibs !! n
where fibs = 0 : 1 : zipWith (+) fibs (tail fibs)

-- Using a generator function
fib n = fibs (0,1) !! n
where fibs (a,b) = a : fibs (b,a+b)
\end{sourcecode}

\begin{sourcecode}{pascal}{Ejemplo pascal.}
PROGRAM NotasDeAlumnos;
uses crt;
Type
vecalumnos = array [1..40] of string;
var
Nombre, Apellido: vecalumnos;
Nota: array [1..40] of real;
Begin
clrscr; /*Limpia pantalla*/
For i:= 1 to 40 do
	begin
		write('Ingrese Nombre: ');
		readln(Nombre[i]);
		write('Ingrese Apellido: ');
		readln(Apellido[i]);
		write('Ingrese Nota: ');
		readln(Nota[i]);
	end;
For i:= 1 to 40 do
	begin
		write(Nombre[i], ' ',Apellido[i]);
		if (Nota[i] >=7) then
			writeln(' aprobó')
		else
			writeln(' no aprobó');
	end;
writeln('');
Write ('Pulse [Intro] para finalizar...');
Readln
end.
\end{sourcecode}

\begin{sourcecode}{fortran}{Ejemplo en Fortran.}
REGRESION LINEAL.FORTRANS
APLICACION
     DIMENSION TIEMPO(1000),PROD(1000)
        OPEN(1,FILE='HISTORIA.txt')
        I=0
 10     READ(1,*,END=80)T,P
        I=I+1
        TIEMPO(.l.)=T
        PROD(I)=P
        GO TO 10
 80     NDATOS=I
        CALL AJULIN(TIEMPO,PROD,NDATOS,A,B)
        WRITE(*,90)A,B
 90     FORMAT('LA ECUACION ES:Y=',F10.2,'+',F10.2,'X')
 20     FORMAT(20F10.0)
        END


        SUBROUTINE AJULIN(X,Y,N,A,B)
         DIMENSION X(1),Y(1)
          SUMX=0.
          SUMY=0.
          SUMX2=0.
          SUMY2=0
          SUMXY=0
        DO 20 I=1,N
         SUMX=SUMX+X(.l.)
         SUMY=SUMY+Y(.l.)
         SUMX2=SUMX2+(X(I)*X(.l.))
         SUMY2=SUMY2+Y(I)**2
         SUMXY=SUMXY+(X(I)*Y(I))
 20    CONTINUE
        PROD=SUMX*SUMY
        B=(SUMXY-PROD/N)/(SUMX2-SUMX**2/N)
        A=(SUMY/N-B*SUMX/N)
       RETURN
       END
\end{sourcecode}

\begin{sourcecode}{octave}{Ejemplo en Octave.}
function out = invar_table(n,m,N)
if n>m, error('first number must be smaller than the second'), endif

t = cputime;
out = zeros(m-n+1,2);

for i=n:m
  out(i+1-n,1) = i;
  out(i+1-n,2) = invar(braidmatrix(i),N);
end

printf('Total CPU time: %f seconds\n', cputime-t);
end
\end{sourcecode}

\clearpage
\begin{sourcecode}{vbscript}{Ejemplo en Visual Basic.}
Option Explicit

Sub Signal(strSignalfolge As String, Optional lngTakt As Long = 100)
'Prozedur erzeugt eine Serie von Warntönen.
'Die optionale Variable lngTakt gibt den Takt in Millisekunden vor (Standard: 100 ms)
'Das Muster kann über die String-Variable strSignalfolge beeinflusst werden:
' Stern (*)     -> 1 Warnton
' Ziffern 1..9  -> 1..9 Takte Pause
' Leerzeichen   -> 1 Sekunde Pause
' Minus (-)     -> 1.5 Sekunden Pause
    Dim i As Integer
    Dim b As String
    For i = 1 To Len(strSignalfolge)
        b = Mid(strSignalfolge, i, 1)
        Select Case b
            Case "*": beep
            Case 1 To 9: DELAY CInt(b) * lngTakt
            Case " ": DELAY 1000
            Case "-": DELAY 1500
        End Select
        DELAY lngTakt
    Next i

End Sub

Public Function strParse(Data As String, Trenn As String, Nr As Integer)
'Funktion trennt die Zeichenkette <Data>
    On Error Resume Next
    Dim MainData() As String, SplitData() As String
    MainData = Split(Data, Trenn)
    SplitData = Split(MainData(Nr - 1), Trenn)
    strParse = SplitData(0)
End Function
\end{sourcecode}

\begin{sourcecode}{lisp}{Ejemplo en Lisp.}
(defun find-symbol-between-sigma-sets (sigma-a sigma-b scan-node symbols g)
	(let* ((scan-node-set-in-sigma-a (find-scan-node sigma-a scan-node))
		   (next-scan-node-set-in-sigma-b 
			(get-next-node-set-from-scan-node-set 
			 scan-node-set-in-sigma-a sigma-b g))
		   (union-node-set-in-sigma-a 
			(get-previous-nodes next-scan-node-set-in-sigma-b g)))
	  (if 
	   (and 
		(memberp union-node-set-in-sigma-a scan-node-set-in-sigma-a)
		(path-exists-between-set-a-and-set-b 
		 union-node-set-in-sigma-a next-scan-node-set-in-sigma-b g))
	   (car 
		(unique-get-symbols 
		 union-node-set-in-sigma-a next-scan-node-set-in-sigma-b symbols))
	   nil)))
\end{sourcecode}

\begin{sourcecode}{scheme}{Ejemplo en Scheme.}
;;; If the next character on p is a letter, get-word reads a word
;;; from p and returns it in a string.  If the character is not a
;;; letter, get-word returns the character (on eof, the eof-object).
(define get-word
	(lambda (p)
	(let ((c (read-char p)))
		(if (eq? (char-type c) 'letter)
			(list->string
			(let loop ((c c))
				(cons c
				(if (memq (char-type (peek-char p)) '(letter digit))
					(loop (read-char p))
					'()))))
			c))))

;;; char-type tests for the eof-object first, since the eof-object
;;; may not be a valid argument to char-alphabetic? or char-numeric?
;;; It returns the eof-object, the symbol letter, the symbol digit,
;;; or the argument itself if it is not a letter or digit.
(define char-type
	(lambda (c)
	(cond
		((eof-object? c) c)
		((char-alphabetic? c) 'letter)
		((char-numeric? c) 'digit)
		(else c))))

;;; Trees are represented as vectors with four fields: word, left,
;;; right, and count.  Only one field, word, is initialized by an
;;; argument to the constructor procedure make-tree.  The remaining
;;; fields are explicitly initialized and changed by subsequent
;;; operations.  Most Scheme systems provide structure definition
;;; facilities that automate creation of structure manipulation
;;; procedures, but we simply define the procedures by hand here.
(define make-tree
	(lambda (word)
	(vector word '() '() 1)))

(define tree-word (lambda (tree) (vector-ref tree 0)))

(define tree-left (lambda (tree) (vector-ref tree 1)))
(define set-tree-left!
	(lambda (tree new-left)
	(vector-set! tree 1 new-left)))

(define tree-right (lambda (tree) (vector-ref tree 2)))
(define set-tree-right!
	(lambda (tree new-right)
	(vector-set! tree 2 new-right)))

(define tree-count (lambda (tree) (vector-ref tree 3)))
(define set-tree-count!
	(lambda (tree new-count)
	(vector-set! tree 3 new-count)))

;;; If the word already exists in the tree, tree increments its
;;; count.  Otherwise, a new tree node is created and put into the
;;; tree.  In any case, the new or modified tree is returned.
(define tree
	(lambda (node word)
	(cond
		((null? node) (make-tree word))
		((string=? word (tree-word node))
		(set-tree-count! node (+ (tree-count node) 1))
		node)
		((string<? word (tree-word node))
		(set-tree-left! node (tree (tree-left node) word))
		node)
		(else
		(set-tree-right! node (tree (tree-right node) word))
		node))))

;;; tree-print prints the tree in "in-order," i.e., left subtree,
;;; then node, then right subtree.  For each word, the count and the
;;; word are printed on a single line.
(define tree-print
	(lambda (node p)
	(if (not (null? node))
		(begin
			(tree-print (tree-left node) p)
			(write (tree-count node) p)
			(write-char #\space p)
			(display (tree-word node) p)
			(newline p)
			(tree-print (tree-right node) p)))))

;;; frequency is the driver routine.  It opens the files, reads the
;;; words, and enters them into the tree.  When the input port
;;; reaches end-of-file, it prints the tree and closes the ports.
(define frequency
	(lambda (infn outfn)
	(let ((ip (open-input-file infn))
			(op (open-output-file outfn)))
		(let loop ((root '()))
		(let ((w (get-word ip)))
			(cond
				((eof-object? w) (tree-print root op))
				((string? w) (loop (tree root w)))
				(else (loop root)))))
		(close-input-port ip)
		(close-output-port op))))
\end{sourcecode}

\begin{sourcecode}{maple}{Ejemplo en Maple.}
restart:
with(geom3d):
eqS:=Equation(sphere(S,(x-1)^2 + (y-1)^2 +(z-1)^2 -121=0,[x,y,z],'centername'=T)):
L:=[]:
for x from -5 to 10  do
for y from -5 to 10  do
for z from -5 to 10  do  if  x<>xcoord(T) and y<>ycoord(T) and z <> zcoord(T) and   type(x,integer) and type(y,integer)and type(z,integer) and eqS then
L:=[op(L), [x,y,z]] fi;
od: od: od: 
nops(L);  
L;
\end{sourcecode}

\begin{sourcecode}{matlab}{Matlab 2.}
	function [f, psd, fft, fftcomp, envFormaModal, tlocMean, tlocStd, locMean, ...
    locStd, locFreq, maxlocs, pks, betaNodo, betaFreqNodo, fftmean, fftstd, ...
    psdmean, psdstd] = PSD(a, fs, gdl, varargin)
% PSD: Power Spectral Density. Calcula la FFT de un registro sismico
% analizado para varios nodos con distintos grados de libertad. La funcion
% permite calcular distintos tipos de periodos naturales, arrojando un
% vector. Adicionalmente se calcula una razon de amortiguamiento con cada
% modo.
%
% Input:
%   a               Registro de aceleracion
%   fs              Factor de muestreo del registro
%   gdl             Vector con los grados de libertad de los nodos en el analisis
%
% Parametros opcionales:
%   betaFFT         Realiza el calculo del amortiguamiento con FFT o PSD
%   betaFFTMax      El amortiguamiento se calcula con el maximo FFT de todos los nodos
%   peakFFT         Realiza el calculo de peaks con FFT o PSD
%   peakMinDistance Distancia minima entre peaks requerida
%   tukeywinr       Factor de la ventana de tukey
%   zerofill        Indica relleno de ceros para FFT
%
% Output:
%   f               Vector de frecuencias
%   psd             Cell PSD de cada registro de los nodos
%   fft             Cell FFT de cada registro de los nodos
%   fftcomp         Registro completo (Re+Im)
%   envFormaModal   Cell con formas modales para cada periodo
%   tlocMean        Periodos medios de cada modo obtenidos de los peaks
%   tlocStd         Desviacion estandar de cada modo obtenido de los peaks
%   locMean         Frecuencia media de cada modo obtenido en los peaks
%   locStd          Desviacion estandar de la frecuencia de cada modo
%   locFreq         Posicion en el vector de frecuencias de cada modo
%   maxlocs         Numero de modos encontrados en los peaks
%   pks             Vector de peaks
%   betaNodo        Vector de amortiguamientos por cada modo
%   betaFreqNodo    Valor del FFT objetivo por cada amortiguamiento modal
%   fftmean         Promedios FFT
%   fftstd          Desviacion estandar FFT
%   psdmean         Promedios PSD
%   psdstd          Desviacion estandar PSD

%% Parametros opcionales
p = inputParser;
p.KeepUnmatched = true;
addOptional(p, 'betaFFT', true);
addOptional(p, 'betaFFTMax', false); % Calcula el amortiguamiento con el maximo
addOptional(p, 'peakFFT', true);
addOptional(p, 'peakMinDistance', 0.5); % Requerido para el calculo
addOptional(p, 'tmax', -1);
addOptional(p, 'tmin', 0);
addOptional(p, 'tukeywinr', 0.01);
addOptional(p, 'zerofill', 0);
parse(p, varargin{:});
r = p.Results;

if r.betaFFT
    fprintf('\tSe calculan amortiguamientos con FFT\n');
else
    fprintf('\tSe calculan amortiguamientos con PSD\n');
end
if r.peakFFT
    fprintf('\tSe calculan peaks periodos con FFT\n');
else
    fprintf('\tSe calculan peaks periodos con PSD\n');
end

% Numero de grados de libertad
ng = length(gdl);

% Se obtiene la ventana de tiempo
c1 = 1;
if r.tmin ~= 0
    if r.tmin < 0
        error('El tiempo inferior no puede ser cero');
    end
    c1 = fix(r.tmin*fs);
end
cend = false;
if r.tmax ~= -1
    if r.tmin >= r.tmax
        error('El tiempo inferior tmin no puede ser mayor a tmax');
    end
    c2 = fix(r.tmax*fs);
else
    c2 = -1;
    cend = true;
end

%% Calcula la FFT
fft = cell(1, ng);
psd = cell(1, ng);
for k = 1:ng
    
    % Obtiene la aceleracion del grado de libertad analizado
    acc = a(gdl(k), :);
    
    % Limita la ventana
    if cend
        c2 = length(acc);
    end
    if c1 > length(acc)
        error('El tiempo inferior excede el largo del vector de aceleracion');
    end
    if c2 > length(acc)
        error('El tiempo superior excede el largo del vector de aceleracion');
    end
    acc = acc(c1:c2);
    
    % Rellena con ceros
    acc = [acc, zeros(1, floor(r.zerofill*length(acc)))]; %#ok<AGROW>
    tuck = tukeywin(length(acc), r.tukeywinr)';
    acctuck = acc .* tuck;
    [f, fftt, ~] = DFT(fs, acctuck);
    
    % Solo conservo la mitad
    tf = find(f == 0);
    f = f(tf:end);
    fftt = fftt(tf:end);
    fftcomp{k} = fftt; %#ok<AGROW> % Guarda el registro complejo
    fftt = abs(fftt); % O si no plot reclama
    fft{k} = fftt; % Guarda el registro
    
    % Calcula el PSD
    psd{k} = fftt.^2 / 2;
    
end % for k

%% Calcula el promedio y la desviacion estandar de los fft
fftmean = zeros(1, length(f));
fftstd = zeros(1, length(f));
fftdata = zeros(1, ng);
fftmax = zeros(1, ng);
for i = 1:length(f)
    for j = 1:ng % Recorre cada grado de libertad
        fftdata(j) = fft{j}(i);
    end % for j
    fftmean(i) = mean(fftdata);
    fftstd(i) = std(fftdata);
    fftmax(i) = max(fftdata);
end % for i

%% Calcula el promedio y la desviacion estandar de los psd
psdmean = zeros(1, length(f));
psdstd = zeros(1, length(f));
psddata = zeros(1, ng);
psdmax = zeros(1, ng);
for i = 1:length(f)
    for j = 1:ng % Recorre cada grado de libertad
        psddata(j) = psd{j}(i);
    end % for j
    psdmean(i) = mean(psddata);
    psdstd(i) = std(psddata);
    psdmax(i) = max(psddata);
end % for i

%% Calcula los peaks
locs = cell(1, ng);
maxlocs = 0;
for i = 1:ng
    if r.peakFFT
        [~, ploc] = findpeaks(fft{i}, f, ...
            'MinPeakDistance', r.peakMinDistance);
    else
        [~, ploc] = findpeaks(psd{i}, f, ...
            'MinPeakDistance', r.peakMinDistance);
    end
    % [maxtab, mintab] = peakdet(fft{i}, r.peakMinDistance, f);
    locs{i} = ploc;
    maxlocs = max(length(ploc), maxlocs);
end % for i

%% Calcula el promedio y la desviacion estandar de las frecuencias
locMean = zeros(1, maxlocs);
locStd = zeros(1, maxlocs);
locFreq = zeros(1, maxlocs); % Frecuencias (posicion)

% Calcula datos pero en periodos
tlocMean = zeros(1, maxlocs);
tlocStd = zeros(1, maxlocs);
for i = 1:maxlocs
    locData = []; % Datos para la posicion i
    tlocData = [];
    for k = 1:ng % Recorre cada nodo de analisis
        if i <= length(locs{k})
            locData = [locData, locs{k}(i)]; %#ok<AGROW>
            tlocData = [tlocData, 1 / locs{k}(i)]; %#ok<AGROW>
        end
    end % for k
    
    locMean(i) = mean(locData);
    locStd(i) = std(locData);
    
    % Estadistica para los periodos
    tlocMean(i) = mean(tlocData);
    tlocStd(i) = std(tlocData);
end % for i

%% Busca las posiciones de la frecuencia para locMean
j = 1; % Indice a locMean
for i = 1:length(f)
    if f(i) >= locMean(j)
        locFreq(j) = i;
        j = j + 1; % Avanza
        if j > maxlocs
            break;
        end
    end
end % for i

% Peaks periodos
if r.peakFFT
    pks = fftmax(locFreq);
else
    pks = psdmax(locFreq);
end

%% Calcula los amortiguamientos por cada periodo de cada nodo registrado
if r.betaFFT
    pksBeta = fftmax(locFreq);
else
    pksBeta = psdmax(locFreq);
end

betaNodo = cell(1, ng);
betaFreqNodo = cell(1, ng);

% Recorre cada registro
for k = 1:ng
    
    if ~r.betaFFTMax % Si se usan todos los registros
        if r.betaFFT
            ftNodo = fft{k};
        else
            ftNodo = psd{k};
        end
        pksNodo = ftNodo(locFreq);
        pksObj = pksNodo ./ sqrt(2);
    else % Si se usa solo el maximo
        if r.betaFFT
            ftNodo = fftmax;
        else
            ftNodo = psdmax;
        end
        pksObj = pksBeta ./ sqrt(2);
    end
    beta = zeros(1, maxlocs);
    betaFreq = cell(1, maxlocs);
    lastj = 1;
    
    % Recorre cada peak del nodo registrado
    for i = 1:length(pks)
        for j = lastj:length(f) - 1 % Al comenzar desde el punto anterior asegura que no se repitan frecuencias
            if (ftNodo(j) - pksObj(i)) * (ftNodo(j+1) - pksObj(i)) < 0 % Cruzo el objetivo en i              
                % Si el ultimo que cruzo fue superior a la frecuencia del peak
                % objetivo entonces este corresponde a la frecuencia derecha, y
                % el anterior a la izquierda
                if j > locFreq(i)
                    izq = f(lastj);
                    der = f(j);
                    lastj = j;
                    if r.betaFFT
                        beta(i) = (der - izq) / (der + izq);
                    else
                        beta(i) = (der^2 - izq^2) / (der^2 + izq^2);
                    end
                    betaFreq{i} = [izq, der, f(locFreq(i)), pksObj(i)];
                    break;
                end
                lastj = j + 1; % Ultimo en atravesar
            end
        end % for j
    end % for i
    
    % Guarda el resultado
    betaNodo{k} = beta;
    betaFreqNodo{k} = betaFreq;
    
    % Termina la ejecucion (k==1)
    if r.betaFFTMax
        betaNodo = betaNodo{1};
        betaFreqNodo = betaFreqNodo{1};
        break;
    end
    
end % for k

%% Calcula la envolvente de los peaks por cada una de las formas modales, usa FFT
envFormaModal = cell(1, maxlocs);
for k = 1:maxlocs
    envModo = zeros(1, ng);
    for i = 1:ng % Recorre cada nodo
        % Obtiene la fft asociada al periodo k del registro i
        envModo(i) = fft{i}(locFreq(k));
    end % for i
    envModo = envModo ./ max(envModo);
    envFormaModal{k} = envModo;
end % for k

end % PSD function
\end{sourcecode}

\begin{sourcecode}{verilog}{Ejemplo en Verilog.}
module Mixing {
	///////// ADC /////////
	inout              ADC_CS_N,
	output             ADC_DIN,
	input              ADC_DOUT,
	output             ADC_SCLK,
	
	///////// ADC /////////
	input              AUD_ADCDAT,
	inout              AUD_ADCLRCK,
	inout              AUD_BCLK,
	output             AUD_DACDAT,
	inout              AUD_DACLRCK,
	output             AUD_XCK,
	
	///////// clocks /////////
	input              clock2_50,
	input              clock3_50,
	input              clock4_50,
	input              clock_50,
	
	///////// HEX /////////
	output      [6:0]  HEX0,
	output      [6:0]  HEX1,
	output      [6:0]  HEX2,
	output      [6:0]  HEX3,
	output      [6:0]  HEX4,
	output      [6:0]  HEX5,
	
	///////// FOO /////////
	output      [2]    FOO,
}
\end{sourcecode}

\begin{sourcecode}{vhdl}{Ejemplo en VHDL.}
-- (this is a VHDL comment)

-- import std_logic from the IEEE library
library IEEE;
use IEEE.std_logic_1164.all;

-- this is the entity
entity ANDGATE is
	port (
		I1 : in std_logic;
		I2 : in std_logic;
		O  : out std_logic);
end entity ANDGATE;

-- this is the architecture
architecture RTL of ANDGATE is
begin
	O <= I1 and I2;
end architecture RTL;
\end{sourcecode}

\begin{sourcecode}{assemblerx64}{Assembler x64.}
cdqe 1, r8
push 1
add rsp, 4
push 1
\end{sourcecode}

\begin{sourcecode}{assemblerx86}{Assembler x86.}
; ---------------------------------------------
; Programa que imprime un string en la pantalla
; ---------------------------------------------
	.model small              ; modelo de memoria

	.stack                    ; segmento del stack

	.data                     ; segmento de datos
	Cadena1 DB 'Hola Mundo.$' ; string a imprimir (finalizado en $)

	.code                     ; segmento del código

; ---------------------------------------------
; Inicio del programa
; ---------------------------------------------
	programa:
		; ------------------------------------
		; inicia el segmento de datos
		; ------------------------------------
		MOV AX, @data          ; carga en AX la dirección del segmento de datos
		MOV DS, AX             ; mueve la dirección al registro de segmento por medio de AX
		
		; ------------------------------------
		; Imprime un string en pantalla
		; ------------------------------------
		MOV DX, offset Cadena1 ; mueve a DX la dirección del string a imprimir
		MOV AH, 9              ; AH = código para indicar al MS DOS que imprima en la pantalla, el string en DS:DX
		INT 21h                ; llamada al MS DOS para ejecutar la función (en este caso especificada en AH)
		
		; ------------------------------------
		; Finaliza el programa
		; ------------------------------------
		INT 20h                ; llamada al MS DOS para finalizar el programa

	end programa
\end{sourcecode}

\begin{sourcecode}{lua}{Ejemplo en Lua.}
-- defines a factorial function
function fact (n)
	if n == 0 then
		return 1
	else
		return n * fact(n-1)
	end
end

print("enter a number:")
a = io.read("*number")        -- read a number
print(fact(a))
\end{sourcecode}

\begin{sourcecode}{ini}{Configuración de los anchors.}
[anchor_parameters]
# Sizes should correlate to how the network processes an image.
sizes   = 32 64 128 256 512
# Strides should correlate to how the network strides over an image.
strides = 8 16 32 64 128
# The different ratios to use per anchor location.
ratios  = 0.5 1 2 3
# The different scaling factors to use per anchor location.
scales  = 1 1.2 1.6
\end{sourcecode}

\begin{sourcecode}{csv}{Archivo en csv.}
orand-car-with-bbs/training/images/0001_5179655_46066.jpg,83,8,126,61,4
orand-car-with-bbs/training/images/0001_5179655_46066.jpg,143,11,182,61,6
orand-car-with-bbs/training/images/0001_5179655_46066.jpg,204,11,247,54,0
orand-car-with-bbs/training/images/0001_5179655_46066.jpg,264,15,307,58,6
orand-car-with-bbs/training/images/0001_5179655_46066.jpg,329,11,365,62,6
orand-car-with-bbs/training/images/0002_5179657_53862.jpg,80,11,114,56,5
orand-car-with-bbs/training/images/0002_5179657_53862.jpg,116,13,150,62,3
orand-car-with-bbs/training/images/0002_5179657_53862.jpg,166,15,205,68,8
\end{sourcecode}

\begin{sourcecode}{make}{Ejemplo en make.}
all: helloworld variables if_else while_for functions strings pointer reference complexv1 complexv1sep complexv2 complexv3 stack tests

helloworld:
g++ 01_helloworld.cpp -o 01_helloworld

variables:
g++ 02_variables.cpp -o 02_variables

if_else:
g++ 03_if_else.cpp -o 03_if_else

while_for:
g++ 04_while_for.cpp -o 04_while_for
\end{sourcecode}

\begin{sourcecode}{cmake}{Ejemplo en cmake.}
# Propiedades del make
cmake_minimum_required(VERSION 3.10)
project(CC7515-AUX-1)
set(CMAKE_CXX_STANDARD 11)

# Crea un set, contiene varios archivos que permiten compilar al principal
set(AUX1_EX10
src/10_complex_v1_sep.cpp)

set(AUX1_STACK
src/13_stack.h
src/12_complex_v3.cpp)

# Define ejecutables
add_executable(EX-1-HELLO-WORLD src/01_helloworld.cpp)
add_executable(EX-2-VARIABLES src/02_variables.cpp)
add_executable(EX-3-IF-ELSE src/03_if_else.cpp)
add_executable(EX-4-WHILE-FOR src/04_while_for.cpp)
...
\end{sourcecode}

\begin{sourcecode}{cmake}{Ejemplo en cmake 2.}
cmake_minimum_required (VERSION 2.8)
cmake_policy(VERSION 2.8)

project (LearnOpenGL)

IF(NOT CMAKE_BUILD_TYPE)
  SET(CMAKE_BUILD_TYPE Debug CACHE STRING "Choose the type of build (Debug or Release)" FORCE)
ENDIF(NOT CMAKE_BUILD_TYPE)

set(CMAKE_MODULE_PATH ${CMAKE_MODULE_PATH} "${CMAKE_SOURCE_DIR}/cmake/modules/")

if(WIN32)
	set(CMAKE_RUNTIME_OUTPUT_DIRECTORY ${CMAKE_SOURCE_DIR}/bin)
endif(WIN32)

link_directories(${CMAKE_SOURCE_DIR}/lib)

list(APPEND CMAKE_CXX_FLAGS "-std=c++11")

# find the required packages
find_package(GLM REQUIRED)
message(STATUS "GLM included at ${GLM_INCLUDE_DIR}")
find_package(GLFW3 REQUIRED)
message(STATUS "Found GLFW3 in ${GLFW3_INCLUDE_DIR}")
find_package(ASSIMP REQUIRED)
message(STATUS "Found ASSIMP in ${ASSIMP_INCLUDE_DIR}")
# find_package(SOIL REQUIRED)
# message(STATUS "Found SOIL in ${SOIL_INCLUDE_DIR}")
# find_package(GLEW REQUIRED)
# message(STATUS "Found GLEW in ${GLEW_INCLUDE_DIR}")

if(WIN32)
  set(LIBS glfw3 opengl32 assimp freetype irrKlang)
elseif(UNIX AND NOT APPLE)
  set(CMAKE_CXX_FLAGS_DEBUG "${CMAKE_CXX_FLAGS_DEBUG} -Wall")
  find_package(OpenGL REQUIRED)
  add_definitions(${OPENGL_DEFINITIONS})
  find_package(X11 REQUIRED)
  # note that the order is important for setting the libs
  # use pkg-config --libs $(pkg-config --print-requires --print-requires-private glfw3) in a terminal to confirm
  set(LIBS ${GLFW3_LIBRARY} X11 Xrandr Xinerama Xi Xxf86vm Xcursor GL dl pthread freetype ${ASSIMP_LIBRARY})
  set (CMAKE_CXX_LINK_EXECUTABLE "${CMAKE_CXX_LINK_EXECUTABLE} -ldl")
elseif(APPLE)
  INCLUDE_DIRECTORIES(/System/Library/Frameworks)
  FIND_LIBRARY(COCOA_LIBRARY Cocoa)
  FIND_LIBRARY(OpenGL_LIBRARY OpenGL)
  FIND_LIBRARY(IOKit_LIBRARY IOKit)
  FIND_LIBRARY(CoreVideo_LIBRARY CoreVideo)
  MARK_AS_ADVANCED(COCOA_LIBRARY OpenGL_LIBRARY)
  SET(APPLE_LIBS ${COCOA_LIBRARY} ${IOKit_LIBRARY} ${OpenGL_LIBRARY} ${CoreVideo_LIBRARY})
  SET(APPLE_LIBS ${APPLE_LIBS} ${GLFW3_LIBRARY} ${ASSIMP_LIBRARY})
  set(LIBS ${LIBS} ${APPLE_LIBS})
else()
  set(LIBS )
endif(WIN32)

set(CHAPTERS
    1.getting_started
    2.lighting
    3.model_loading
    4.advanced_opengl
    5.advanced_lighting
    6.pbr
    7.in_practice
)

set(1.getting_started
    1.1.hello_window
    1.2.hello_window_clear
    2.1.hello_triangle
    2.2.hello_triangle_indexed
    2.3.hello_triangle_exercise1
    2.4.hello_triangle_exercise2
    2.5.hello_triangle_exercise3
    3.1.shaders_uniform
    3.2.shaders_interpolation
    3.3.shaders_class
    4.1.textures
    4.2.textures_combined
    4.4.textures_exercise2
    4.5.textures_exercise3
    4.6.textures_exercise4
    5.1.transformations
    5.2.transformations_exercise2
    6.1.coordinate_systems
    6.2.coordinate_systems_depth
    6.3.coordinate_systems_multiple
    7.1.camera_circle
    7.2.camera_keyboard_dt
    7.3.camera_mouse_zoom
    7.4.camera_class
)

set(2.lighting
    1.colors
    2.1.basic_lighting_diffuse
    2.2.basic_lighting_specular
    3.1.materials
    3.2.materials_exercise1
    4.1.lighting_maps_diffuse_map
    4.2.lighting_maps_specular_map
    4.4.lighting_maps_exercise4
    5.1.light_casters_directional
    5.2.light_casters_point
    5.3.light_casters_spot
    5.4.light_casters_spot_soft
    6.multiple_lights
)

set(3.model_loading
    1.model_loading
)

set(4.advanced_opengl
    1.1.depth_testing
    1.2.depth_testing_view
    2.stencil_testing
    3.1.blending_discard
    3.2.blending_sort
    5.1.framebuffers
    5.2.framebuffers_exercise1
    6.1.cubemaps_skybox
    6.2.cubemaps_environment_mapping
    8.advanced_glsl_ubo
    9.1.geometry_shader_houses
    9.2.geometry_shader_exploding
    9.3.geometry_shader_normals
    10.1.instancing_quads
    10.2.asteroids
    10.3.asteroids_instanced
    11.1.anti_aliasing_msaa
    11.2.anti_aliasing_offscreen
)

set(5.advanced_lighting
    1.advanced_lighting
    2.gamma_correction
    3.1.1.shadow_mapping_depth
    3.1.2.shadow_mapping_base
    3.1.3.shadow_mapping
    3.2.1.point_shadows
    3.2.2.point_shadows_soft
    4.normal_mapping
    5.1.parallax_mapping
    5.2.steep_parallax_mapping
    5.3.parallax_occlusion_mapping
    6.hdr
    7.bloom
    8.1.deferred_shading
    8.2.deferred_shading_volumes
    9.ssao
)

set(6.pbr
    1.1.lighting
    1.2.lighting_textured
    2.1.1.ibl_irradiance_conversion
    2.1.2.ibl_irradiance
    2.2.1.ibl_specular
    2.2.2.ibl_specular_textured
)

set(7.in_practice
    1.debugging
    2.text_rendering
    #3.2d_game
)



configure_file(configuration/root_directory.h.in configuration/root_directory.h)
include_directories(${CMAKE_BINARY_DIR}/configuration)

# first create relevant static libraries requried for other projects
add_library(STB_IMAGE "src/stb_image.cpp")
set(LIBS ${LIBS} STB_IMAGE)

add_library(GLAD "src/glad.c")
set(LIBS ${LIBS} GLAD)

macro(makeLink src dest target)
  add_custom_command(TARGET ${target} POST_BUILD COMMAND ${CMAKE_COMMAND} -E create_symlink ${src} ${dest}  DEPENDS  ${dest} COMMENT "mklink ${src} -> ${dest}")
endmacro()

# then create a project file per tutorial
foreach(CHAPTER ${CHAPTERS})
    foreach(DEMO ${${CHAPTER}})
        file(GLOB SOURCE
            "src/${CHAPTER}/${DEMO}/*.h"
            "src/${CHAPTER}/${DEMO}/*.cpp"
            "src/${CHAPTER}/${DEMO}/*.vs"
            "src/${CHAPTER}/${DEMO}/*.fs"
            "src/${CHAPTER}/${DEMO}/*.gs"
        )
        set(NAME "${CHAPTER}__${DEMO}")
        add_executable(${NAME} ${SOURCE})
        target_link_libraries(${NAME} ${LIBS})
        if(WIN32)
            set_target_properties(${NAME} PROPERTIES RUNTIME_OUTPUT_DIRECTORY "${CMAKE_SOURCE_DIR}/bin/${CHAPTER}")
            set_target_properties(${NAME} PROPERTIES VS_DEBUGGER_WORKING_DIRECTORY "${CMAKE_SOURCE_DIR}/bin/${CHAPTER}/Debug")
        elseif(UNIX AND NOT APPLE)
            set_target_properties(${NAME} PROPERTIES RUNTIME_OUTPUT_DIRECTORY "${CMAKE_SOURCE_DIR}/bin/${CHAPTER}")
        elseif(APPLE)
            set_target_properties(${NAME} PROPERTIES RUNTIME_OUTPUT_DIRECTORY "${CMAKE_SOURCE_DIR}/bin/${CHAPTER}")
            set_target_properties(${NAME} PROPERTIES RUNTIME_OUTPUT_DIRECTORY_DEBUG "${CMAKE_SOURCE_DIR}/bin/${CHAPTER}")
            set_target_properties(${NAME} PROPERTIES RUNTIME_OUTPUT_DIRECTORY_RELEASE "${CMAKE_SOURCE_DIR}/bin/${CHAPTER}")
        endif(WIN32)
        # copy shader files to build directory
        file(GLOB SHADERS
                 "src/${CHAPTER}/${DEMO}/*.vs"
                 # "src/${CHAPTER}/${DEMO}/*.frag"
                 "src/${CHAPTER}/${DEMO}/*.fs"
                 "src/${CHAPTER}/${DEMO}/*.gs"
        )
        foreach(SHADER ${SHADERS})
            if(WIN32)
                # configure_file(${SHADER} "test")
                add_custom_command(TARGET ${NAME} PRE_BUILD COMMAND ${CMAKE_COMMAND} -E copy ${SHADER} $<TARGET_FILE_DIR:${NAME}>)
            elseif(UNIX AND NOT APPLE)
                file(COPY ${SHADER} DESTINATION ${CMAKE_SOURCE_DIR}/bin/${CHAPTER})
            elseif(APPLE)
                # create symbolic link for *.vs *.fs *.gs
                get_filename_component(SHADERNAME ${SHADER} NAME)
                makeLink(${SHADER} ${CMAKE_SOURCE_DIR}/bin/${CHAPTER}/${SHADERNAME} ${NAME})
            endif(WIN32)
        endforeach(SHADER)
        # if compiling for visual studio, also use configure file for each project (specifically to set up working directory)
        if(MSVC)
            configure_file(${CMAKE_SOURCE_DIR}/configuration/visualstudio.vcxproj.user.in ${CMAKE_CURRENT_BINARY_DIR}/${NAME}.vcxproj.user @ONLY)
        endif(MSVC)
    endforeach(DEMO)
endforeach(CHAPTER)

include_directories(${CMAKE_SOURCE_DIR}/includes)
\end{sourcecode}

% \clearpage
% \importsourcecode{latex}{test/general.tex}{Carga código fuente archivo LaTeX.}
% \importsourcecode{latex}{lib/etc/example.tex}{Carga código fuente ejemplo LaTeX.}
% \importsourcecode{latex}{lib/cfg/init.tex}{Carga código fuente init LaTeX.}